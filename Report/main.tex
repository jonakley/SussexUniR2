\documentclass{article}


\usepackage{arxiv}

\usepackage[utf8]{luainputenc}	% allow utf-8 input with LuaLaTeX
\usepackage[T1]{fontenc}    	% use 8-bit T1 fonts
\usepackage{palatino}			% use palatino font
\usepackage{hyperref}       	% hyperlinks
\usepackage{url}            	% simple URL typesetting
\usepackage{booktabs}       	% professional-quality tables
\usepackage{amsfonts}       	% blackboard math symbols
\usepackage{amsthm}
\usepackage{nicefrac}       	% compact symbols for 1/2, etc.
\usepackage{microtype}      	% microtypography
\usepackage{amsmath}
\usepackage[compat=1.1.0]{tikz-feynman}
\usepackage{tikz}
\usepackage{slashed}
\usepackage{mathtools}
\usepackage{dsfont}
\usepackage{geometry}
\usepackage{caption}
\usepackage[onehalfspacing]{setspace}
\usepackage{titlesec}
\usepackage[stable]{footmisc}	% suppress footnotes in table of contents

% fix wrong title spacing induced by onehalfspacing from setspace package
\titlespacing*{\section}
{0pt}{0ex}{0ex}
\titlespacing*{\subsection}
{0pt}{0ex}{0ex}
\titlespacing*{\subsubsection}
{0pt}{0ex}{0ex}

% define tick
\def\checkmark{\tikz\fill[scale=0.4](0,.35) -- (.25,0) -- (1,.7) -- (.25,.15) -- cycle;}  


\numberwithin{equation}{section} % numbers equations by section

\newcommand{\Tr}{\mathrm{Tr}}
\newcommand{\wtilde}{\widetilde}
\newcommand{\fsl}{\slashed}
\newcommand{\lbr}{\left\lbrace}
\newcommand{\rbr}{\right\rbrace}
\newcommand*\circled[1]{\tikz[baseline=(char.base)]{
            \node[shape=circle,draw,inner sep=0.5pt] (char) {#1};}}
\newcommand{\numberthis}{\addtocounter{equation}{1}\tag{\theequation}} % number equation in unnumbered align/equation environment
\newcommand{\numberitem}{\hfill\refstepcounter{equation}\textup{(\theequation)}}




\title{Contribution of QED to Rational Terms in 1-loop Feynman Diagrams in the Standard Model}


\author{
  Jonathan Kley \\
  Department of Physics\\
  Technical University of Munich\\
  James-Franck-Str. 1, 85748 Garching \\
  \texttt{jonathan.kley@tum.de} \\
}

\begin{document}

\maketitle

\begin{abstract}
For years it has been of high interest to automate the calculation of Feynman diagrams in the Standard Model (SM) in order to be able to keep up with the increasing precision of experimental data. A lot of efforts have been made in the early 2000s to make progress in this direction where one of the big milestones was the proof that any amplitude in the SM can be decomposed into a box-, a triangle-, a bubble-, a tadpole-subdiagram and a rational term which can be of two types. In this paper we investigate the $\epsilon$-dimensional contribution of dimensional regularisation to this rational term. If all of the coefficients in the composition into the mentioned subdiagrams alongside the rational terms are known we obtain an effective treelike theory giving us 1-loop results for the whole SM. \\
This paper is structured as follows: After a short introduction to the subject, we first calculate $R_2$ in pure QED as a warm-up and then the QED contribution to $R_2$ in the SM. Then, we renormalize QED in terms of scalar integrals and calculate the contribution of QED to the renormalization of the SM. Once this is done, we are able to implement an algorithm to automate calculations in the SM and can compare our results to the results from other references.\end{abstract}


% keywords can be removed
\keywords{QFT \and 1-loop Feynman Diagrams \and Rational Terms \and More}

\newpage
\tableofcontents
\newpage

\section{Introduction}
\label{sec:Introduction} 

The introduction goes here.

\begin{equation}
\mathcal{M} = \sum_i d_i \mathrm{Box}_i + \sum_i c_i \mathrm{Triangle}_i + \sum_i b_i \mathrm{Bubble}_i + \sum_i a_i \mathrm{Tadpole}_i + R
\end{equation}

\begin{equation}
R = R_1 + R_2
\end{equation}
where $R_2$ is the $\epsilon$-dimensional contribution of dimensional regularization to the amplitude which is just a rational combination of Lorentz tensors and parameters of the theory, i.e. the couplings or masses of the particles in the theory.


We can decompose any m-point 1-loop function $\bar{A}\bar{q})$ in a numerator $\bar{N}( \bar{q})$ and denominators $\bar{D}_i$
\begin{equation}
\label{eqn:amp}
\bar{A} (\bar{q}) = \frac{\bar{N}(\bar{q})}{\bar{D}_0\bar{D}_1\cdots\bar{D}_{m-1}}, \qquad \bar{D}_i = \left( \bar{q} + \bar{p}_i \right)^2 - m_i^2, p_0 \neq 0
\end{equation}
where $\bar{q}$ is the $d$-dimensional loop momentum and $m_i$ is the mass of the particle corresponding to the propagator with the numerator $D_i$. The $d$-dimensional numerator function $\bar{N}(\bar{q})$ can be split in a 4-dimensional and an $\epsilon$-dimensional part 
\begin{equation}
\bar{N}( \bar{q}) = N (q) + \tilde{N} (\tilde{q}^2,q,\epsilon)
\end{equation}
where only $\tilde{N} (\tilde{q}^2,q,\epsilon)$ is interesting to us because it appears in the definitions of rational terms of the form $R_2$ which are defined as
\begin{equation}
R_2 \equiv \frac{1}{\left( 2\pi \right)^4} \int d^d \bar{q} \frac{\tilde{N} ( \tilde{q}^2,q,\epsilon)}{\bar{D}_0\bar{D}_1\cdots\bar{D}_{m-1}}
\end{equation}
which is the $\epsilon$-dimensional contribution to the amplitude in eqn. \ref{eqn:amp} integrated over the $d$-dimensional loop momentum $\bar{q}$. It can be obtained by splitting the $d$-dimensional Lorentz tensors appearing in the amplitude into a 4-dimensional and an $\epsilon$-dimensional part
\begin{equation}
\bar{A}^{\mu_1\dots\mu_n} = A^{\mu_1\dots\mu_n} + \tilde{A}^{\mu_1\dots\mu_n}.
\end{equation}
To simplify our calculations later, we can establish a few identities for the manipulation of $d$-dimensional momenta. If we contract a $d$-dimensional object with an observable Lorentz tensor (like the momentum of an external particle) only the 4-dimensional part survives, e.g. for a loop momentum $\bar{q}^{\mu}$ and an external momentum $p^{\mu}$
\begin{equation}
\bar{q}\cdot p = q \cdot p.
\end{equation}
Thus, if an amplitude transforms with indices $\mu_1,\dots,\mu_n$ under a Lorentz transformation, the tensors in the amplitude bearing these indices will only appear as 4-dimensional.

e.g. in reference \cite{R2QCD} and \cite{R2QED}.

Since, we want to perform calculations in QED which contains a fermion, we have to extend the Clifford algebra $\lbr \gamma^{\mu},\gamma^{\nu} \rbr = 2g^{\mu\nu} \mathds{1}_4$ to $d$ dimensions. This is straightforward by promoting $\gamma^{\mu} \rightarrow \bar{\gamma}^{\mu}$ and extending the Minkowski metric to d dimensions by adding additional -1 on the diagonal for the extra spatial dimensions. We have
\begin{equation}
\lbr \bar{\gamma}^{\mu},\bar{\gamma}^{\nu} \rbr = 2\bar{g}^{\mu\nu} \mathds{1}_d
\end{equation}
If we want to preserve the Clifford algebra separately in 4 and $\epsilon$ dimensions this implies
\begin{equation}
\lbr \gamma^{\mu},\tilde{\gamma}^{\nu} \rbr = 0
\end{equation}
As opposed to QED the Standard Model is a chiral theory, i.e. it couples differently to left- and right-handed currents. This means that also axial-vector currents appear in the theory which are formulated with the fifth gamma matrix. The extension of $\gamma_5$ to $d$ dimensions is not as straightforward as with the four gamma matrices. This is because chirality is a property of four dimensions.\\
If we also want to impose $\lbr \gamma_5,\gamma^{\mu} \rbr = 0$ for $d \neq 4$, then $\Tr\left( \gamma_5 \gamma^{\alpha}\gamma^{\beta}\gamma^{\gamma}\gamma^{\delta} \right) = 0$  for $d \neq 0,2,4$ which clashes with $\Tr\left( \gamma_5 \gamma^{\alpha}\gamma^{\beta}\gamma^{\gamma}\gamma^{\delta} \right) = -4i \epsilon^{\alpha\beta\gamma\delta}$ \cite{Gamma5}. But the identity is essential in the evaluation of the triangle diagram for the Adler-Bell-Jackiw anomaly. The only definition of $\gamma_5$ which is consistent with the chiral anomaly is the definition of 't Hooft and Veltman \cite{HVgamma5}: $\gamma_5 = i/4! \ \epsilon_{\mu_1 \dots \mu_4} \gamma^{\mu_1} \cdots \gamma^{\mu_4}$. This definition implies
\begin{equation}
\lbr \gamma_5, \gamma^{\mu} \rbr = 0  \text{ and }  \left[ \gamma_5, \tilde{\gamma}^{\mu} \right] = 0.
\end{equation} 




\section{R$_2$ in Pure QED}
\label{sec:pureQED} 
Before we calculate anything in the Standard Model, let us start with the $R_2$ of pure QED. We have to consider all n-point functions up to $n=4$ which are allowed by the Feynman rules of QED and calculate their contribution to equation \ref{eqn:R2}.
\subsection{2-point functions}
The Feynman rules of QED allow two 2-point functions; the self-energy diagrams of the photon and the electron. Let us start with the photon self-energy which has the simplest Lorentz structure and therefore an easy to evaluate numerator function. \\

{\bf Photon self-energy} \\
The photon 2-point function is given by
\begin{align*}
\begin{gathered}
\feynmandiagram [layered layout, horizontal=b to c] {
	a [particle=\(\alpha\)] -- [photon, momentum=\(p_1\)] b
	  -- [fermion, half left, looseness=1.5, momentum=\(p_1+q\)] c
	  -- [fermion, half left, looseness=1.5, momentum=\(q\)] b,
	c -- [photon, momentum=\(p_1\)] d [particle=\(\beta\)] ,
};
\end{gathered} \qquad
& =\int\frac{d^d\bar{q}}{\left( 2\pi \right)^d} \left( -1 \right) \mathrm{Tr} \lbr ie \bar{\gamma}^{\alpha} \frac{i \left( \bar{\fsl{p}}_1+\bar{\fsl{q}}+m \right)}{\left( p_1+q \right)^2 - m^2} ie \bar{\gamma}^{\beta} \frac{i \left( \bar{\fsl{q}}+m \right)}{q^2 - m^2} \rbr & \\
& \equiv \int\frac{d^d\bar{q}}{\left( 2\pi \right)^d} \frac{\bar{N}}{\bar{D}_1\bar{D}_0} &
\end{align*}
Where we defined the numerator and denominator functions in the last step. Now we can extract the $\epsilon$-dimensional contribution from the numerator
\begin{equation*}
\bar{N} (\bar{q}) = - e^2 \ \mathrm{Tr} \lbr \bar{\gamma}^{\alpha} \left( \bar{\fsl{p}}_1 + \bar{\fsl{q}} + m \right) \bar{\gamma}^{\beta} \left( \bar{\fsl{q}} + m \right) \rbr = - e^2 \ \mathrm{Tr} \lbr \gamma^{\alpha} \left( \fsl{p}_1 + \fsl{q} + m \right) \gamma^{\beta} \left( \fsl{q} + m \right) + \gamma^{\alpha} \wtilde{\fsl{q}} \gamma^{\beta} \wtilde{\fsl{q}} \rbr \equiv N + \wtilde{N}
\end{equation*}
Here, the first term is the 4-dimensional numerator which also appears in normal loop calculations and the second term is the $\epsilon$-dimensional part which we need for the calculation of $R_2$.\\
We can now evaluate the trace and get
\begin{equation*}
\wtilde{N} = -e^2 \ \mathrm{Tr} \lbr \gamma^{\alpha} \wtilde{\fsl{q}} \gamma^{\beta} \wtilde{\fsl{q}} \rbr = 4 e^2 \wtilde{q}^2 g^{\alpha\beta}
\end{equation*}
Where we have used that the $\epsilon$-dimensional gamma matrices anti-commute with the 4-dimensional gamma matrices and the trace identity for 2 gamma matrices (equation \ref{eqn:Tr2g}) which can be found alongside a proof in appendix \ref{app:Traceology}. Plugging the expression for $\wtilde{N}$ in equation \ref{eqn:R2} gives
\begin{equation}
\label{eqn:R2photon}
R_2^{\gamma\gamma} = \frac{1}{\left( 2\pi \right) ^4} \int d^d\bar{q} \frac{\wtilde{N}}{\bar{D}_1\bar{D}_0} = \frac{4e^2}{16\pi^4} \underbrace{\int d^d\bar{q} \frac{\wtilde{q}^2}{\bar{D}_1\bar{D}_0}}_{-i\frac{\pi}{2} \left( 2m^2 - p_1^2/3 \right)} = \frac{-ie^2}{8\pi^2} g^{\alpha\beta} \left( 2m^2 -\frac{p_1^2}{3} \right)
\end{equation}\\
In the last step we have used the 2-point integral \ref{eqn:2ptintq2}.

{\bf Electron self-energy} \\
The other 2-point function in QED is the electron 2-point function which is given by
\begin{align*}
\begin{gathered}
\feynmandiagram [layered layout, horizontal=b to c] {
	a -- [fermion, momentum=\(p_1\)] b,
	c -- [photon, half left, looseness=1.5, momentum=\(q\)] b,
	b -- [fermion, momentum=\(p_1+q\)] c,
	c -- [fermion, momentum=\(p_1\)] d,
};
\end{gathered} \qquad
& =\int\frac{d^dq}{\left( 2\pi \right)^d} ie \gamma^{\alpha} \frac{i \left( \fsl{p}_1+\fsl{q}+m \right)}{\left( p_1+q \right)^2 - m^2} ie \gamma^{\beta} \frac{-i g_{\alpha\beta}}{q^2} =\int\frac{d^dq}{\left( 2\pi \right)^d} \left( -e^2 \right) \gamma^{\alpha} \frac{\left( \fsl{p}_1+\fsl{q}+m \right)}{\left( p_1+q \right)^2 - m^2} \gamma_{\alpha} \frac{1}{q^2} & \\
& \equiv \int\frac{d^dq}{\left( 2\pi \right)^d} \frac{\bar{N}}{\bar{D}_1\bar{D}_0} &
\end{align*}
Now we extract again the $\epsilon$-dimensional part from the numerator function we defined in the last step. We get
\begin{equation*}
\bar{N} (\bar{q}) = \left( -e^2 \right) \bar{\gamma}^{\alpha} \left( \bar{\fsl{p}}_1 + \bar{\fsl{q}} + m \right) \bar{\gamma}_{\alpha} = -e^2 \lbr \gamma^{\alpha} \left( \bar{\fsl{p}}_1 + \bar{\fsl{q}} + m \right) \gamma_{\alpha} + \wtilde{\gamma}^{\alpha} \left( \bar{\fsl{p}}_1 + \bar{\fsl{q}} + m \right) \wtilde{\gamma}_{\alpha} + \gamma^{\alpha} \wtilde{\fsl{q}} \gamma_{\alpha} + \wtilde{\gamma}^{\alpha} \wtilde{\fsl{q}} \wtilde{\gamma}_{\alpha} \rbr \equiv N + \wtilde{N}
\end{equation*}
Here, the first term is again the normal 4-dimensional numerator and the rest the $\epsilon$-dimensional part we are interested in. It can be simplified further
\begin{equation*}
\wtilde{N} = -e^2 \lbr \wtilde{\gamma}^{\alpha} \left( \bar{\fsl{p}}_1 + \bar{\fsl{q}} + m \right) \wtilde{\gamma}_{\alpha} + \gamma^{\alpha} \wtilde{\fsl{q}} \gamma_{\alpha} + \wtilde{\gamma}^{\alpha} \wtilde{\fsl{q}} \wtilde{\gamma}_{\alpha} \rbr = -e^2 \lbr - \underbrace{\wtilde{\gamma}^{\alpha} \wtilde{\gamma}_{\alpha}}_{= \epsilon} \left( \bar{\fsl{p}}_1 + \bar{\fsl{q}} - m \right) - \underbrace{\gamma^{\alpha} \gamma_{\alpha}}_{=4} \wtilde{\fsl{q}} + \wtilde{\gamma}^{\alpha} \wtilde{\fsl{q}} \wtilde{\gamma}_{\alpha} \rbr
\end{equation*}
where we have used again that the $\epsilon$-dimensional gamma matrices anti-commute with the 4-dimensional gamma matrices and equation \ref{eqn:2g} to simplify the expression. Plugging $\tilde{N}$ into the definition of $R_2$ we get
\begin{align*}
R_2^{\mathrm{ee}} & = \frac{1}{\left( 2\pi \right) ^4} \int d^d\bar{q} \frac{\wtilde{N}}{\bar{D}_1\bar{D}_0} = \frac{-e^2}{\left( 2\pi \right) ^4} \int d^d\bar{q} \frac{1}{\bar{D}_1\bar{D}_0} \left( -\epsilon \left( \fsl{p}_1 + \fsl{q} - m \right) + \underbrace{\wtilde{\fsl{q}} \left( \ldots \right)}_{=0} \right) = & \\ 
& = \frac{e^2}{\left( 2\pi \right) ^4} \lbr \underbrace{\int d^d\bar{q} \frac{\epsilon \left( \fsl{p}_1 - m \right)}{\bar{D}_1\bar{D}_0}}_{=-2\epsilon \frac{i \pi^2}{\epsilon}  \left( \fsl{p}_1 -m \right)}  + \underbrace{\int d^d\bar{q} \frac{\epsilon \fsl{q}}{\bar{D}_1\bar{D}_0}}_{=\epsilon \frac{i\pi^2}{\epsilon} \fsl{p}_1} \rbr = \frac{e^2}{\left( 2\pi \right) ^4} \epsilon \frac{i\pi^2}{\epsilon} \left( \left( -2 \right) \left( \fsl{p}_1 - m \right) + \fsl{p}_1 \right) = \frac{-ie^2}{16 \pi^2}  \left( \fsl{p}_1 - 2m \right) & \numberthis \label{eqn:R2ee}
\end{align*}
Where we have used that the integral over an odd function in q integrated over the whole space vanishes at the end of the first line. We also used the 2-point integrals \ref{eqn:2ptintsca} and  \ref{eqn:2ptintq}. \\
These are all the 2-point functions that are allowed by the Feynman rules of QED. So we continue with the 3-point functions now.

\subsection{3-point functions}
There are two possible 3-point functions, the 1PI contribution to the electron-photon vertex at 1-loop in QED and the 3-photon triangle diagram. Let us start with the electron-photon vertex which is given by
\begin{align*}
\begin{gathered}
\feynmandiagram [horizontal=i1 to v2] {
	i1 [particle=\mu] -- [photon, momentum=\(p_2-p_1\)] v2,
	f1 -- [fermion, momentum=\(p_1\)] v1
	   -- [fermion, momentum=\(p_1+q\)] v2
	   -- [fermion, momentum=\(p_2+q\)] v3
	   -- [fermion, momentum=\(p_2\)] f3,
	v3 -- [boson, momentum=\(q\)] v1, 
};
\end{gathered} \qquad
& =\int\frac{d^dq}{\left( 2\pi \right)^d} ie \gamma^{\beta} \frac{i \left( \fsl{p}_1+\fsl{q}+m \right)}{\left( p_1+q \right)^2 - m^2} ie \gamma^{\mu} \frac{i \left( \fsl{p}_2 + \fsl{q}+m \right)}{\left( p_2 + q \right)^2 - m^2} ie \gamma^{\alpha} \frac{-ig_{\alpha\beta}}{q^2} & \\
& \equiv \int\frac{d^dq}{\left( 2\pi \right)^d} \frac{\bar{N}}{\bar{D}_1\bar{D}_2\bar{D}_0} &
\end{align*}
We get for $\bar{N}$
\begin{align*}
\bar{N} (\bar{q}) & = e^3 \lbr \bar{\gamma}^{\beta} \left( \bar{\fsl{p}}_1 + \bar{\fsl{q}} + m \right) \bar{\gamma}^{\mu} \left( \bar{\fsl{p}}_2 + \bar{\fsl{q}} + m \right) \bar{\gamma}_{\beta} \rbr = e^3 \lbr \gamma^{\beta} \left( \fsl{p}_1 + \fsl{q} + m \right) \gamma^{\mu} \left( \fsl{p}_2 + \fsl{q} + m \right) \gamma_{\beta} + \right. & \\
& \left. + \wtilde{\gamma}^{\beta} \left( \fsl{p}_1 + \fsl{q} + m \right) \gamma^{\mu} \left( \fsl{p}_2 + \fsl{q} + m \right) \wtilde{\gamma}_{\beta} + \underbrace{\gamma^{\beta} \wtilde{\fsl{q}} \gamma^{\mu} \wtilde{\fsl{q}} \gamma_{\beta}}_{\equiv \ \circled{1}} + \underbrace{\wtilde{\gamma}^{\beta} \wtilde{\fsl{q}} \gamma^{\mu} \wtilde{\fsl{q}} \wtilde{\gamma}_{\beta}}_{\equiv \ \circled{2}}  \rbr \equiv N + \wtilde{N} &
\end{align*}
Where the last 3 terms define $\wtilde{N}$. Let us work on $\circled{1}$ and $\circled{2}$ separately. Using the fact that 4-dimensional and $\epsilon$-dimensional gamma matrices anticommute and equations \ref{eqn:3g} and \ref{eqn:fsldot} we get
\begin{equation*}
\circled{1} = \wtilde{q}_{\rho}\wtilde{q}_{\sigma} \gamma^{\beta} \wtilde{\gamma}^{\rho} \gamma^{\mu} \wtilde{\gamma}^{\sigma} \gamma_{\beta} = \wtilde{q}_{\rho}\wtilde{q}_{\sigma} \left( -1 \right)^3 \wtilde{\gamma}^{\rho} \wtilde{\gamma}^{\sigma} \gamma^{\beta} \gamma^{\mu} \gamma_{\beta} = 2\wtilde{\fsl{q}}\wtilde{\fsl{q}}\gamma^{\mu} = 2 \wtilde{q}^2 \gamma^{\mu}
\end{equation*}
And for the other term
\begin{equation*}
\circled{2} = \wtilde{q}_{\rho}\wtilde{q}_{\sigma} \wtilde{\gamma}^{\beta} \wtilde{\gamma}^{\rho} \gamma^{\mu} \wtilde{\gamma}^{\sigma} \wtilde{\gamma}_{\beta} = \wtilde{q}_{\rho}\wtilde{q}_{\sigma} \left( -1 \right)^2 \gamma^{\mu} \wtilde{\gamma}^{\beta} \wtilde{\gamma}^{\rho} \wtilde{\gamma}^{\sigma} \wtilde{\gamma}_{\beta} = \wtilde{q}_{\rho}\wtilde{q}_{\sigma} \gamma^{\mu} \left( \epsilon \wtilde{\gamma}^{\rho}\wtilde{\gamma}^{\sigma} + 2 \left[ \wtilde{\gamma}^{\rho}, \wtilde{\gamma}^{\sigma} \right] \right) = \epsilon \wtilde{q}^2 \gamma^{\mu}
\end{equation*}
where we have used that the 4- and $\epsilon$-dimensional gamma matrices anticommute and equation \ref{eqn:4g} for $d=\epsilon$. And in the last step we used equation \ref{eqn:fsldot} which also implies $\left[ \wtilde{\fsl{q}}, \wtilde{\fsl{q}} \right] = 0$.\\
Hence, after summing all of the terms we have
\begin{equation*}
\wtilde{N} = - e^3 \epsilon \left( \fsl{p}_1 + \fsl{q} - m \right) \gamma^{\mu} \left( \fsl{p}_2 + \fsl{q} - m \right) + \left( 2 + \epsilon \right) \wtilde{q}^2 \gamma^{\mu}
\end{equation*}
We can again plug this in the definition of $R_2$ and get
\begin{align*}
R_2^{\gamma\mathrm{ee}} = & \frac{1}{\left( 2\pi \right) ^4} \int d^d\bar{q} \frac{\wtilde{N}}{\bar{D}_0 \bar{D}_1 \bar{D}_2} = \frac{1}{\left( 2\pi \right) ^4} \int d^d\bar{q} \frac{e^3}{\bar{D}_0 \bar{D}_1 \bar{D}_2} \lbr -\epsilon \left( \fsl{p}_1 + \fsl{q} - m \right) \gamma^{\mu} \left( \fsl{p}_2 + \fsl{q} - m \right) + \left( 2 + \epsilon \right) \wtilde{q}^2 \gamma^{\mu} \rbr = & \\
& = \frac{e^3}{\left( 2\pi \right) ^4} \int d^d\bar{q} \frac{1}{\bar{D}_0 \bar{D}_1 \bar{D}_2} \lbr -\epsilon \fsl{q}\gamma^{\mu}\fsl{q} + \left( 2 + \epsilon \right) \wtilde{q}^2 \gamma^{\mu} \rbr = \frac{e^3}{\left( 2\pi \right) ^4} \lbr -\epsilon \gamma^{\alpha}\gamma^{\mu}\gamma^{\beta} \left( \frac{-i\pi^2}{2\epsilon} g_{\alpha\beta} \right) + \frac{-i\pi^2}{2} \left(2 + \epsilon \right) \gamma^{\mu} \rbr = & \\
& = \frac{e^3}{\left( 2\pi \right) ^4} \frac{i \pi^2}{2} \lbr \gamma^{\alpha}\gamma^{\mu}\gamma_{\alpha} - 2\gamma^{\mu} + O(\epsilon) \rbr = \frac{-ie^3}{8\pi^2} \gamma^{\mu} & \numberthis \label{eqn:R2photonee}
\end{align*}
Where we used the 3-point integrals \ref{eqn:3ptintq2} and \ref{eqn:3ptintq2vec} and in the last step equation \ref{eqn:3g}\\
As already mentioned, there is one more 3-point function at the 1-loop level which is permitted by the Feynman rules: the 3-point function with only photons as external particles. But it does not contribute to $R_2$ which we will show now. Because of the symmetry of the 3-point function there are 2 contributing diagrams
\begin{align*}
\begin{gathered}
\feynmandiagram [small, horizontal=i1 to v] {
	i1 [particle=\alpha] -- [photon, momentum=\(-p_1-p_2\)] v [crossed dot],
	i2 [particle=\beta] -- [photon, momentum=\(p_1\)] v,
	i3 [particle=\gamma] -- [photon, momentum=\(p_2\)] v,
};
\end{gathered}
\quad = \quad  
\begin{gathered}
\feynmandiagram [horizontal=i1 to v1] {
	i1 [particle=\alpha] -- [photon, momentum=\(-p_1-p_2\)] v1,
	i3 [particle=\gamma] -- [photon, momentum=\(p_2\)] v3,
	i2 [particle=\beta] -- [photon, momentum=\(p_1\)] v2,
	v3 -- [fermion, momentum'=\(q\)] v2
	   -- [fermion, momentum'=\(p_1+q\)] v1
	   -- [fermion, momentum'=\(q-p_2\)] v3,
};
\end{gathered}
\quad + \quad
\begin{gathered}
\feynmandiagram [horizontal=i1 to v1] {
	i1 [particle=\alpha] -- [photon, momentum=\(-p_1-p_2\)] v1,
	i3 [particle=\gamma] -- [photon, momentum=\(p_2\)] v3,
	i2 [particle=\beta] -- [photon, momentum=\(p_1\)] v2,
	v2 -- [fermion, momentum=\(q\)] v3
	   -- [fermion, momentum=\(p_2+q\)] v1
	   -- [fermion, momentum=\(q-p_1\)] v2,
};
\end{gathered}
\end{align*}

We only calculate the first diagram and then symmetrize the result with $p_1 \leftrightarrow p_2$, $\beta \leftrightarrow \gamma$. Evaluating the first diagram gives
\begin{align*}
\begin{gathered}
\feynmandiagram [horizontal=i1 to v1] {
	i1 [particle=\alpha] -- [photon, momentum=\(-p_1-p_2\)] v1,
	i3 [particle=\gamma] -- [photon, momentum=\(p_2\)] v3,
	i2 [particle=\beta] -- [photon, momentum=\(p_1\)] v2,
	v3 -- [fermion, momentum'=\(q\)] v2
	   -- [fermion, momentum'=\(p_1+q\)] v1
	   -- [fermion, momentum'=\(q-p_2\)] v3,
};
\end{gathered}
\qquad
& =\int\frac{d^dq}{\left( 2\pi \right)^d} \Tr \lbr ie\gamma^{\beta} \frac{i \left( \fsl{q} + m \right)}{q^2 - m^2} ie \gamma^{\gamma} \frac{i \left( \fsl{q} - \fsl{p}_2 + m \right)}{\left( q- p_2 \right) -m^2} ie \gamma^{\alpha} \frac{i \left( \fsl{q} + \fsl{p}_1 + m \right)}{\left( q + p_1 \right) -m^2} \rbr =  & \\
& = \int\frac{d^dq}{\left( 2\pi \right)^d} e^3 \Tr \lbr \gamma^{\beta} \frac{\left( \fsl{q} + m \right)}{q^2 - m^2} \gamma^{\gamma} \frac{\left( \fsl{q} - \fsl{p}_2 + m \right)}{\left( q- p_2 \right) -m^2} \gamma^{\alpha} \frac{\left( \fsl{q} + \fsl{p}_1 + m \right)}{\left( q + p_1 \right) -m^2} \rbr = & \\
& \equiv \int\frac{d^dq}{\left( 2\pi \right)^d} \frac{\bar{N}}{\bar{D}_1\bar{D}_{-2}\bar{D}_0} &
\end{align*}
From here we can again extract the $d$-dimensional numerator
\begin{align*}
\bar{N} (\bar{q}) & = e^3 \Tr \lbr \bar{\gamma}^{\beta} \left( \bar{\fsl{q}} + m \right) \bar{\gamma}^{\gamma} \left( \bar{\fsl{q}} - \bar{\fsl{p}}_2 + m \right) \bar{\gamma}^{\alpha} \left( \bar{\fsl{q}} + \bar{\fsl{p}}_1 + m \right) \rbr = & \\
& = e^3 \Tr \lbr \gamma^{\beta} \left( \fsl{q} + m \right) \gamma^{\gamma} \left( \fsl{q} - \fsl{p}_2 + m \right) \gamma^{\alpha} \left( \fsl{q} + \fsl{p}_1 + m \right) +  \gamma^{\beta} \left( \fsl{q} + \wtilde{\fsl{q}} + m \right) \gamma^{\gamma} \left( \fsl{q} + \wtilde{\fsl{q}} - \fsl{p}_2 + m \right) \gamma^{\alpha} \left( \fsl{q} + \wtilde{\fsl{q}} + \fsl{p}_1 + m \right) \rbr = & \\
& \equiv N + \wtilde{N} &
\end{align*}
The last term is $\wtilde{N}$ which can be further simplified as follows
\begin{align*}
\wtilde{N} & = e^3 \Tr \lbr \gamma^{\beta} \left( \fsl{q} + \wtilde{\fsl{q}} + m \right) \gamma^{\gamma} \left( \fsl{q} + \wtilde{\fsl{q}} - \fsl{p}_2 + m \right) \gamma^{\alpha} \left( \fsl{q} + \wtilde{\fsl{q}} + \fsl{p}_1 + m \right) \rbr = & \\
& = e^3 \Tr \lbr \gamma^{\beta} \fsl{q} \gamma^{\gamma} \wtilde{\fsl{q}} \gamma^{\alpha} \wtilde{\fsl{q}} + \gamma^{\beta} \wtilde{\fsl{q}} \gamma^{\gamma} \fsl{q} \gamma^{\alpha} \wtilde{\fsl{q}} + \gamma^{\beta} \wtilde{\fsl{q}} \gamma^{\gamma} \wtilde{\fsl{q}} \gamma^{\alpha} \fsl{q} + \gamma^{\beta} \wtilde{\fsl{q}} \gamma^{\gamma} \left( - \fsl{p}_2 \right) \gamma^{\alpha} \wtilde{\fsl{q}} + \gamma^{\beta} \wtilde{\fsl{q}} \gamma^{\gamma} \wtilde{\fsl{q}} \gamma^{\alpha} \fsl{p}_1 \rbr = & \\
& = - 4e^3 \wtilde{q}^2 \lbr q_{\mu} \left[ \left( g^{\beta\mu}g^{\gamma\alpha} - g^{\beta\gamma}g^{\mu\alpha} + g^{\beta\alpha}g^{\mu\gamma} \right) + \left( g^{\beta\gamma}g^{\alpha\mu} - g^{\beta\mu}g^{\gamma\alpha} + g^{\beta\alpha}g^{\mu\gamma} \right) + \left( g^{\beta\gamma}g^{\mu\alpha} - g^{\beta\alpha}g^{\mu\gamma} + g^{\beta\mu}g^{\alpha\gamma} \right) \right] + \right. & \\ 
& \left. + p_{1\mu} \left( g^{\beta\gamma}g^{\alpha\mu} - g^{\beta\alpha}g^{\gamma\mu} + g^{\beta\mu}g^{\alpha\gamma} \right) - p_{2\mu} \left( g^{\beta\gamma}g^{\mu\alpha} - g^{\beta\mu}g^{\alpha\gamma} + g^{\alpha\beta}g^{\gamma\mu} \right) \rbr = & \\
& = -4e^3 \wtilde{q}^2 \lbr q^{\beta}g^{\alpha\gamma} + q^{\gamma}g^{\alpha\beta} + q^{\alpha}g^{\beta\gamma} + p_1^{\alpha}g^{\beta\gamma} - p_1^{\gamma}g^{\alpha\beta} + p_1^{\beta}g^{\alpha\gamma} - p_2^{\alpha}g^{\beta\gamma} + p_2^{\beta}g^{\alpha\gamma} - p_2^{\gamma}g^{\alpha\beta} \rbr &
\end{align*}
From the second to the third line we have used that 4- and $\epsilon$-dimensional gamma matrices commute as well as equations \ref{eqn:Tr4g} and \ref{eqn:fsldot}. \\
This gives for the $R_2$ contribution of the first diagram
\begin{align*}
R_2^1 & = \frac{1}{\left( 2\pi \right)^4} \int d^d\bar{q} \frac{\wtilde{N}}{\bar{D}_1\bar{D}_{-2}\bar{D}_0} = & \\
& = \frac{-4e^3}{\left( 2\pi \right)^4} \int d^d \bar{q} \frac{1}{\bar{D}_1\bar{D}_{-2}\bar{D}_0} \lbr \wtilde{q}^2q^{\beta}g^{\alpha\gamma} + \wtilde{q}^2q^{\gamma}g^{\alpha\beta} + \wtilde{q}^2q^{\alpha}g^{\beta\gamma} + \wtilde{q}^2 \left[ \left( p_1 - p_2 \right)^{\alpha} g^{\beta\gamma} + \left( p_1 + p_2 \right)^{\beta} g^{\alpha\gamma} - \left( p_1 + p_2 \right)^{\gamma} g^{\alpha\beta} \right] \rbr = & \\
& = \frac{-4e^3}{\left( 2\pi \right)^4} \lbr \frac{i\pi^2}{6} \left[ \left( p_1-p_2 \right)^{\beta}g^{\alpha\gamma} + \left( p_1-p_2 \right)^{\gamma}g^{\alpha\beta} + \left( p_1-p_2 \right)^{\alpha}g^{\beta\gamma} \right] - \frac{i\pi^2}{2} \left[ \left( p_1-p_2 \right)^{\alpha} g^{\beta\gamma} + \left( p_1+p_2 \right)^{\beta} g^{\alpha\gamma} + \right. \right. & \\
& \left. \left. - \left( p_1+p_2 \right)^{\gamma} g^{\alpha\beta} \right] \rbr = & \\
& = \frac{-4e^3}{\left( 2\pi \right)^4} \lbr g^{\alpha\beta} \left[ \frac{i\pi^2}{6} \left( p_1-p_2 \right)^{\gamma} + \frac{i\pi^2}{2} \left( p_1+p_2 \right)^{\gamma} \right] + g^{\beta\gamma} \left[ \frac{i\pi^2}{6} \left( p_1-p_2 \right)^{\alpha} - \frac{i\pi^2}{2} \left( p_1-p_2 \right)^{\alpha} \right] + \right. & \\
& \left. + g^{\alpha\gamma} \left[ \frac{i\pi^2}{6} \left( p_1-p_2 \right)^{\beta} - \frac{i\pi^2}{2} \left( p_1+p_2 \right)^{\beta} \right] \rbr &
\end{align*}
Here we have used the 3-point integrals \ref{eqn:3ptintq2} and \ref{eqn:3ptintq3}.\\
To obtain the contribution of the second diagram we can simply exchange $p_1 \leftrightarrow p_2$, $\beta \leftrightarrow \gamma$ in the result of the first diagram. This yields
\begin{align*}
R_2^2 & = R_2^1(p_1 \leftrightarrow p_2, \ \beta \leftrightarrow \gamma) = & \\
& = \frac{-4e^3}{\left( 2\pi \right)^4} \lbr g^{\alpha\gamma} \left[ \frac{i\pi^2}{6} \left( p_2-p_1 \right)^{\beta} + \frac{i\pi^2}{2} \left( p_2+p_1 \right)^{\beta} \right] + g^{\beta\gamma} \left[ \frac{i\pi^2}{6} \left( p_2-p_1 \right)^{\alpha} - \frac{i\pi^2}{2} \left( p_2-p_1 \right)^{\alpha} \right] + \right. & \\
& \left. + g^{\alpha\beta} \left[ \frac{i\pi^2}{6} \left( p_2-p_1 \right)^{\gamma} - \frac{i\pi^2}{2} \left( p_2+p_1 \right)^{\gamma} \right] \rbr = & \\
& = \frac{-4e^3}{\left( 2\pi \right)^4} \lbr -g^{\alpha\beta} \left[ \frac{i\pi^2}{6} \left( p_1-p_2 \right)^{\gamma} + \frac{i\pi^2}{2} \left( p_1+p_2 \right)^{\gamma} \right] - g^{\beta\gamma} \left[ \frac{i\pi^2}{6} \left( p_1-p_2 \right)^{\alpha} - \frac{i\pi^2}{2} \left( p_1-p_2 \right)^{\alpha} \right] + \right. & \\
& \left. - g^{\alpha\gamma} \left[ \frac{i\pi^2}{6} \left( p_1-p_2 \right)^{\beta} - \frac{i\pi^2}{2} \left( p_1+p_2 \right)^{\beta} \right] \rbr = -R_2^1 &
\end{align*}
Now we add up both diagrams to get the full contribution of the photon triangle diagram. We get
\begin{equation}
R_2^{3\gamma} = R_2^1 + R_2^2 = R_2^1 - R_2^1 = 0
\end{equation}

\subsection{4-point function}
For the 4-point function we have to be more careful. The 1PI contribution at the 1-loop level consists of several diagrams. They are obtained by symmetrizing the external momenta of the diagram as follows
\begin{align*}
\begin{gathered}
\feynmandiagram [small, horizontal=i1 to i2] {
	i4 [particle=\delta] -- [photon, momentum=\(p_2\)] v [crossed dot],
	i2 [particle=\beta] -- [photon, momentum=\(p_3\)] v,
	i1 [particle=\alpha] -- [photon, momentum=\(p_1\)] v,
	i3 [particle=\gamma] -- [photon, momentum=\(p_4\)] v,
};
\end{gathered}
= 2 \times \lbr
\begin{gathered}
\feynmandiagram [horizontal=i4 to i3] {
	i2 [particle=\beta] -- [photon, momentum=\(p_3\)] v2,
	i4 [particle=\delta] -- [photon, momentum=\(p_2\)] v4,
	i1 [particle=\alpha] -- [photon, momentum=\(p_1\)] v1,
	i3 [particle=\gamma] -- [photon, momentum=\(p_4\)] v3,
	v4 -- [fermion, momentum=\(q\)] v1
	   -- [fermion, momentum=\(p_1+q\)] v2
	   -- [fermion, momentum=\(q+p_1+p_3\)] v3
	   -- [fermion, momentum=\(q-p_2\)] v4,
};
\end{gathered}
\ \ + \ \ \left( \alpha \leftrightarrow \beta; \ p_1 \leftrightarrow p_3 \right) \ + \ \left( \alpha \leftrightarrow \delta; \ p_1 \leftrightarrow p_2 \right) \rbr 
\end{align*}

We only calculate one of the diagrams and do the symmetrizing with the result of our calculation, so we only have to evaluate one diagram. The first of the three diagrams gives
\begin{align*}
\begin{gathered}
\feynmandiagram [horizontal=i4 to i3] {
	i2 [particle=\beta] -- [photon, momentum=\(p_3\)] v2,
	i4 [particle=\delta] -- [photon, momentum=\(p_2\)] v4,
	i1 [particle=\alpha] -- [photon, momentum=\(p_1\)] v1,
	i3 [particle=\gamma] -- [photon, momentum=\(p_4\)] v3,
	v4 -- [fermion, momentum=\(q\)] v1
	   -- [fermion, momentum=\(p_1+q\)] v2
	   -- [fermion, momentum=\(q+p_1+p_3\)] v3
	   -- [fermion, momentum=\(q-p_2\)] v4,
};
\end{gathered}\qquad
& = \int\frac{d^dq}{\left( 2\pi \right)^d} \left( -1 \right) \mathrm{Tr} \lbr ie \gamma^{\alpha} \frac{i \left( \fsl{p}_1 + \fsl{q} + m \right)}{\left( p_1 +q \right) ^2 -m^2} ie \gamma^{\beta} \frac{i \left( \fsl{q} + \fsl{p}_3 + \fsl{p}_1 + m \right)}{\left( p_3 + p_1 +q \right) ^2 -m^2} \times \right. & \\
& \left. \times ie \gamma^{\gamma} \frac{i \left( \fsl{q} - \fsl{p}_2 + m \right)}{\left( q - p_2 \right) ^2 - m^2} ie \gamma^{\delta} \frac{i \left( \fsl{q} + m \right)}{q ^2 - m^2} \rbr \equiv \int\frac{d^dq}{\left( 2\pi \right)^d} \frac{\bar{N}}{\bar{D}_1\bar{D}_{13}\bar{D}_{-2}\bar{D}_0} &
\end{align*}
where $D_{13} = \left( p_3 + p_1 +q \right) ^2 -m^2$. From this we get for the $d$-dimensional numerator the following
\begin{align*}
\bar{N} (\bar{q}) & = -e^4 \mathrm{Tr} \lbr \bar{\gamma}^{\alpha} \left( \bar{\fsl{p}}_1 + \bar{\fsl{q}} + m \right) \bar{\gamma}^{\beta} \left( \bar{\fsl{q}} +  \bar{\fsl{p}}_1 + \bar{\fsl{p}}_3 + m \right) \bar{\gamma}^{\gamma} \left( \bar{\fsl{q}} - \bar{\fsl{p}}_2 + m \right) \bar{\gamma}^{\delta} \left( \bar{\fsl{q}} + m \right) \rbr = & \\
& = -e^4 \mathrm{Tr} \lbr \gamma^{\alpha} \left( \fsl{p}_1 + \fsl{q} + m \right) \gamma^{\beta} \left( \fsl{q} +  \fsl{p}_1 + \fsl{p}_3 + m \right) \gamma^{\gamma} \left( \fsl{q} - \fsl{p}_2 + m \right) \gamma^{\delta} \left( \fsl{q} + m \right) + \right. &  \\
&  \left. + \gamma^{\alpha} \wtilde{\fsl{q}} \gamma^{\beta} \wtilde{\fsl{q}} \gamma^{\gamma} \wtilde{\fsl{q}} \gamma^{\delta} \wtilde{\fsl{q}} + \gamma^{\alpha} \wtilde{\fsl{q}} \gamma^{\beta} \wtilde{\fsl{q}} \gamma^{\gamma} \fsl{q} \gamma^{\delta} \fsl{q}  + \gamma^{\alpha} \fsl{q} \gamma^{\beta} \wtilde{\fsl{q}} \gamma^{\gamma} \wtilde{\fsl{q}} \gamma^{\delta} \fsl{q} + \gamma^{\alpha} \fsl{q} \gamma^{\beta} \fsl{q} \gamma^{\gamma} \wtilde{\fsl{q}} \gamma^{\delta} \wtilde{\fsl{q}} + \gamma^{\alpha} \wtilde{\fsl{q}} \gamma^{\beta} \fsl{q} \gamma^{\gamma} \fsl{q} \gamma^{\delta} \wtilde{\fsl{q}} + \right. & \\
& \left. + \gamma^{\alpha} \wtilde{\fsl{q}} \gamma^{\beta} \fsl{q} \gamma^{\gamma} \wtilde{\fsl{q}} \gamma^{\delta} \fsl{q} + \gamma^{\alpha} \fsl{q} \gamma^{\beta} \wtilde{\fsl{q}} \gamma^{\gamma} \fsl{q} \gamma^{\delta} \wtilde{\fsl{q}} \rbr \equiv N + \wtilde{N} &
\end{align*}
Where all of the terms besides the first one define $\wtilde{N}$. Furthermore, we have used that the trace of an odd number of Dirac matrices is zero. Using the fact that 4- and $\epsilon$-dimensional gamma matrices commute as well as equation \ref{eqn:fsldot}, $\wtilde{N}$ can be further simplified to
\begin{align*}
\wtilde{N} = & -e^4 \mathrm{Tr} \lbr \left( -1 \right)^{10} \wtilde{q}^4 \gamma^{\alpha} \gamma^{\beta} \gamma^{\gamma} \gamma^{\delta} + \wtilde{q}^2 \left[ \left( -1 \right) ^3  \gamma^{\alpha} \gamma^{\beta} \gamma^{\gamma} \fsl{q} \gamma^{\delta} \fsl{q}  + \left( -1 \right)^7 \gamma^{\alpha} \fsl{q} \gamma^{\beta} \gamma^{\gamma} \gamma^{\delta} \fsl{q} + \left( -1 \right)^{11} \gamma^{\alpha} \fsl{q} \gamma^{\beta} \fsl{q} \gamma^{\gamma} \gamma^{\delta} + \right. \right. & \\
& \left. \left. + \left( -1 \right)^7 \gamma^{\alpha} \gamma^{\beta} \fsl{q} \gamma^{\gamma} \fsl{q} \gamma^{\delta} + \left( -1 \right)^5 \gamma^{\alpha} \gamma^{\beta} \fsl{q} \gamma^{\gamma} \gamma^{\delta} \fsl{q} + \left( -1 \right)^9 \gamma^{\alpha} \fsl{q} \gamma^{\beta} \gamma^{\gamma} \fsl{q} \gamma^{\delta} \right] \rbr = & \\
& = -e^4 \mathrm{Tr} \lbr \wtilde{q}^4 \gamma^{\alpha} \gamma^{\beta} \gamma^{\gamma} \gamma^{\delta} - \wtilde{q}^2 \left( \gamma^{\alpha} \gamma^{\beta} \gamma^{\gamma} \fsl{q} \gamma^{\delta} \fsl{q} + \gamma^{\alpha} \fsl{q} \gamma^{\beta} \gamma^{\gamma} \gamma^{\delta} \fsl{q} + \gamma^{\alpha} \fsl{q} \gamma^{\beta} \fsl{q} \gamma^{\gamma} \gamma^{\delta} + \gamma^{\alpha} \gamma^{\beta} \fsl{q} \gamma^{\gamma} \fsl{q} \gamma^{\delta} + \right. \right. & \\
& \left. \left. + \gamma^{\alpha} \gamma^{\beta} \fsl{q} \gamma^{\gamma} \gamma^{\delta} \fsl{q} + \gamma^{\alpha} \fsl{q} \gamma^{\beta} \gamma^{\gamma} \fsl{q} \gamma^{\delta} \right) \rbr &
\end{align*}
Since this expression involves the trace over up to 6 Dirac matrices, the calculation is very cumbersome. We can evaluate this expression with the help of the Mathematica package FeynCalc \cite{FeynCalc,FeynCalc2} 
\begin{figure}[h!]
  \begin{center}
    \includegraphics[width=1.05\textwidth]{Figures/Trace_4ptfct_Mathematica}
  \end{center}
  \setlength{\belowcaptionskip}{-20pt}
  \caption*{}
\end{figure} \\
As usual we plug this in the definition of $R_2$ and evaluate the integrals to get the expression of $R_2$ for the first of the contributing diagrams.
\begin{align*}
R_2 = & \frac{1}{\left( 2\pi \right) ^4} \int d^d\bar{q} \frac{\wtilde{N}}{\bar{D}_1 \bar{D}_{13} \bar{D}_{-2} \bar{D}_0} = \frac{1}{\left( 2\pi \right) ^4} \int d^d\bar{q} \frac{4e^4}{\bar{D}_1 \bar{D}_{13} \bar{D}_2 \bar{0}} \wtilde{q}^2 \lbr \left( 2q^2 + \wtilde{q}^2 \right) \left( g^{\alpha\delta}g^{\beta\gamma} - g^{\alpha\gamma}g^{\beta\delta} + g^{\alpha\beta}g^{\gamma\delta} \right) + \right. & \\
& \left. - 2 \left( g^{\alpha\beta}q^{\gamma}q^{\delta} + g^{\gamma\delta}q^{\alpha}q^{\beta} + g^{\alpha\delta}q^{\beta}q^{\gamma} + g^{\beta\gamma}q^{\alpha}q^{\delta} \right) \rbr = & \\
& = \frac{-4e^4}{\left( 2\pi \right) ^4} \lbr \left( 2 \left( \frac{-i\pi^2}{3} \right) + \left( \frac{-i\pi^2}{6} \right) \right) \left( g^{\alpha\delta}g^{\beta\gamma} - g^{\alpha\gamma}g^{\beta\delta} + g^{\alpha\beta}g^{\gamma\delta} \right) - 2 \left( \frac{-i\pi^2}{12} \right) \left( g^{\alpha\beta}g^{\gamma\delta} + g^{\gamma\delta}g^{\alpha\beta} + \right. \right. & \\
& \left. \left. + g^{\alpha\delta}g^{\beta\gamma} + g^{\beta\gamma}g^{\alpha\delta} \right) \rbr = \frac{ie^4}{4\pi^2} \lbr \frac{5}{6} \left( g^{\alpha\delta}g^{\beta\gamma} - g^{\alpha\gamma}g^{\beta\delta} + g^{\alpha\beta}g^{\gamma\delta} \right) - \frac{1}{6} \left( 2 g^{\alpha\beta}g^{\gamma\delta} + 2g^{\alpha\delta}g^{\beta\gamma} +  \right) \rbr = & \\
& = \frac{ie^4}{24 \pi^2} \left( 3 g^{\alpha\beta}g^{\gamma\delta} - 5 g^{\alpha\gamma}g^{\beta\delta} + 3 g^{\beta\gamma}g^{\alpha\delta} \right) &
\end{align*}
Where we have used the 4-point integrals \ref{eqn:4ptintq4}, \ref{eqn:4ptintvec} and \ref{eqn:4ptintq2q2}. This is independent of momenta, so we only have to symmetrize the indices to get the full 4-photon $R_2$. 
\begin{align*}
R_2^{4\gamma} = & 2 \left[ R_2 + R_2 \left( \alpha \leftrightarrow \delta \right) + R_2 \left( \alpha \leftrightarrow \beta \right) \right] = \frac{2ie^4}{24 \pi^2} \lbr \left( 3 g^{\alpha\beta}g^{\gamma\delta} - 5 g^{\alpha\gamma}g^{\beta\delta} + 3 g^{\beta\gamma}g^{\alpha\delta} \right) + \left( 3 g^{\beta\delta}g^{\alpha\gamma} - 5 g^{\gamma\delta}g^{\alpha\beta} + 3 g^{\beta\gamma}g^{\alpha\delta} \right) + \right. & \\
& \left. + \left( 3 g^{\alpha\beta}g^{\gamma\delta} - 5 g^{\beta\gamma}g^{\alpha\delta} + 3 g^{\alpha\gamma}g^{\beta\delta} \right) \rbr = \frac{ie^4}{12 \pi^2} \left( g^{\alpha\beta}g^{\gamma\delta} + g^{\alpha\gamma}g^{\beta\delta} + g^{\beta\gamma}g^{\alpha\delta} \right) & \numberthis \label{eqn:R24photon}
\end{align*}
Like for the 3-point functions all of the other 4-point functions which are permitted by the Feynman rules vanish. We will not show this here because the calculations for the 4-point functions are quite lengthy. \\
We have derived the complete set of $R_2$ in pure QED. Now we can go to the more complex Standard Model to see how QED contributes to the rational terms in the full Standard Model.
\section{QED Contribution to R$_2$ in the Standard Model}
\label{sec:QED2SM}
As already mentioned above, the Standard Model is a chiral theory which will make the calculations more complicated. What is also new in the Standard Model is that the uncharged (in the sense of the whole gauge group) gauge boson can now fluctuate to a different uncharged gauge boson which leads to a lot more diagrams for the 2-point functions. Let us start again with the simplest correlation functions, the 2-point functions.
\subsection{2-point functions}
{\bf Z-boson self-energy}\\
One of the 2-point functions in the Standard Model which gets a contribution from QED at 1-loop level is the Z-boson 2-point function. The fermionic loop correction is given by
\begin{align*}
\begin{gathered}
\feynmandiagram [layered layout, horizontal=b to c] {
	a [particle={\(Z,\alpha\)}] -- [photon, momentum=\(p_1\)] b
	  -- [fermion, half left, looseness=1.5, momentum=\(p_1+q\)] c
	  -- [fermion, half left, looseness=1.5, momentum=\(q\)] b,
	c -- [photon, momentum=\(p_1\)] d [particle={\(\beta,Z\)}] ,
};
\end{gathered} \qquad
& =\int\frac{d^dq}{\left( 2\pi \right)^d} \left( -1 \right) \Tr \lbr \frac{ig}{cos\theta_W} \gamma^{\alpha} \left( g_V - g_A \gamma_5 \right) \frac{i \left( \fsl{p}_1+\fsl{q}+m_f \right)}{\left( p_1+q \right)^2 - m_f^2} \frac{ig}{cos\theta_W} \gamma^{\beta} \times \right. & \\
& \times \left. \left( g_V - g_A \gamma_5 \right) \frac{i \left( \fsl{q}+m_f \right)}{q^2 - m_f^2} \rbr = & \\
& =\int\frac{d^dq}{\left( 2\pi \right)^d} \frac{-g^2}{\cos^2\theta_W} \Tr \lbr \gamma^{\alpha} \left( g_V - g_A \gamma_5 \right) \frac{ \left( \fsl{p}_1+\fsl{q}+m_f \right)}{\left( p_1+q \right)^2 - m_f^2} \gamma^{\beta} \left( g_V - g_A \gamma_5 \right) \frac{ \left( \fsl{q}+m_f \right)}{q^2 - m_f^2} \rbr & \\
& \equiv \int\frac{d^dq}{\left( 2\pi \right)^d} \frac{\bar{N}}{\bar{D}_1\bar{D}_0} &
\end{align*}
From the $d$-dimensional numerator function we can as usual extract the $\epsilon$-dimensional part we are interested in.
\begin{align*}
\bar{N} (\bar{q}) = & - \frac{g^2}{\cos^2\theta_W} \ \Tr  \lbr \bar{\gamma}^{\alpha} \left( g_V - g_A \gamma_5 \right) \left( \bar{\fsl{p}}_1+\bar{\fsl{q}}+m \right) \bar{\gamma}^{\beta} \left( g_V - g_A \gamma_5 \right) \left( \bar{\fsl{q}}+m \right) \rbr = & \\
&= \frac{-g^2}{\cos^2\theta_W} \Tr \lbr \gamma^{\alpha} \left( g_V - g_A \gamma_5 \right) \left( \fsl{p}_1 + \fsl{q} + m \right) \gamma^{\beta} \left( g_V - g_A \gamma_5 \right) \left( \fsl{q} + m \right) + \gamma^{\alpha} \left( g_V^2 + g_A^2 \right) \wtilde{\fsl{q}} \gamma^{\beta} \wtilde{\fsl{q}} \rbr \equiv N + \wtilde{N} &
\end{align*}
Where we used $ \left[ \gamma_5, \wtilde{\gamma}^{\mu} \right] = 0 $ and the fact that the gamma matrices will be contracted with external momenta. We can further simplify $\wtilde{N}$ by evaluating the trace over the gamma matrices
\begin{equation*}
\wtilde{N} = \frac{-g^2}{\cos^2\theta_W} \left( g_V^2 + g_A^2 \right) \left( - \wtilde{q}^2 \right) \Tr\left( \gamma^{\alpha}\gamma^{\beta} \right) = \frac{4g^2\wtilde{q}^2}{\cos^2\theta_W} \left( g_V^2 + g_A^2 \right) g^{\alpha\beta}
\end{equation*}
Plugging this into the definition of $R_2$ gives
\begin{align*}
R_2^{ZZ} & = \frac{1}{\left( 2\pi \right) ^4} \int d^d\bar{q} \frac{\wtilde{N}}{\bar{D}_1\bar{D}_0} = \frac{4g^2g^{\alpha\beta}}{\left( 2\pi \right)^4 \cos^2\theta_W} \left( g_V^2 + g_A^2 \right) \int d^d\bar{q} \frac{\wtilde{q}^2}{\bar{D}_1\bar{D}_0} = & \\
& = \frac{4g^2g^{\alpha\beta}}{\left( 2\pi \right)^4 \cos^2\theta_W} \left( g_V^2 + g_A^2 \right) \left( - \frac{i\pi^2}{2} \right) \left( 2m^2 - \frac{p_1^2}{3} \right) = \frac{-ig^2}{ 8 \pi^2 \cos^2\theta_W} \left( g_V^2 + g_A^2 \right)  \left( 2m^2 - \frac{p_1^2}{3} \right) g^{\alpha\beta} & \\ \numberthis \label{eqn:R2ZZ}
\end{align*}
\\
{\bf Photon/Z-boson mixed self-energy}\\
Since the Z-boson is a singlet under $SU(2)_L \times U(1)_Y$ it can also oscillate into a photon via a fermion loop. The amplitude of this process is given by
\begin{align*}
\begin{gathered}
\feynmandiagram [layered layout, horizontal=b to c] {
	a [particle={\(\gamma,\alpha\)}] -- [photon, momentum=\(p_1\)] b
	  -- [fermion, half left, looseness=1.5, momentum=\(p_1+q\)] c
	  -- [fermion, half left, looseness=1.5, momentum=\(q\)] b,
	c -- [photon, momentum=\(p_1\)] d [particle={\(\beta,Z\)}] ,
};
\end{gathered} \qquad
& =\int\frac{d^dq}{\left( 2\pi \right)^d} \left( -1 \right) \Tr \lbr \left( - i e Q_f \right) \gamma^{\alpha} \frac{i \left( \fsl{p}_1+\fsl{q}+m \right)}{\left( p_1+q \right)^2 - m^2} \frac{ig}{cos\theta_W} \gamma^{\beta} \times \right. & \\
& \times \left. \left( g_V - g_A \gamma_5 \right) \frac{i \left( \fsl{q}+m \right)}{q^2 - m^2} \rbr = & \\
& =\int\frac{d^dq}{\left( 2\pi \right)^d} \frac{eQ_fg}{\cos\theta_W} \Tr \lbr \gamma^{\alpha} \frac{ \left( \fsl{p}_1+\fsl{q}+m \right)}{\left( p_1+q \right)^2 - m^2} \gamma^{\beta} \left( g_V - g_A \gamma_5 \right) \frac{ \left( \fsl{q}+m \right)}{q^2 - m^2} \rbr & \\
& \equiv \int\frac{d^dq}{\left( 2\pi \right)^d} \frac{\bar{N}}{\bar{D}_1\bar{D}_0} &
\end{align*}
We simplify the $d$-dimensional denominator function
\begin{align*}
\bar{N} (\bar{q}) & = \frac{eQ_fg}{\cos\theta_W} \ \Tr \lbr \bar{\gamma}^{\alpha} \left( \bar{\fsl{p}}_1+\bar{\fsl{q}}+m \right) \bar{\gamma}^{\beta} \left( g_V - g_A \gamma_5 \right) \left( \bar{\fsl{q}}+m \right) \rbr = & \\
& = \frac{eQ_fg}{\cos\theta_W} \Tr \lbr \gamma^{\alpha} \left( \fsl{p}_1 + \fsl{q} + m \right) \gamma^{\beta} \left( g_V - g_A \gamma_5 \right) \left( \fsl{q} + m \right) + \gamma^{\alpha} \wtilde{\fsl{q}} \gamma^{\beta} g_V \wtilde{\fsl{q}} \rbr \equiv N + \wtilde{N} &
\end{align*}
Where we have used $\Tr\left( \gamma^{\alpha}\gamma^{\beta}\gamma_5 \right) = 0$ and the fact that the gamma matrices are contracted with external momenta. The last term defines $\wtilde{N}$ and can be further simplified by evaluating the trace
\begin{equation*}
\wtilde{N} = \frac{eQ_fg}{cos\theta_W} \Tr\lbr \gamma^{\alpha} \wtilde{\fsl{q}} \gamma^{\beta} g_V \wtilde{\fsl{q}} \rbr = \frac{-4eQ_fgg_V}{\cos\theta_W} \wtilde{q}^2 g^{\alpha\beta}
\end{equation*}
With this we can calculate the $R_2$ for the mixed photon/Z-boson self-energy.
\begin{align*}
R_2^{\gamma Z} & = \frac{1}{\left( 2\pi \right)^4} \int d^d \bar{q} \frac{\wtilde{N}}{\bar{D}_1\bar{D}_0} = \frac{-4eQ_fgg_V}{\left( 2\pi \right)^4\cos\theta_W} g^{\alpha\beta} \int d^d\bar{q} \frac{\wtilde{q}^2}{\bar{D}_1\bar{D}_0} = & \\
& = \frac{-4eQ_fgg_V}{\left( 2\pi \right)^4\cos\theta_W} \left( -\frac{i\pi^2}{2} \right) g^{\alpha\beta} \left( 2m^2 - \frac{p_1^2}{3} \right) = \frac{ieQ_fgg_V}{8\pi^2\cos\theta_W} g^{\alpha\beta} \left( 2m^2 - \frac{p_1^2}{3} \right) & \numberthis \label{eqn:R2photonZ}
\end{align*}

{\bf Gluon self-energy} \\
Because the gluon (just as the photon) couples to a pure vector current, the calculation for the gluon self-energy $R_2$ is the same as for the photon self-energy $R_2$ replacing the electric charge generator with the colour charge generator. So, from equation \ref{eqn:R2photon} with $e Q_f \rightarrow g_S T^a$ we get
\begin{equation}
\label{eqn:R2gluon}
R_2^{gg} = R_2^{\gamma\gamma} \left( e Q_f \rightarrow g_S T^a \right) = \frac{-ig_S^2}{8\pi^2} \Tr \left( T^a T^b \right) g^{\alpha\beta} \left( 2m^2 -\frac{p_1^2}{3} \right) 
\end{equation}

\subsection{3-point functions}
{\bf Gluon-quark vertex}
\begin{align*}
\begin{gathered}
\feynmandiagram [horizontal=i1 to v2] {
	i1 [particle=\(\mu\)] -- [gluon, momentum=\(p_2-p_1\)] v2,
	f1 -- [fermion, momentum=\(p_1\)] v1
	   -- [fermion, momentum=\(p_1+q\)] v2
	   -- [fermion, momentum=\(p_2+q\)] v3
	   -- [fermion, momentum=\(p_2\)] f3,
	v3 -- [boson, momentum=\(q\)] v1, 
};
\end{gathered} \qquad
& = \int\frac{d^dq}{\left( 2\pi \right)^d} \left( -ie Q_q \gamma^{\beta} \right) \frac{i \left( \fsl{p}_1+\fsl{q}+m \right)}{\left( p_1+q \right)^2 - m^2} \left( -ig_S \gamma^{\mu} T^a \right) \frac{i \left( \fsl{p}_2 + \fsl{q}+m \right)}{\left( p_2 + q \right)^2 - m^2} \left( -ie Q_q \gamma^{\alpha} \right) \frac{-ig_{\alpha\beta}}{q^2} = & \\
& = \int\frac{d^dq}{\left( 2\pi \right)^d} -e^2 Q_q^2 g_S \gamma^{\beta} \frac{ \left( \fsl{p}_1+\fsl{q}+m \right)}{\left( p_1+q \right)^2 - m^2} \gamma^{\mu} T^a \frac{ \left( \fsl{p}_2 + \fsl{q}+m \right) }{\left( p_2 + q \right)^2 - m^2} \gamma_{\beta} \frac{1}{q^2} = & \\
& \equiv \int\frac{d^dq}{\left( 2\pi \right)^d} \frac{\bar{N}}{\bar{D}_1\bar{D}_2\bar{D}_0} &
\end{align*}

\begin{align*}
\bar{N} (\bar{q}) & = -e^2 Q_q^2 g_S \lbr \bar{\gamma}^{\beta} \left( \bar{\fsl{p}}_1 + \bar{\fsl{q}} + m \right) \bar{\gamma}^{\mu} T^a \left( \bar{\fsl{p}}_2 + \bar{\fsl{q}} + m \right) \bar{\gamma}_{\beta} \rbr = -e^2 Q_q^2 g_S \lbr \gamma^{\beta} \left( \fsl{p}_1 + \fsl{q} + m \right) \gamma^{\mu} T^a \left( \fsl{p}_2 + \fsl{q} + m \right) \gamma_{\beta} + \right. & \\
& \left. + \gamma^{\beta} \wtilde{\fsl{q}} \gamma^{\mu} T^a \wtilde{\fsl{q}} \gamma_{\beta} + \wtilde{\gamma}^{\beta} \fsl{q} \gamma^{\mu} T^a \fsl{q} \wtilde{\gamma}_{\beta} \rbr \equiv N + \wtilde{N} &
\end{align*}

\begin{equation*}
\wtilde{N} = -e^2 Q_q^2 g_S \lbr -\wtilde{q}^2 \underbrace{\gamma^{\beta}\gamma^{\mu}\gamma_{\beta}}_{-2\gamma^{\mu}} T^a - \epsilon q_{\alpha}q_{\beta} \gamma^{\alpha}\gamma^{\mu}\gamma^{\beta}  T^a \rbr = -e^2 Q_q^2 g_S \lbr 2\wtilde{q}^2 \gamma^{\mu} T^a - \epsilon q_{\alpha}q_{\beta} \gamma^{\alpha}\gamma^{\mu}\gamma^{\beta} T^a \rbr
\end{equation*}

\begin{align*}
R_2^{gqq} = & \frac{1}{\left( 2\pi \right)^4} \int d^d \bar{q} \frac{\wtilde{N}}{\bar{D}_1\bar{D}_2\bar{D}_0} = \frac{-e^2Q_q^2g_S}{\left( 2\pi \right)^4} \int d^d \bar{q} \frac{1}{\bar{D}_1\bar{D}_2\bar{D}_0} \lbr 2\wtilde{q}^2 \gamma^{\mu} T^a - \epsilon q_{\alpha}q_{\beta} \gamma^{\alpha}\gamma^{\mu}\gamma^{\beta} T^a \rbr = & \\
& = \frac{-e^2Q_q^2g_S}{\left( 2\pi \right)^4} \lbr 2 \left( \frac{-i\pi^2}{2} \right) \gamma^{\mu} T^a - \epsilon \left( \frac{-i\pi^2}{2\epsilon} \right) \underbrace{g_{\alpha\beta} \gamma^{\alpha}\gamma^{\mu}\gamma^{\beta}}_{-2\gamma^{\mu}} T^a \rbr = \frac{-e^2 Q_q^2 g_S}{16 \pi^4} \left( \frac{-i\pi^2}{2} \right) \lbr 2\gamma^{\mu} T^a + 2\gamma^{\mu} T^a \rbr = & \\
& = \frac{ie^2 Q_q^2 g_S}{8 \pi^2} \gamma^{\mu} T^a & \numberthis \label{eqn:R2gluonqq}
\end{align*}


{\bf Z-fermion vertex}
\begin{align*}
\begin{gathered}
\feynmandiagram [horizontal=i1 to v2] {
	i1 [particle={\(Z,\mu\)}] -- [photon, momentum=\(p_2-p_1\)] v2,
	f1 -- [fermion, momentum=\(p_1\)] v1
	   -- [fermion, momentum=\(p_1+q\)] v2
	   -- [fermion, momentum=\(p_2+q\)] v3
	   -- [fermion, momentum=\(p_2\)] f3,
	v3 -- [boson, momentum=\(q\)] v1, 
};
\end{gathered} \qquad
& = \int\frac{d^dq}{\left( 2\pi \right)^d} \left( -ie Q_f \gamma^{\beta} \right) \frac{i \left( \fsl{p}_1+\fsl{q}+m \right)}{\left( p_1+q \right)^2 - m^2} \frac{ig}{\cos\theta_W} \gamma^{\mu} \left( g_V - g_A \gamma_5 \right) \frac{i \left( \fsl{p}_2 + \fsl{q}+m \right)}{\left( p_2 + q \right)^2 - m^2} \times & \\
& \times \left( -ie Q_f \gamma^{\alpha} \right) \frac{-ig_{\alpha\beta}}{q^2} = & \\
& = \int\frac{d^dq}{\left( 2\pi \right)^d} \frac{e^2 Q_f^2g}{\cos\theta_W} \gamma^{\beta} \frac{ \left( \fsl{p}_1+\fsl{q}+m \right)}{\left( p_1+q \right)^2 - m^2} \gamma^{\mu} \left( g_V - g_A \gamma_5 \right) \frac{ \left( \fsl{p}_2 + \fsl{q}+m \right) }{\left( p_2 + q \right)^2 - m^2} \gamma_{\beta} \frac{1}{q^2} = & \\
& \equiv \int\frac{d^dq}{\left( 2\pi \right)^d} \frac{\bar{N}}{\bar{D}_1\bar{D}_2\bar{D}_0} &
\end{align*}

\begin{align*}
\bar{N} (\bar{q}) & = \frac{e^2Q_f^2g}{\cos\theta_W} \lbr \bar{\gamma}^{\beta} \left( \bar{\fsl{p}}_1 + \bar{\fsl{q}} + m \right) \bar{\gamma}^{\mu} \left( g_V - g_A \gamma_5 \right) \left( \bar{\fsl{p}}_2 + \bar{\fsl{q}} + m \right) \bar{\gamma}_{\beta} \rbr = \frac{e^2Q_f^2g}{\cos\theta_W} \lbr \gamma^{\beta} \left( \fsl{p}_1 + \fsl{q} + m \right) \gamma^{\mu} \left( g_V - g_A \gamma_5 \right) \times \right. & \\
& \left. \times \left( \fsl{p}_2 + \fsl{q} + m \right) \gamma_{\beta} + \wtilde{\gamma}^{\beta} \left( \fsl{p}_1 + \fsl{q} + m \right) \gamma^{\mu} \left( g_V - g_A \gamma_5 \right) \left( \fsl{p}_2 + \fsl{q} + m \right) \wtilde{\gamma}_{\beta} + \left( \gamma^{\beta} + \wtilde{\gamma}^{\beta} \right) \wtilde{\fsl{q}} \gamma^{\mu} \left( g_V - g_A \gamma_5 \right) \wtilde{\fsl{q}} \left( \gamma_{\beta} + \wtilde{\gamma}_{\beta} \right) \rbr = & \\
& \equiv N + \wtilde{N} &
\end{align*}

\begin{align*}
\wtilde{N} & = \frac{e^2Q_f^2g}{\cos\theta_W} \lbr \left( \fsl{p}_1 + \fsl{q} - m \right) \wtilde{\gamma}^{\beta} \gamma^{\mu} \wtilde{\gamma}_{\beta} \left( g_V - g_A \gamma_5 \right) \left( \fsl{p}_2 + \fsl{q} - m \right) + \gamma^{\beta} \wtilde{\fsl{q}} \gamma^{\mu} \left( g_V - g_A \gamma_5 \right) \wtilde{\fsl{q}} \gamma_{\beta} + \wtilde{\gamma}^{\beta} \wtilde{\fsl{q}} \gamma^{\mu} \left( g_V - g_A \gamma_5 \right) \wtilde{\fsl{q}} \wtilde{\gamma}_{\beta}   \rbr = & \\
& = \frac{e^2Q_f^2g}{\cos\theta_W} \lbr -\epsilon \left( \fsl{p}_1 + \fsl{q} - m \right) \gamma^{\mu} \left( g_V - g_A \gamma_5 \right) \left( \fsl{p}_2 + \fsl{q} - m \right) - \wtilde{q}^2 \gamma^{\beta}\gamma^{\mu}\gamma_{\beta} \left( g_V + g_A \gamma_5 \right) - \wtilde{q}^2 \left( -\epsilon \gamma^{\mu} \right) \left( g_V - g_A \gamma_5 \right) \rbr = & \\
& = \frac{e^2Q_f^2g}{\cos\theta_W} \lbr -\epsilon \left( \fsl{p}_1 + \fsl{q} - m \right) \gamma^{\mu} \left( g_V - g_A \gamma_5 \right) \left( \fsl{p}_2 + \fsl{q} - m \right) + \wtilde{q}^2 \left( 2 \gamma^{\mu} \left( g_V + g_A \gamma_5 \right) + \epsilon \gamma^{\mu} \left( g_V - g_A \gamma_5 \right) \right) \rbr = &
\end{align*}

\begin{align*}
R_2^{Zff} & = \frac{1}{\left( 2\pi \right)^4} \int d^d \bar{q} \frac{\wtilde{N}}{\bar{D}_1\bar{D}_2\bar{D}_0} = \frac{e^2Q_f^2g}{\left( 2\pi \right)^4 \cos\theta_W} \int d^d \bar{q} \frac{1}{\bar{D}_1\bar{D}_2\bar{D}_0} \lbr -\epsilon \fsl{q} \gamma^{\mu} \left( g_V - g_A \gamma_5 \right) \fsl{q} + \wtilde{q}^2 \left( 2 \gamma^{\mu} \left( g_V + g_A \gamma_5 \right) + \right. \right. & \\
& \left. \left. +  \epsilon \gamma^{\mu} \left( g_V - g_A \gamma_5 \right) \right) \rbr = \frac{e^2Q_f^2g}{\cos\theta_W} \lbr -\epsilon \left( -\frac{i\pi^2}{2\epsilon} \right) g_{\alpha\beta} \gamma^{\alpha}\gamma^{\mu}\gamma^{\beta} \left( g_V + g_A \gamma_5 \right) + 2 \left( - \frac{i\pi^2}{2} \right) \gamma^{\mu}  \left( g_V + g_A \gamma_5 \right) \rbr = & \\
& = \frac{e^2Q_f^2g}{\cos\theta_W} \left( - \frac{i\pi^2}{2} \right) \gamma^{\mu} \lbr 2 \left( g_V + g_A \gamma_5 \right) + 2 \left( g_V + g_A \gamma_5 \right) \rbr = \frac{-ie^2Q_f^2g}{8\pi^2\cos\theta_W} \gamma^{\mu} \left( g_V + g_A \gamma_5 \right)& \numberthis \label{eqn:R2Zff}
\end{align*}
where we used that scalar 3-point integrals do not contribute to $R_2$. The last term in the integral is of order $\epsilon$ so it will not contribute in the limit $\epsilon \rightarrow 0$

{\bf Higgs-fermion Yukawa vertex}
\begin{align*}
\begin{gathered}
\feynmandiagram [horizontal=i1 to v2] {
	i1 [particle=\(H\)] -- [scalar, momentum=\(p_2-p_1\)] v2,
	f1 -- [fermion, momentum=\(p_1\)] v1
	   -- [fermion, momentum=\(p_1+q\)] v2
	   -- [fermion, momentum=\(p_2+q\)] v3
	   -- [fermion, momentum=\(p_2\)] f3,
	v3 -- [boson, momentum=\(q\)] v1, 
};
\end{gathered} \qquad
& = \int\frac{d^dq}{\left( 2\pi \right)^d} \left( -ie Q_f \gamma^{\beta} \right) \frac{i \left( \fsl{p}_1+\fsl{q}+m \right)}{\left( p_1+q \right)^2 - m^2} \left( -\frac{ig}{2}\frac{m}{m_W} \right) \frac{i \left( \fsl{p}_2 + \fsl{q}+m \right)}{\left( p_2 + q \right)^2 - m^2} \left( -ie Q_f \gamma^{\alpha} \right) \frac{-ig_{\alpha\beta}}{q^2} = & \\
& = \int\frac{d^dq}{\left( 2\pi \right)^d} \frac{-e^2Q_f^2gm}{2m_W} \gamma^{\beta} \frac{ \left( \fsl{p}_1+\fsl{q}+m \right)}{\left( p_1+q \right)^2 - m^2} \gamma^{\mu} \frac{ \left( \fsl{p}_2 + \fsl{q}+m \right) }{\left( p_2 + q \right)^2 - m^2} \gamma_{\beta} \frac{1}{q^2} = & \\
& \equiv \int\frac{d^dq}{\left( 2\pi \right)^d} \frac{\bar{N}}{\bar{D}_1\bar{D}_2\bar{D}_0} &
\end{align*}

\begin{align*}
\bar{N} (\bar{q}) & = \frac{e^2Q_f^2gm}{2m_W} \bar{\gamma}^{\beta} \left( \bar{\fsl{p}}_1 + \bar{\fsl{q}} + m \right) \left( \bar{\fsl{p}}_2 + \bar{\fsl{q}} + m \right) \bar{\gamma}_{\beta} = & \\
& = \frac{e^2Q_f^2gm}{2m_W} \lbr \gamma^{\beta} \left( \fsl{p}_1 + \fsl{q} + m \right) \left( \fsl{p}_2 + \fsl{q} + m \right) \gamma_{\beta} + \wtilde{\gamma}^{\beta} \left( \fsl{p}_1 + \fsl{q} + m \right) \left( \fsl{p}_2 + \fsl{q} + m \right) \wtilde{\gamma}_{\beta} + \gamma^{\beta} \wtilde{\fsl{q}} \wtilde{\fsl{q}} \gamma_{\beta} \rbr \equiv N + \wtilde{N} &
\end{align*}

\begin{equation*}
\wtilde{N} = -\frac{e^2Q_f^2gm}{2m_W} \lbr \wtilde{\gamma}^{\beta} \left( \fsl{p}_1 + \fsl{q} + m \right) \left( \fsl{p}_2 + \fsl{q} + m \right) \wtilde{\gamma}_{\beta} + \gamma^{\beta} \wtilde{\fsl{q}} \wtilde{\fsl{q}} \gamma_{\beta} \rbr = -\frac{e^2Q_f^2gm}{2m_W} \lbr \wtilde{\gamma}^{\beta}\wtilde{\gamma}_{\beta} \fsl{q}\fsl{q} + \wtilde{\fsl{q}} \wtilde{\fsl{q}} \gamma^{\beta}\gamma_{\beta} \rbr 
\end{equation*}

\begin{align*}
R_2 & = \frac{1}{\left( 2\pi \right)^4} \int d^d \bar{q} \frac{\wtilde{N}}{\bar{D}_1\bar{D}_0\bar{D}_2} = \frac{1}{\left( 2\pi \right)^4} \int d^d \bar{q} \frac{1}{\bar{D}_1\bar{D}_0\bar{D}_2} \left(-\frac{e^2Q_f^2gm}{2m_W} \right) \lbr \wtilde{\gamma}^{\beta}\wtilde{\gamma}_{\beta} \fsl{q}\fsl{q} + \wtilde{\fsl{q}} \wtilde{\fsl{q}} \gamma^{\beta}\gamma_{\beta} \rbr  = & \\
& -\frac{e^2Q_f^2gm}{2m_W} \lbr \epsilon \left( -\frac{i\pi^2}{2\epsilon} \right) \gamma^{\alpha} \gamma^{\beta} g_{\alpha\beta} \fsl{q}\fsl{q} + \wtilde{\fsl{q}} \wtilde{\fsl{q}} \gamma^{\beta}\gamma_{\beta} + \left( - \frac{i\pi^2}{2} \right) 4 \rbr = \frac{-e^2Q_f^2gm}{\left( 2\pi \right)^4 2m_W} \left( - \frac{i\pi^2}{2} \right) 8 = \frac{ie^2Q_f^2gm}{8\pi^2m_W} &
\end{align*}
\section{Perturbative Renormalization in Terms of Scalar Integrals}
We start from the QED Lagrangian 
\begin{equation}
\mathcal{L} = - \frac{1}{4} F_{\mu\nu}^0 F^{\mu\nu}_0 + \bar{\psi}_0 \left( i \fsl{\partial} -m_0 \right) \psi_0 - e_0 \bar{\psi}_0 \fsl{A}_0 \psi_0
\end{equation}
where $F^{\mu\nu}_0 = \partial^{\mu}A^{\nu}_0 - \partial^{\nu}A^{\mu}_0$. Now, we reinterpret the fields and parameters in the Lagrangian as "bare" fields and parameters which are given by the actual "renormalized" quantities times a renormalization constant
\begin{align*}
& \psi_0 = \sqrt{Z_2} \psi & \\
& A^{\mu}_0 = \sqrt{Z_3} A^{\mu} & \\
& m_0 = Z_m m & \\ 
& e_0 = Z_e e \mu^{-\frac{\epsilon}{2}} & \numberthis \label{eqn:RenormCon}
\end{align*}
The renormalization constants $Z_i$ absorb the divergences which appear in loop calculations. We can split them as $Z_i = 1 + \delta_i$ to extract the renormalized Lagrangian which is divergence free and the so called counter-term Lagrangian which absorbs the divergences
\begin{align*}
\mathcal{L} & = - \frac{1}{4} Z_3 F_{\mu\nu} F^{\mu\nu} + i Z_2 \bar{\psi} \fsl{\partial} \psi - Z_m Z_2 m \bar{\psi} \psi - e Z_1 \bar{\psi} \fsl{A} \psi = & \\
& = - \frac{1}{4} F_{\mu\nu} F^{\mu\nu} + \bar{\psi} \left( i \fsl{\partial} - m \right) \psi - e \bar{\psi} \fsl{A} \psi - \frac{1}{4} \delta_3 F_{\mu\nu} F^{\mu\nu} + i \delta_2 \bar{\psi} \fsl{\partial} \psi - \left( \delta_m + \delta_2 \right) m \bar{\psi} \psi - e \delta_1 \bar{\psi} \fsl{A} \psi \equiv \mathcal{L}_{ren} + \mathcal{L}_{ct} & \numberthis
\end{align*}
where $Z_1 = Z_e Z_2 \sqrt{Z_3} \mu^{-\frac{\epsilon}{2}}$. \\
The counter term Lagrangians gives the following new Feynman rules
\begin{align*}
\begin{gathered}
\feynmandiagram [layered layout, horizontal=a to c] {
	a [particle=\(\alpha\)] -- [photon, momentum=\(p\)] b [crossed dot] -- [photon] c [particle=\(\beta\)],
};
\end{gathered}
& = i \left( p^{\alpha}p^{\beta} - g^{\alpha\beta}p^2 \right) \delta_3 & \\
\begin{gathered}
\feynmandiagram [layered layout, horizontal=a to c] {
	a -- [fermion, momentum=\(p\)] b [crossed dot] -- [fermion] c,
};
\end{gathered}
& = i \left( \fsl{p} \delta_2 - \left( \delta_m + \delta_2 \right) m \right) & \\
\begin{gathered}
\feynmandiagram [layered layout, horizontal=a to b] {
	a[particle=\(\alpha\)] -- [photon] b [crossed dot] -- [fermion] c, b -- [fermion] d,
};
\end{gathered}
& = -ie \gamma^{\mu} \delta_1 & \numberthis
\end{align*}
We can use these new Feynman rules to calculate the $Z_i$ in order to be able to make predictions with perturbative calculations. These renormalization conditions can be obtained by calculating the dressed propagators and requiring that the propagators have a pole at the physical mass. \\
Let's start with the electron propagator. The dressed propagator is given by a sum of so called 1-particle irreducible insertions (i.e. insertions of subdiagrams which do not fall apart when one of the internal lines is cut) as follows
\begin{align*}
\begin{gathered}
\feynmandiagram [small, layered layout, horizontal=a to c] {
	a -- [fermion] b [blob] -- [fermion] c,
};
\end{gathered}
=
\begin{gathered}
\feynmandiagram [small, layered layout, horizontal=a to b] {
	a -- [fermion] b ,
};
\end{gathered}
+
\begin{gathered}
\feynmandiagram [small, layered layout, horizontal=a to c] {
	a -- [fermion] b [empty dot,scale=5] -- [fermion] c,
};
\end{gathered}
+
\begin{gathered}
\feynmandiagram [small, layered layout, horizontal=a to d] {
	a -- [fermion] b [empty dot,scale=5] -- [fermion] c [empty dot,scale=5] -- [fermion] d,
};
\end{gathered}
+ \dots
\end{align*} 
where the empty circles on the right represent renormalized 1-PI interactions and the appropriate counter terms. This gives
\begin{equation}
iS_0(\fsl{p}) = iS(\fsl{p}) + iS(\fsl{p}) i\Sigma'(\fsl{p}) iS(\fsl{p}) + iS(\fsl{p}) i\Sigma'(\fsl{p}) iS(\fsl{p}) i\Sigma'(\fsl{p}) iS(\fsl{p}) + \dots
\end{equation}
where $i\Sigma'(\fsl{p}) = i\Sigma(\fsl{p}) + i \left( \delta_2 \fsl{p} - \left( \delta_2 + \delta_m \right) m \right)$, $iS_0 = \frac{i}{\fsl{p} - m_0}$ and $iS = \frac{i}{\fsl{p} - m}$. Now we can sum the geometric series in $i\Sigma'(\fsl{p})iS(\fsl{p})$ which yields
\begin{equation}
\frac{i}{\fsl{p}-m_0} = \frac{i}{\fsl{p} - m + \left( \Sigma(\fsl{p}) + \delta_2 \fsl{p} - \left( \delta_2 + \delta_m \right) m \right)}
\end{equation}
By requiring the dressed propagator to have a pole at the physical mass $\fsl{p} = m_{\mathrm{phys}} = m$ we obtain 
\begin{equation}
m - m + \Sigma(m) + \delta_2 m - \left( \delta_2 + \delta_m \right) m = 0
\end{equation}
\begin{equation}
\label{eqn:dm}
\Rightarrow \delta_m = \frac{1}{m} \Sigma(m)
\end{equation}
We also want the propagator to have a residue of unity at the pole. This gives the renormalization condition for the electron field
\begin{align*}
\mathrm{Res}_{\fsl{p} = m} \left( S(\fsl{p}) \right) & = \mathrm{Res}_{\fsl{p} = m} \left( \frac{1}{\fsl{p} - m + \left( \Sigma(\fsl{p}) + \delta_2 \fsl{p} - \left( \delta_2 + \delta_m \right) m \right)} \right) = & \\
& = \lim_{\fsl{p} \rightarrow m} \frac{\fsl{p} - m}{\fsl{p} - m + \left( \Sigma(\fsl{p}) + \delta_2 \fsl{p} - \left( \delta_2 + \delta_m \right) m \right)} \overset{\mathrm{L'H}}{=} \lim_{\fsl{p} \rightarrow m} \frac{1}{1 + \frac{d\Sigma}{d\fsl{p}} + \delta_2} \overset{\mathrm{!}}{=} 1 &
\end{align*}
\begin{equation}
\label{eqn:d2}
\Rightarrow \delta_2 = - \frac{d\Sigma(\fsl{p})}{d\fsl{p}} \biggr\rvert_{\fsl{p} = m}
\end{equation}
$Z_1$ and $Z_2$ are related by symmetry, so we do not have to evaluate the electron-photon 3-point function. It was first shown by Ward in 1950 that $Z_1 = Z_2$ \cite{WardId}. \\
The only remaining renormalization constant from equations \ref{eqn:RenormCon} is therefore $Z_3$. It can be obtained from the dressed photon propagator in the same way we obtained the electron field renormalization from the electron propagator. The dressed photon operator is given by
\begin{align*}
\begin{gathered}
\feynmandiagram [small, layered layout, horizontal=a to c] {
	a -- [photon] b [blob] -- [photon] c,
};
\end{gathered}
=
\begin{gathered}
\feynmandiagram [small, layered layout, horizontal=a to b] {
	a -- [photon] b,
};
\end{gathered}
+
\begin{gathered}
\feynmandiagram [small, layered layout, horizontal=a to c] {
	a -- [photon] b [empty dot,scale=5] -- [photon] c,
};
\end{gathered}
+
\begin{gathered}
\feynmandiagram [small, layered layout, horizontal=a to d] {
	a -- [photon] b [empty dot,scale=5] -- [photon] c [empty dot,scale=5] -- [photon] d,
};
\end{gathered}
+ \dots
\end{align*} 
where the empty circles are again insertions of 1-Pi diagrams and the appropriate counter term. So, we have
\begin{equation}
\label{eqn:PhotonProp}
iS^{\alpha\beta}_0(p^2) = iS^{\alpha\beta}(p^2) + \left[iS(p^2) i\Pi'(p^2) iS(p^2)\right]^{\alpha\beta} + \left[iS(p^2) i\Pi'(p^2) iS(p^2) i\Pi'(p^2) iS(p^2)\right]^{\alpha\beta} + \dots
\end{equation}
with $iS^{\alpha\beta}_0 = \frac{-i}{p^2}\left( g^{\alpha\beta} - \frac{p^{\alpha}p^{\beta}}{p^2} \right) = iS^{\alpha\beta}$ and $i\Pi'^{\alpha\beta} = i\Pi^{\alpha\beta} + i \delta_3 \left( p^{\alpha}p^{\beta} - g^{\alpha\beta}p^2 \right)$. Due to gauge invariance and the respective Ward identity we must have $\Pi^{\alpha\beta} = \left( p^{\alpha}p^{\beta} - p^2 g^{\alpha\beta} \right) \Pi(p^2)$, since the Ward identity demands $p_{\alpha}\Pi^{\alpha\beta} = 0 = \left( p^2 p^{\beta} - p^2 p^{\beta} \right) \Pi(p^2)$ \checkmark. So in total we have $i\Pi^{\alpha\beta} = \left( p^{\alpha}p^{\beta} - p^2 g^{\alpha\beta} \right) \left( \Pi(p^2) + \delta_3 \right) \equiv \left( p^{\alpha}p^{\beta} - p^2 g^{\alpha\beta} \right) \Pi'(p^2)$.\\
Now we can sum the geometric series in $i\Pi'(p^2)iS(p^2)$ which yields
\begin{equation}
\frac{-i}{p^2}\left( g^{\alpha\beta} - \frac{p^{\alpha}p^{\beta}}{p^2} \right) = \left( g^{\alpha\beta} - \frac{p^{\alpha}p^{\beta}}{p^2} \right) \frac{-i}{p^2 \left( 1 + \Pi(p^2) + \delta_3 \right)}
\end{equation}
By requiring the propagator to have a residue of unity at the physical photon mass $p^2=0$ we get
\begin{align*}
\mathrm{Res}_{p^2 = 0} \left( S(p^2) \right) & = \mathrm{Res}_{p^2=0} \left( \frac{1}{p^2 \left( 1 + \Pi(p^2) + \delta_3 \right)} \right) = & \\
& = \lim_{p^2 \rightarrow 0} \frac{p^2}{p^2 \left( 1 + \Pi(p^2) + \delta_3 \right)} = \frac{1}{1+\Pi(0)+\delta_3} \overset{\mathrm{!}}{=} 1 &
\end{align*}
\begin{equation}
\label{eqn:d3}
\Rightarrow \delta_3 = - \Pi(0)
\end{equation}
The renormalization procedure for the whole Standard Model is obviously a lot more involved, since there are a lot more fields and parameters in the theory. But it still follows the same lines as for the simpler QED case. The whole derivation for the renormalization conditions of the electroweak part of the Standard Model can be found in \cite{SMrenorm}. We will use the results from there and calculate the needed self-energies in section \ref{sec:SMrenorm}. \\

We now have to calculate all of the necessary 2-point functions to evaluate the renormalization constants. Since our goal is to automate 1-loop calculations in QED and their contributions to the Standard Model it is convenient to express the results in terms of scalar integrals (see Appendix \ref{app:Integrals}) which can be easily implemented and evaluated numerically. In the following all momenta will be d-dimensional until we take the limit $d \rightarrow 4$, so we will not indicate the dimensionality of the momenta by bars. Let us start by defining the following 2-point functions
\begin{align*}
B  & = \int \frac{d^dq}{i\pi^2} \frac{1}{\left( q^2-m_0^2 \right) \left( \left( p+q \right)^2 -m_1^2 \right)} \equiv B_0 & \\
B^{\mu} & = \int \frac{d^dq}{i\pi^2} \frac{q^{\mu}}{\left( q^2-m_0^2 \right) \left( \left( p+q \right)^2 -m_1^2 \right)} \equiv B_1 p^{\mu} & \\ \numberthis \label{eqn:2ptPV}
B^{\mu\nu} & = \int \frac{d^dq}{i\pi^2} \frac{q^{\mu}q^{\nu}}{\left( q^2-m_0^2 \right) \left( \left( p+q \right)^2 -m_1^2 \right)} \equiv B_{00} g^{\mu\nu} + B_{11} p^{\mu}p^{\nu} &
\end{align*}
These are all of the integrals we will need in the following. The same procedure can obviously be done for any n-point function. On the right-hand side of the equations we used the Lorentz invariance of the tensor integrals and decomposed them into the appropriate tensors times a so-called form factor. From equation \ref{eqn:2ptPV} we can get the following expressions
\begin{align*}
p_{\mu} B^{\mu} & = \int \frac{d^dq}{i\pi^2} \frac{p \cdot q}{\left( q^2-m_0^2 \right) \left( \left( p+q \right)^2 -m_1^2 \right)} = B_1 p^2 & \\
g_{\mu\nu} B^{\mu\nu} & = \int \frac{d^dq}{i\pi^2} \frac{q^2}{\left( q^2-m_0^2 \right) \left( \left( p+q \right)^2 -m_1^2 \right)} = d B_{00} + B_{11} p^2 & \\
p_{\mu}p_{\nu} B^{\mu\nu} & = \int \frac{d^dq}{i\pi^2} \frac{\left(p \cdot q\right)^2}{\left( q^2-m_0^2 \right) \left( \left( p+q \right)^2 -m_1^2 \right)} = p^2 B_{00} + B_{11} p^4 &
\end{align*}
which are useful for our calculations. We also have to be careful about taking the limit $d \rightarrow 4$ since $d = O(\epsilon)$ and $B_{i} = O(\epsilon^{-1})$. E.g., for $B_0$ we have in dimensional regularisation with the well-known expression for $B_0$ and $d = 4 - \epsilon$
\begin{equation}
d \cdot B_0 = \left( 4 - \epsilon \right) \left( \frac{2}{\epsilon} - \int_0^1 dx \log \left( \frac{\Delta(x,m_i^2,p^2)}{4\pi e^{-\gamma_E}\mu^2} \right) \right) \rightarrow 4 B_0 - \frac{2\epsilon}{\epsilon} = 4B_0 - 2
\end{equation}
The same can be done for the other form factors. This yields \cite{BfctLimits}
\begin{align*}
d \cdot B_0 & \longrightarrow 4 B_0 - 2 & \\
d \cdot B_1 & \longrightarrow 4 B_1 + 1 & \\
d \cdot B_{00} & \longrightarrow 4 B_{00} - \frac{1}{6} \left( p_1^2 - 3 \left( m_0^2 + m_1^2 \right) \right) & \\
d \cdot B_{11} & \longrightarrow 4 B_{11} - \frac{2}{3} & \numberthis \label{eqn:BfctLims}
\end{align*}

\subsection{Renormalization of Pure QED}
\label{sec:QEDren}
{\bf Photon self-energy} \\
Let us start with the renormalization of the photon field. The photon self-energy contribution is given by
\begin{align*}
\begin{gathered}
\feynmandiagram [layered layout, horizontal=b to c] {
	a [particle=\(\alpha\)] -- [photon, momentum=\(p\)] b
	  -- [fermion, half left, looseness=1.5, momentum=\(p+q\)] c
	  -- [fermion, half left, looseness=1.5, momentum=\(q\)] b,
	c -- [photon, momentum=\(p\)] d [particle=\(\beta\)] ,
};
\end{gathered}
& =\int\frac{d^dq}{\left( 2\pi \right)^d} \left( -1 \right) \mathrm{Tr} \lbr ie \gamma^{\alpha} \frac{i \left( \fsl{p}+\fsl{q}+m \right)}{\left( p+q \right)^2 - m^2} ie \gamma^{\beta} \frac{i \left( \fsl{q}+m \right)}{q^2 - m^2} \rbr \equiv i\Pi^{\alpha\beta} &
\end{align*}
Let's work on the trace so we can express the numerator of the 2-point function in terms of scalar integrals.
\begin{align*}
& \Tr \lbr \gamma^{\alpha} \left( \fsl{p} + \fsl{q} + m \right) \gamma^{\beta} \left( \fsl{q} + m \right) \rbr = \Tr \lbr m^2 \gamma^{\alpha}\gamma^{\beta} + \gamma^{\alpha} \left( \fsl{p} + \fsl{q} \right) \gamma^{\beta} \fsl{q} \rbr = & \\
& = 4 \lbr m^2 g^{\alpha\beta} + \left( p+q \right)_{\mu} q_{\nu} \left( g^{\alpha\mu}g^{\beta\nu} - g^{\alpha\beta}g^{\mu\nu} + g^{\alpha\nu}g^{\beta\mu} \right) \rbr = & \\
& = 4 \left( m^2 g^{\alpha\beta} + \left( p+q \right)^{\alpha} q^{\beta} - g^{\alpha\beta} \left( p+q \right) \cdot q + g^{\alpha} \left( p+q \right)^{\beta} \right) &
\end{align*}
where we have used equation \ref{eqn:Tr2g}. Now we can express the integrals in terms of B-functions as follows
\begin{align*}
i\Pi^{\alpha\beta} & = -e^2 \int \frac{d^dq}{\left( 2\pi \right)^d} \frac{4\lbr m^2g^{\alpha\beta} + p^{\alpha}q^{\beta} + q^{\alpha}p^{\beta} + 2 q^{\alpha}q^{\beta} - g^{\alpha\beta} p \cdot q - g^{\alpha\beta} q^2 \rbr}{\left( \left( p+q \right)^2 - m^2 \right) \left( q^2 - m^2 \right)} = & \\
& = -\frac{4ie^2}{16\pi^2} \lbr m^2 B_0 g^{\alpha\beta} + 2 p^{\alpha}p^{\beta} B_1 + 2 \left( B_{11} p^{\alpha}p^{\beta} + B_{00} g^{\alpha\beta} \right) - g^{\alpha\beta} B_1 p^2 - g^{\alpha\beta} \left( d \cdot B_{00} + B_{11} p^2 \right) \rbr = & \\
& = -\frac{ie^2}{4\pi^2} \lbr \left( 2p^{\alpha}p^{\beta} - g^{\alpha\beta} p^2 \right) \left(B_1 + B_{11} \right) + g^{\alpha\beta} \left( m^2 g^{\alpha\beta} B_0 + \left( 2-d  \right) B_{00} \right) \rbr \longrightarrow & \\
& \longrightarrow -\frac{ie^2}{4\pi^2} \lbr \left( 2p^{\alpha}p^{\beta} - g^{\alpha\beta} p^2 \right) \left(B_1 + B_{11} \right) + g^{\alpha\beta} \left( m^2 g^{\alpha\beta} B_0 - 2 B_{00} + \frac{1}{6} \left( p^2 - 6m^2 \right) \right) \rbr  &
\end{align*}
The arguments of the scalar integrals are suppressed to keep the notation compact. They are the same for all B-functions: $B_i = B_i (p^2,m^2,m^2)$. \\
The expression can be further simplified using identities between the scalar integrals.\\

{\bf Electron self-energy} \\
For the renormalization of the electron field and electron mass we have to evaluate the following diagram
\begin{align*}
\begin{gathered}
\feynmandiagram [layered layout, horizontal=b to c] {
	a -- [fermion, momentum=\(p\)] b,
	c -- [photon, half left, looseness=1.5, momentum=\(q\)] b,
	b -- [fermion, momentum=\(p+q\)] c,
	c -- [fermion, momentum=\(p\)] d,
};
\end{gathered}
& =\int\frac{d^dq}{\left( 2\pi \right)^d} ie \gamma^{\alpha} \frac{i \left( \fsl{p} +\fsl{q}+m \right)}{\left( p+q \right)^2 - m^2} ie \gamma^{\beta} \frac{-i g_{\alpha\beta}}{q^2} =\int\frac{d^dq}{\left( 2\pi \right)^d} \left( -e^2 \right) \gamma^{\alpha} \frac{\left( \fsl{p}+\fsl{q}+m \right)}{\left( p+q \right)^2 - m^2} \gamma_{\alpha} \frac{1}{q^2} \equiv i\Sigma(\fsl{p}) &
\end{align*}
With a bit of gamma-matrix algebra the numerator can be written as
\begin{align*}
\gamma^{\beta} \left( \fsl{p} + \fsl{q} + m \right) \gamma_{\beta} = \left( p+q \right)_{\alpha} \gamma^{\beta}\gamma^{\alpha}\gamma_{\beta} + m \gamma^{\beta}\gamma_{\beta} = d \cdot m + \left( 2-d \right) \left( \fsl{p} + \fsl{q} \right)
\end{align*}
where we used equations \ref{eqn:2g} and \ref{eqn:3g}. For the self-energy we then get
\begin{align*}
i\Sigma(\fsl{p}) & = -e^2 \int \frac{d^dq}{\left( 2\pi \right)^d} \frac{d \cdot m + \left( 2-d \right) \left( \fsl{p} + \fsl{q} \right)}{\left( \left( p+q \right)^2 - m^2 \right)q^2} = - \frac{ie^2}{16\pi^2} \lbr d \cdot m B_0 + \left( 2-d \right) \fsl{p} \left( B_0 + B_1 \right) \rbr \longrightarrow & \\
& \longrightarrow \frac{-ie^2}{16\pi^2} \lbr m \left( 4B_0 - 2 \right) + \fsl{p} \left( 2 \left( B_0 + B_1 \right) - \left( 4B_0 - 2 + 4B_1 + 1 \right) \right) \rbr = \frac{-ie^2}{16\pi^2} \lbr m \left( 4B_0 -2 \right) - \fsl{p} \left( 2 \left( B_0 + B_1 \right) - 1 \right) \rbr &
\end{align*}
Where the arguments of the B-functions are suppressed again. They are $B_i = B_i(p^2,0,m^2)$. \\
This concludes the calculations for the renormalization of pure QED at 1-loop. With the help of equations \ref{eqn:dm}, \ref{eqn:d2} and \ref{eqn:d3}, we found the expressions for an effective Lagrangian which absorbs all infinities at 1-loop level. \\

\subsection{QED Contribution to the Renormalization of the Standard Model}
\label{sec:SMrenorm}
Now, we can continue with the more involved renormalization of the whole SM. In addition to the self-energies that already appear for pure QED, we also have to calculate mixed self-energies that also appeared for the rational terms in the SM. \\

{\bf Z-Boson self-energy} \\
The first addition to renormalization in the full Standard Model is the renormalization of the additional neutral vector boson which gets a contribution from QED. It is given by
\begin{align*}
\begin{gathered}
\feynmandiagram [layered layout, horizontal=b to c] {
	a [particle={\(Z,\alpha\)}] -- [photon, momentum=\(p\)] b
	  -- [fermion, half left, looseness=1.5, momentum=\(p+q\)] c
	  -- [fermion, half left, looseness=1.5, momentum=\(q\)] b,
	c -- [photon, momentum=\(p\)] d [particle={\(\beta,Z\)}] ,
};
\end{gathered}
& = \int\frac{d^dq}{\left( 2\pi \right)^d} \left( -1 \right) \Tr \lbr \frac{ig}{\cos\theta_W} \gamma^{\alpha} \left( g_V - g_A \gamma_5 \right) \frac{i \left( \fsl{p}+\fsl{q}+m_f \right)}{\left( p+q \right)^2 - m^2_f} \frac{ig}{\cos\theta_W} \gamma^{\beta} \times \right. & \\ 
& \left. \times \left( g_V - g_A \gamma_5 \right) \frac{i \left( \fsl{q}+m_f \right)}{q^2 - m^2_f} \rbr \equiv i\Pi^{\alpha\beta}_{ZZ} &
\end{align*}
Let's work on the trace so we can express the numerator of the 2-point function in terms of scalar integrals.

\begin{align*}
& \Tr \lbr \gamma^{\alpha} \left( g_V - g_A \gamma_5 \right) \left( \fsl{p}+\fsl{q}+m_f \right) \gamma^{\beta} \left( g_V - g_A \gamma_5 \right) \left( \fsl{q}+m_f \right) \rbr = \Tr \lbr \gamma^{\alpha} \left( g_V - g_A \gamma_5 \right)^2 \left( \fsl{p}+\fsl{q}-m_f \right) \gamma^{\beta} \left( \fsl{q}+m_f \right) \rbr = & \\
& = \left( g_V^2 + g_A^2 \right) \Tr \lbr \gamma^{\alpha} \left( \fsl{p} + \fsl{q} - m_f \right) \gamma^{\beta} \left( \fsl{q} + m_f \right) \rbr = \left( g_V^2 + g_A^2 \right) \Tr \lbr \gamma^{\alpha} \left( \fsl{p} + \fsl{q} \right) \gamma^{\beta} \fsl{q} - m^2_f \gamma^{\alpha}\gamma^{\beta} \rbr = & \\
& = \left( g_V^2 + g_A^2 \right) \lbr \left( p+q \right)_{\mu} q_{\nu} 4 \left( g^{\alpha\mu}g^{\beta\nu} - g^{\alpha\beta}g^{\mu\nu} + g^{\alpha\nu}g^{\beta\mu} \right) - 4m^2_f g^{\alpha\beta} \rbr = & \\
& = 4 \left( g_V^2 + g_A^2 \right) \lbr \left( p+q \right)^{\alpha}q^{\beta} - g^{\alpha\beta} \left( p+q \right) \cdot q + q^{\alpha} \left( p+q \right)^{\beta} - m^2_f g^{\alpha\beta} \rbr &
\end{align*}
Here, we used equations \ref{eqn:Tr2g} and \ref{eqn:Tr4g}. We can now express the integrals in terms of B-functions
\begin{align*}
i\Pi^{\alpha\beta}_{ZZ} & = \int \frac{d^4q}{\left( 2\pi \right)^4} \frac{4g^2\left(g_V^2 + g_A^2 \right)}{\cos^2\theta_W} \frac{\left( p+q \right)^{\alpha}q^{\beta} - g^{\alpha\beta} \left( p+q \right) \cdot q + q^{\alpha} \left( p+q \right)^{\beta} - m^2_f g^{\alpha\beta}}{\left( \left( p+q \right)^2 -m^2_f \right) \left( q^2 - m^2_f \right)} = & \\
& = \frac{4g^2\left(g_V^2 + g_A^2 \right)}{\cos^2\theta_W} \int \frac{d^4q}{\left( 2\pi \right)^4} \frac{p^{\alpha}q^{\beta}+q^{\alpha}p^{\beta}+2q^{\alpha}q^{\beta} - g^{\alpha\beta} \left( m^2_f + q \cdot p + q^2 \right)}{\left( \left( p+q \right)^2 -m^2_f \right) \left( q^2 - m^2_f \right)} = & 
\end{align*}
\begin{align*}
& = \frac{4g^2\left(g_V^2 + g_A^2 \right)}{\cos^2\theta_W} \frac{i\pi^2}{\left( 2\pi \right)^4} \lbr 2 p^{\alpha}p^{\beta} B_1 + 2 \left( B_{00} g^{\alpha\beta} + B_{11} p^{\alpha}p^{\beta} \right) - g^{\alpha\beta} \left( m^2_f B_0 + B_1 p^2 + d B_{00} + B_{11} p^2 \right) \rbr = & \\
& = \frac{ig^2\left(g_V^2 + g_A^2 \right)}{4\pi^2\cos^2\theta_W} \lbr \left( 2p^{\alpha}p^{\beta} -p^2 g^{\alpha\beta} \right) \left( B_1 + B_{11} \right) + g^{\alpha\beta} \left( \left( 2-d \right) B_{00} - m^2_f B_0  \right) \rbr \longrightarrow & \\
& \longrightarrow \frac{ig^2\left(g_V^2 + g_A^2 \right)}{4\pi^2\cos^2\theta_W} \lbr \left( 2p^{\alpha}p^{\beta} -p^2 g^{\alpha\beta} \right) \left( B_1 + B_{11} \right) - g^{\alpha\beta} \left( 2 B_{00} - \frac{1}{6} \left( p^2 - 6m^2_f \right) + m^2_f B_0  \right) \rbr
\end{align*}
The arguments of the B-functions were suppressed again, they are $B_i = B_i(p^2,m^2_f,m^2_f)$. \\

{\bf Photon/Z-boson mixed self-energy} \\
As for the $R_2$ contribution there is also a mixed contribution to the renormalization of the neutral vector bosons in the Standard Model. It is given by
\begin{align*}
\begin{gathered}
\feynmandiagram [layered layout, horizontal=b to c] {
	a [particle={\(Z,\alpha\)}] -- [photon, momentum=\(p\)] b
	  -- [fermion, half left, looseness=1.5, momentum=\(p+q\)] c
	  -- [fermion, half left, looseness=1.5, momentum=\(q\)] b,
	c -- [photon, momentum=\(p\)] d [particle={\(\beta,\gamma\)}] ,
};
\end{gathered}
& = \int\frac{d^dq}{\left( 2\pi \right)^d} \left( -1 \right) \Tr \lbr \left( -ie Q_f \right) \gamma^{\alpha} \frac{i \left( \fsl{p}+\fsl{q}+m_f \right)}{\left( p+q \right)^2 - m^2_f} i \frac{g}{\cos\theta_W} \gamma^{\beta} \left( g_V - g_A \gamma_5 \right) \times \right. & \\
& \left. \times \frac{i \left( \fsl{q}+m_f \right)}{q^2 - m^2_f} \rbr \equiv i\Pi^{\alpha\beta}_{\gamma Z} &
\end{align*}
Let's work on the trace so we can express the numerator of the 2-point function in terms of scalar integrals.
\begin{align*}
& \Tr \lbr \gamma^{\alpha} \left( \fsl{p} + \fsl{q} + m_f \right) \gamma^{\beta} \left( g_V - g_A \gamma_5 \right) \left( \fsl{q} + m_f \right) \rbr = g_V \Tr \lbr \gamma^{\alpha} \left( \fsl{p} + \fsl{q} \right) \gamma^{\beta} \fsl{q} + m^2_f \gamma^{\alpha}\gamma^{\beta} \rbr - g_A \Tr \lbr \gamma^{\alpha} \left( \fsl{p} + \fsl{q} \right) \gamma^{\beta} \gamma_5 \fsl{q} \rbr = & \\
& = 4 g_V \lbr \left( p+q \right)_{\mu} q_{\nu} \left( g^{\alpha\mu}g^{\beta\nu} - g^{\alpha\beta}g^{\mu\nu} + g^{\alpha\nu}g^{\beta\mu} \right) + m^2_f g^{\alpha\beta} \rbr - 4i g_A \left( p+q \right)_{\mu} q_{\nu} \epsilon^{\alpha\mu\beta\nu} = & \\
& = 4 \lbr g_V \left[ \left( p+q \right)^{\alpha}q^{\beta} - g^{\alpha\beta} \left( p+q \right) \cdot q + q^{\alpha} \left( p+q \right)^{\beta} + m^2_f g^{\alpha\beta} \right] - i g_A \epsilon^{\alpha\mu\beta\nu} p_{\mu}q_{\nu} \rbr &
\end{align*}
Where we used that a symmetric tensor contracted with an antisymmetric tensor vanishes and equation \ref{eqn:Tr4g}. Now we can express $i\Pi^{\alpha\beta}_{\gamma Z}$ in terms of B-functions
\begin{align*}
i\Pi^{\alpha\beta}_{\gamma Z} & = \frac{4Q_feg}{\cos\theta_W} \int \frac{d^dq}{\left( 2\pi \right)^d} \frac{g_V \left( \left( p+q \right)^{\alpha}q^{\beta} - g^{\alpha\beta} \left( p+q \right) \cdot q + q^{\alpha} \left( p+q \right)^{\beta} + m^2_f g^{\alpha\beta} \right) - i g_A \epsilon^{\alpha\mu\beta\nu} p_{\mu}q_{\nu}}{\left( \left( p+q \right)^2 -m^2_f \right) \left( q^2 - m^2_f \right)} = & \\
& = \frac{4Q_feg}{\cos\theta_W} \frac{i\pi^2}{\left( 2\pi \right)^4} \lbr -ig_A \epsilon^{\alpha\mu\beta\nu}p_{\mu}B_1 p_{\nu} + g_V \left[ B_1 p^{\alpha}p^{\beta} + B_{00}g^{\alpha\beta} + B_{11} p^{\alpha}p^{\beta} - g^{\alpha\beta} \left( B_1 p^2 + d B_{00} + B_{11} p^2 \right) + \right. \right. & \\
& \left. \left. + B_1 p^{\alpha}p^{\beta} + B_{00} g^{\alpha\beta} + B_{11} p^{\alpha}p^{\beta} + B_0 m^2_f g^{\alpha\beta} \right]\rbr = & \\
& = \frac{iQ_fegg_V}{4\pi^2\cos\theta_W} \lbr 2 p^{\alpha}p^{\beta} \left( B_1 + B_{11} \right) + g^{\alpha\beta} \left( m^2_f B_0 + \left( 2 - d \right) B_{00} -p^2 \left( B_1 + B_{11} \right) \right) \rbr \longrightarrow & \\
& \longrightarrow \frac{iQ_fegg_V}{4\pi^2\cos\theta_W} \lbr 2 p^{\alpha}p^{\beta} \left( B_1 + B_{11} \right) + g^{\alpha\beta} \left( m^2_f B_0 - 2 B_{00} + \frac{1}{6} \left( p^2 - 6m^2 \right) -p^2 \left( B_1 + B_{11} \right) \right) \rbr = & \\
& = \frac{iQ_fegg_V}{4\pi^2\cos\theta_W} \lbr \left( 2 p^{\alpha}p^{\beta} - p^2g^{\alpha\beta} \right) \left( B_1 + B_{11} \right) + g^{\alpha\beta} \left( m^2_f B_0 - 2 B_{00} + \frac{1}{6} \left( p^2 - 6m^2_f \right) \right) \rbr &
\end{align*}
From the second to third line we used that a symmetric tensor contracted with an antisymmetric tensor vanishes. The arguments of the B-functions are $B_i = B_i(p^2,m^2_f,m^2_f)$.\\

{\bf Gluon self-energy} \\
The gluon self-energy contribution can again be extracted from the photonic contribution by exchanging the vector couplings and noticing that gluons only couple to quarks.
\begin{align*}
& i\Pi^{\alpha\beta}_{gg} = 
\begin{gathered}
\feynmandiagram [layered layout, horizontal=b to c] {
	a [particle=\(\alpha\)] -- [gluon, momentum=\(p\)] b
	  -- [fermion, half left, looseness=1.5, momentum=\(p+q\)] c
	  -- [fermion, half left, looseness=1.5, momentum=\(q\)] b,
	c -- [gluon, momentum=\(p\)] d [particle=\(\beta\)] ,
};
\end{gathered}
=
\begin{gathered}
\feynmandiagram [layered layout, horizontal=b to c] {
	a [particle=\(\alpha\)] -- [photon, momentum=\(p\)] b
	  -- [fermion, half left, looseness=1.5, momentum=\(p+q\)] c
	  -- [fermion, half left, looseness=1.5, momentum=\(q\)] b,
	c -- [photon, momentum=\(p\)] d [particle=\(\beta\)] ,
};
\end{gathered}
\left( eQ_f \rightarrow g_S T^a, \ m \rightarrow m_q \right) = & \\
& = -\frac{ig_S^2\Tr\left( T^a T^b\right)}{4\pi^2} \lbr \left( 2p^{\alpha}p^{\beta} - g^{\alpha\beta} p^2 \right) \left(B_1 + B_{11} \right) + g^{\alpha\beta} \left( m^2_q g^{\alpha\beta} B_0 - 2 B_{00} + \frac{1}{6} \left( p^2 - 6m^2_q \right) \right) \rbr &
\end{align*}
Here, the arguments of the B-functions are $B_i = B_i(p^2,m^2_q,m^2_q)$.
\newline\newline\newline\newline\newline

\appendix
{\bf\Large Appendices} \\
\section{Feynman Rules}
\label{sec:FeynmanRules}
In this appendix all of the Feynman rules which were used for the calculations are listed. The Feynman rules are given for the whole Standard Model, but the pure QED Feynman rules can be obtained by taking $Q_f \rightarrow Q_e = -1$ and $m_f \rightarrow m_e$. \\

{\bf Propagator}
\begin{equation*}
\begin{gathered}
\feynmandiagram [horizontal=a to b] {
	a -- [fermion,momentum=\(p\)] b ,
};
\end{gathered}
= \frac{i\left( \fsl{p} + m_f \right)}{p^2 - m_f^2+i\epsilon} \qquad\qquad\qquad
\begin{gathered}
\feynmandiagram [horizontal=a to b] {
	a [particle=\alpha] -- [photon,momentum=\(p\)] b [particle=\beta],
};
\end{gathered}
= \frac{-ig^{\alpha\beta}}{p^2+i\epsilon} \qquad\text{in 't Hooft-Feynman gauge}
\end{equation*}\\

{\bf Interactions}\\
\begin{equation*}
\begin{gathered}
\feynmandiagram [horizontal=a to b] {
	a -- [scalar] b ,
	c -- [fermion] b -- [fermion] d,
};
\end{gathered}
= \frac{-ig}{2}\frac{m_f}{m_W} \qquad\qquad\qquad\qquad\qquad
\begin{gathered}
\feynmandiagram [horizontal=a to b] {
	a [particle={\gamma,\alpha}] -- [photon] b ,
	c -- [fermion] b -- [fermion] d,
};
\end{gathered}
= -ieQ_f\gamma^{\alpha}
\end{equation*}
\begin{equation*}
\begin{gathered}
\feynmandiagram [horizontal=a to b] {
	a [particle={Z,\alpha}] -- [photon] b ,
	c -- [fermion] b -- [fermion] d,
};
\end{gathered}
= \frac{ig}{\cos\theta_W}\gamma^{\alpha} \left(g_V-g_A \gamma_5\right) \qquad\qquad\qquad
\begin{gathered}
\feynmandiagram [horizontal=a to b] {
	a [particle={a,\alpha}] -- [gluon] b ,
	c [particle=i] -- [fermion] b -- [fermion] d [particle=j],
};
\end{gathered}
= ig_S T^a_{ij}\gamma^{\alpha}
\end{equation*}


\section{Important Integrals}
\label{app:Integrals}
In the calculation of $R_2$ we have to evaluate 2-,3- and 4-point functions. They can be reduced to a set of integrals which are knwon in a general form. The integrals we need are \cite{R2QCD} \\

{\bf 2-point integrals}
\begin{equation}
\label{eqn:2ptintq2} 
\int  d^d \bar{q} \frac{\tilde{q}^2}{\bar{D}_i\bar{D}_j} =  -\frac{i\pi^2}{2} \left[ m_i^2 + m_j^2 - \frac{\left( p_i - p_j \right)^2}{3} \right] + O(\epsilon)
\end{equation}
\begin{equation}
\label{eqn:2ptintsca} 
\mathrm{P.P.} \left( \int  d^d \bar{q} \frac{1}{\bar{D}_i\bar{D}_j} \right) = -2\frac{i\pi^2}{\epsilon}
\end{equation}
\begin{equation}
\label{eqn:2ptintq} 
\mathrm{P.P.} \left( \int  d^d \bar{q} \frac{q_{\mu}}{\bar{D}_i\bar{D}_j} \right) =  \frac{i\pi^2}{\epsilon} \left( p_i + p_j \right)_{\mu}
\end{equation}\\

{\bf 3-point integrals} 
\begin{equation}
\label{eqn:3ptintq2}
\int  d^d \bar{q} \frac{\tilde{q}^2}{\bar{D}_i\bar{D}_j\bar{D}_k} = -\frac{i\pi^2}{2} + O(\epsilon)
\end{equation}
\begin{equation}
\label{eqn:3ptintq3}
\int  d^d \bar{q} \frac{\tilde{q}^2q_{\mu}}{\bar{D}_i\bar{D}_j\bar{D}_k} = \frac{i\pi^2}{6} \left( p_i + p_j + p_k \right)_{\mu} + O(\epsilon)
\end{equation}
\begin{equation}
\label{eqn:3ptintq2vec}
\mathrm{P.P.} \left( \int  d^d \bar{q} \frac{q_{\mu}q_{\nu}}{\bar{D}_i\bar{D}_j\bar{D}_k} \right) =  -\frac{i\pi^2}{2\epsilon} g_{\mu\nu}
\end{equation}\\

{\bf 4-point integrals} 
\begin{equation}
\label{eqn:4ptintq4}
\int  d^d \bar{q} \frac{\tilde{q}^4}{\bar{D}_i\bar{D}_j\bar{D}_k\bar{D}_l} =  -\frac{i\pi^2}{6} + O(\epsilon)
\end{equation}
\begin{equation}
\label{eqn:4ptintvec}
\int  d^d \bar{q} \frac{\tilde{q}^2q_{\mu}q_{\nu}}{\bar{D}_i\bar{D}_j\bar{D}_k\bar{D}_l} =  -\frac{i\pi^2}{12} g_{\mu\nu} + O(\epsilon)
\end{equation}
\begin{equation}
\label{eqn:4ptintq2q2}
\int  d^d \bar{q} \frac{\tilde{q}^2 q^2}{\bar{D}_i\bar{D}_j\bar{D}_k\bar{D}_l} =  -\frac{i\pi^2}{3} + O(\epsilon) 
\end{equation}

\section{Traceology}
\label{app:Traceology}
In a theory with fermions the Dirac matrices appear as the generators of the spinor representation of the Poincaré algebra. The following identities for Dirac matrices are very useful when evaluating Feynman diagrams
\begin{itemize}
\item[1.] $\Tr \left( \gamma^{\alpha} \gamma^{\beta} \right) = dg^{\alpha\beta}$ \numberitem\label{eqn:Tr2g}

\item[2.] $\Tr \left( \gamma^{\alpha}\gamma^{\beta}\gamma^{\gamma}\gamma^{\delta} \right) = d \left( g^{\alpha\beta}g^{\gamma\delta} - g^{\alpha\gamma}g^{\beta\delta} + g^{\alpha\delta}g^{\beta\gamma} \right)$ \numberitem\label{eqn:Tr4g}

\item[3.] $\gamma^{\alpha}\gamma_{\alpha} = d$ \numberitem\label{eqn:2g}

\item[4.] $\gamma^{\alpha}\gamma^{\beta}\gamma_{\alpha} = \left( 2-d \right) \gamma^{\beta}$ \numberitem\label{eqn:3g}

\item[5.] $\gamma^{\alpha}\gamma^{\beta}\gamma^{\gamma}\gamma_{\alpha} = d\gamma^{\beta}\gamma^{\gamma} + 2 \left[ \gamma^{\gamma}, \gamma^{\beta} \right]$ \numberitem\label{eqn:4g}

\item[6.] $\fsl{a}\fsl{b} = a \cdot b$ \numberitem\label{eqn:fsldot}

\end{itemize}
The Dirac matrices obey the Clifford algebra $\lbr \gamma^{\mu}, \gamma^{\nu} \rbr = 2 g^{\mu\nu} \mathds{1}_d$ with $g^{\mu\nu}$ the Minkowski metric in $d$ dimensions
\begin{equation*}
 g^{\mu\nu} = 
\begin{cases}
1 \quad\qquad\qquad \text{for } \mu = \nu = 0 \\
-1 \ \qquad\qquad \text{for } \mu = \nu = 1,2,\ldots,d-1 \\
0 \quad\qquad\qquad \text{for } \mu \neq \nu
\end{cases}
\end{equation*}



{\bf Proofs for identities}

1. $\mathrm{Tr} \left( \gamma^{\alpha} \gamma^{\beta} \right) = dg^{\alpha\beta}$
\begin{proof}
\begin{equation*}
\mathrm{Tr} \left( \gamma^{\alpha} \gamma^{\beta} \right) = \mathrm{Tr} \left( 2 g^{\alpha\beta} - \gamma^{\beta}\gamma^{\alpha} \right) = 2g^{\alpha\beta} \mathrm{Tr} \left( \mathds{1}_d \right) - \mathrm{Tr} \left( \gamma^{\beta} \gamma^{\alpha} \right) = 2dg^{\alpha\beta} - \mathrm{Tr} \left( \gamma^{\alpha} \gamma^{\beta} \right)
\end{equation*}
\begin{equation*}
\Rightarrow \mathrm{Tr} \left( \gamma^{\alpha} \gamma^{\beta} \right) = dg^{\alpha\beta}
\end{equation*}
\end{proof}

2. $\mathrm{Tr} \left( \gamma^{\alpha}\gamma^{\beta}\gamma^{\gamma}\gamma^{\delta} \right) = d \left( g^{\alpha\beta}g^{\gamma\delta} - g^{\alpha\gamma}g^{\beta\delta} + g^{\alpha\delta}g^{\beta\gamma} \right)$
\begin{proof}
\begin{align*}
\mathrm{Tr} \left( \gamma^{\alpha}\gamma^{\beta}\gamma^{\gamma}\gamma^{\delta} \right) & = \mathrm{Tr} \left( \left( 2g^{\alpha\beta} - \gamma^{\beta}\gamma^{\alpha} \right) \gamma^{\gamma} \gamma^{\delta} \right) = 2g^{\alpha\beta} \mathrm{Tr} \left( \gamma^{\gamma}\gamma^{\delta}\right) - \mathrm{Tr} \left( \gamma^{\beta} \left( 2g^{\alpha\gamma} - \gamma^{\gamma} \gamma^{\alpha} \right) \gamma^{\delta} \right) = & \\
& = 2d g^{\alpha\beta} g^{\gamma\delta} - 2g^{\alpha\gamma} \mathrm{Tr} \left( \gamma^{\beta}\gamma^{\delta} \right) + \mathrm{Tr} \left( \gamma^{\beta} \gamma^{\gamma} \left( 2g^{\alpha\delta} - \gamma^{\delta}\gamma^{\alpha} \right) \right) = & \\
& = 2d \left( g^{\alpha\beta} g^{\gamma\delta} - g^{\alpha\gamma} g^{\beta\delta} \right) + 2g^{\alpha\delta} \mathrm{Tr} \left( \gamma^{\beta}\gamma^{\gamma} \right) - \mathrm{Tr} \left( \gamma^{\beta}\gamma^{\gamma}\gamma^{\delta}\gamma^{\alpha} \right) = & \\
& = 2d \left( g^{\alpha\beta} g^{\gamma\delta} - g^{\alpha\gamma}g^{\beta\delta} + g^{\alpha\delta} g^{\beta\gamma} \right) - \mathrm{Tr} \left( \gamma^{\alpha}\gamma^{\beta}\gamma^{\gamma}\gamma^{\delta} \right) & 
\end{align*}
\begin{equation*}
\Rightarrow \mathrm{Tr} \left( \gamma^{\alpha}\gamma^{\beta}\gamma^{\gamma}\gamma^{\delta} \right) = d \left( g^{\alpha\beta}g^{\gamma\delta} - g^{\alpha\gamma}g^{\beta\delta} + g^{\alpha\delta}g^{\beta\gamma} \right)
\end{equation*}
\end{proof}


3. $\gamma^{\alpha}\gamma_{\alpha} = d$
\begin{proof}
\begin{equation*}
\gamma^{\alpha}\gamma_{\alpha} = \frac{1}{2} \left( \gamma^{\alpha}\gamma_{\alpha} + \gamma_{\alpha}\gamma^{\alpha} \right) = \frac{1}{2} \lbr \gamma^{\alpha}, \gamma_{\alpha} \rbr = \frac{1}{2} 2 g^{\alpha}_{ \ \alpha} = d
\end{equation*}
\end{proof}

4. $\gamma^{\alpha}\gamma^{\beta}\gamma_{\alpha} = \left( 2-d \right) \gamma^{\beta}$
\begin{proof}
\begin{equation*}
\gamma^{\alpha}\gamma^{\beta}\gamma_{\alpha} = \left( 2g^{\alpha\beta} - \gamma^{\beta}\gamma^{\alpha} \right) \gamma_{\alpha} = \left( 2-d \right) \gamma^{\beta}
\end{equation*}
\end{proof}

5. $\gamma^{\alpha}\gamma^{\beta}\gamma^{\gamma}\gamma_{\alpha} = d\gamma^{\beta}\gamma^{\gamma} + 2 \left[ \gamma^{\gamma}, \gamma^{\beta} \right]$
\begin{proof}
\begin{equation*}
\gamma^{\alpha}\gamma^{\beta}\gamma^{\gamma}\gamma_{\alpha} = \left( 2g^{\alpha\beta} - \gamma^{\beta}\gamma^{\alpha} \right) \gamma^{\gamma}\gamma_{\alpha} \overset{4.}{=} 2 \left( \gamma^{\gamma}\gamma^{\beta} - \gamma^{\beta}\gamma^{\gamma} \right) + d\gamma^{\beta}\gamma^{\gamma} = d\gamma^{\beta}\gamma^{\gamma} + 2 \left[ \gamma^{\gamma}, \gamma^{\beta} \right]		
\end{equation*}
\end{proof}

6. $\fsl{a}\fsl{b} = a \cdot b$
\begin{proof}
\begin{equation*}
\fsl{a} \fsl{b} = a_{\alpha} b_{\beta} \gamma^{\alpha}\gamma^{\beta} = a_{\alpha} b_{\beta} \left( 2 g^{\alpha\beta} - \gamma^{\beta}\gamma^{\alpha} \right) = 2 a \cdot b - \fsl{a}\fsl{b}
\end{equation*}
\begin{equation*}
\Rightarrow \fsl{a}\fsl{b} = a \cdot b
\end{equation*}
\end{proof}


\section{Relation Between Left- \& Right-handed Currents and Axial \& Vector Currents}
\label{app:Currents}
A classical Lagrangian permits symmetries which can be implemented by Lie groups $G$. An element $g \in G$ of a Lie group can be parametrized as $g = \mathrm{exp} \left(i \alpha^a T^a \right)$ where $\alpha^a$ are real parameters and $T^a$ the generators of the Lie group. Noether's theorem predicts a clasically conserved current for each generator of a continuous symmetry. For a field $\phi$ with trafo $\delta\phi = \phi' - \phi = g \phi - \phi \approx \left( 1 + i \alpha^a T^a \right) \phi - \phi = i \alpha^a T^a \phi$ the conserved current can be shown to be
\begin{equation*}
j^{\mu a} = \frac{\partial \mathcal{L}}{\partial \left( \partial_{\mu} \phi \right)} \frac{\partial \delta\phi}{\partial \alpha_a}
\end{equation*} 
The left- and right-handed part $j^{\mu a}_{L/R}$ of a fermionic current are
\begin{equation*}
j^{\mu a}_L = \bar{\psi}_L \gamma^{\mu} T^a \psi_L = \left( P_L \psi \right)^{\dagger} \gamma^0 \gamma^{\mu} T^a P_L \psi \overset{P_L^{\dagger}=P_L}{=} \psi^{\dagger} P_L \gamma^0 \gamma^{\mu} T^a P_L \psi = \bar{\psi} \gamma^{\mu} T^a P_L^2 \psi \overset{P_L^2 = P_L}{=} \bar{\psi} \gamma^{\mu} T^a P_L \psi
\end{equation*}
\begin{equation*}
j^{\mu a}_R = \bar{\psi}_R \gamma^{\mu} T^a \psi_R = \bar{\psi} \gamma^{\mu} T^a P_R \psi
\end{equation*}
where $P_{L/R} = \frac{1}{2} \left( 1 \mp \gamma_5 \right)$ is the left-/right-handed projector. \\
From the left- and right-handed currents we can define axial-vector and vector currents
\begin{equation*}
j^{\mu a} = j^{\mu a}_R + j^{\mu a}_L = \bar{\psi} \gamma^{\mu} T^a \left( P_R + P_L \right) = \bar{\psi} \gamma^{\mu} T^a \psi
\end{equation*}
\begin{equation*}
j^{\mu a}_5 = j^{\mu a}_R - j^{\mu a}_L = \bar{\psi} \gamma^{\mu} T^a \left( P_R - P_L \right) \psi = \bar{\psi} \gamma^{\mu} T^a \gamma_5 \psi
\end{equation*}
Now we can couple the currents to vector fields to obtain interactions. E.g., the vector coupling in QED is given by the Lagrangian
\begin{equation*}
\mathcal{L}_{coupl}^{QED} = e A_ {\mu} j^{\mu} = e A_{\mu} \bar{\psi} \gamma^{\mu} Q_e \psi = -e A_{\mu} \bar{\psi} \gamma^{\mu} \psi 
\end{equation*}
In general, we can couple any linear combination of currents to a vector field as long as the combination is Lorentz and gauge invariant. E.g., the neutral current in the electroweak theory is a superposition of a vector and an axialvector current
\begin{equation*}
\mathcal{L}_{coupl}^{NC} = g Z_{\mu} \left( g_V j^{\mu} - g_A j^{\mu}_5 \right)
\end{equation*}
We can use the above relations to express this coupling in terms of right- and left-handed currents
\begin{align*}
\mathcal{L}_{coupl}^{NC} & = g Z_{\mu} \left( g_V \bar{\psi} \gamma^{\mu} \psi - g_A \bar{\psi} \gamma^{\mu} \gamma_5 \psi \right) = & \\
& = g Z_{\mu} \left( g_V \bar{\psi} \gamma^{\mu} \psi + \frac{g_A}{2} \bar{\psi} \gamma^{\mu} \psi - \frac{g_A}{2} \bar{\psi} \gamma^{\mu} \psi - g_A \bar{\psi} \gamma^{\mu} \gamma_5 \psi + \frac{g_V}{2} \bar{\psi} \gamma^{\mu} \gamma_5 \psi - \frac{g_V}{2} \bar{\psi} \gamma^{\mu} \gamma_5 \psi \right) = & \\
& = g Z_{\mu} \left( \left( g_V + g_A \right) \bar{\psi} \gamma^{\mu} \frac{1}{2} \left( 1 - \gamma_5 \right) \psi + \left( g_V - g_A \right) \bar{\psi} \gamma^{\mu} \frac{1}{2} \left( 1 + \gamma_5 \right) \psi \right) = & \\
& = g Z_{\mu} \left( \left( g_V + g_A \right) \bar{\psi} \gamma^{\mu} P_L \psi + \left( g_V - g_A \right) \bar{\psi} \gamma^{\mu} P_R \psi \right) \equiv g Z_{\mu} \left( g_L j^{\mu}_L + g_R j^{\mu}_R \right)  &
\end{align*}
This gives the following relation between the (axial-)vector and the left-/right-handed couplings
\begin{align*}
&g_L = g_V + g_A &\\
&g_R = g_V - g_A &
\end{align*}


\bibliographystyle{unsrt}  
\bibliography{references}


\end{document}
