\section{R$_2$ in Pure QED}
\label{sec:pureQED} 
Before we calculate anything in the Standard Model, let us start with the $R_2$ of pure QED. We have to consider all n-point functions up to $n=4$ which are allowed by the Feynman rules of QED and calculate their contribution to equation \ref{eqn:R2}.
\subsection{2-point functions}
The Feynman rules of QED allow two 2-point functions; the self-energy diagrams of the photon and the electron. Let us start with the photon self-energy which has the simplest Lorentz structure and therefore an easy to evaluate numerator function. \\

{\bf Photon self-energy} \\
The photon 2-point function is given by
\begin{align*}
\begin{gathered}
\feynmandiagram [layered layout, horizontal=b to c] {
	a [particle=\(\alpha\)] -- [photon, momentum=\(p_1\)] b
	  -- [fermion, half left, looseness=1.5, momentum=\(p_1+q\)] c
	  -- [fermion, half left, looseness=1.5, momentum=\(q\)] b,
	c -- [photon, momentum=\(p_1\)] d [particle=\(\beta\)] ,
};
\end{gathered} \qquad
& =\int\frac{d^d\bar{q}}{\left( 2\pi \right)^d} \left( -1 \right) \mathrm{Tr} \lbr ie \bar{\gamma}^{\alpha} \frac{i \left( \bar{\fsl{p}}_1+\bar{\fsl{q}}+m \right)}{\left( p_1+q \right)^2 - m^2} ie \bar{\gamma}^{\beta} \frac{i \left( \bar{\fsl{q}}+m \right)}{q^2 - m^2} \rbr & \\
& \equiv \int\frac{d^d\bar{q}}{\left( 2\pi \right)^d} \frac{\bar{N}}{\bar{D}_1\bar{D}_0} &
\end{align*}
Where we defined the numerator and denominator functions in the last step. Now we can extract the $\epsilon$-dimensional contribution from the numerator
\begin{equation*}
\bar{N} (\bar{q}) = - e^2 \ \mathrm{Tr} \lbr \bar{\gamma}^{\alpha} \left( \bar{\fsl{p}}_1 + \bar{\fsl{q}} + m \right) \bar{\gamma}^{\beta} \left( \bar{\fsl{q}} + m \right) \rbr = - e^2 \ \mathrm{Tr} \lbr \gamma^{\alpha} \left( \fsl{p}_1 + \fsl{q} + m \right) \gamma^{\beta} \left( \fsl{q} + m \right) + \gamma^{\alpha} \wtilde{\fsl{q}} \gamma^{\beta} \wtilde{\fsl{q}} \rbr \equiv N + \wtilde{N}
\end{equation*}
Here, the first term is the 4-dimensional numerator which also appears in normal loop calculations and the second term is the $\epsilon$-dimensional part which we need for the calculation of $R_2$.\\
We can now evaluate the trace and get
\begin{equation*}
\wtilde{N} = -e^2 \ \mathrm{Tr} \lbr \gamma^{\alpha} \wtilde{\fsl{q}} \gamma^{\beta} \wtilde{\fsl{q}} \rbr = 4 e^2 \wtilde{q}^2 g^{\alpha\beta}
\end{equation*}
Where we have used that the $\epsilon$-dimensional gamma matrices anti-commute with the 4-dimensional gamma matrices and the trace identity for 2 gamma matrices (equation \ref{eqn:Tr2g}) which can be found alongside a proof in appendix \ref{app:Traceology}. Plugging the expression for $\wtilde{N}$ in equation \ref{eqn:R2} gives
\begin{equation}
\label{eqn:R2photon}
R_2^{\gamma\gamma} = \frac{1}{\left( 2\pi \right) ^4} \int d^d\bar{q} \frac{\wtilde{N}}{\bar{D}_1\bar{D}_0} = \frac{4e^2}{16\pi^4} \underbrace{\int d^d\bar{q} \frac{\wtilde{q}^2}{\bar{D}_1\bar{D}_0}}_{-i\frac{\pi}{2} \left( 2m^2 - p_1^2/3 \right)} = \frac{-ie^2}{8\pi^2} g^{\alpha\beta} \left( 2m^2 -\frac{p_1^2}{3} \right)
\end{equation}\\
In the last step we have used the 2-point integral \ref{eqn:2ptintq2}.

{\bf Electron self-energy} \\
The other 2-point function in QED is the electron 2-point function which is given by
\begin{align*}
\begin{gathered}
\feynmandiagram [layered layout, horizontal=b to c] {
	a -- [fermion, momentum=\(p_1\)] b,
	c -- [photon, half left, looseness=1.5, momentum=\(q\)] b,
	b -- [fermion, momentum=\(p_1+q\)] c,
	c -- [fermion, momentum=\(p_1\)] d,
};
\end{gathered} \qquad
& =\int\frac{d^dq}{\left( 2\pi \right)^d} ie \gamma^{\alpha} \frac{i \left( \fsl{p}_1+\fsl{q}+m \right)}{\left( p_1+q \right)^2 - m^2} ie \gamma^{\beta} \frac{-i g_{\alpha\beta}}{q^2} =\int\frac{d^dq}{\left( 2\pi \right)^d} \left( -e^2 \right) \gamma^{\alpha} \frac{\left( \fsl{p}_1+\fsl{q}+m \right)}{\left( p_1+q \right)^2 - m^2} \gamma_{\alpha} \frac{1}{q^2} & \\
& \equiv \int\frac{d^dq}{\left( 2\pi \right)^d} \frac{\bar{N}}{\bar{D}_1\bar{D}_0} &
\end{align*}
Now we extract again the $\epsilon$-dimensional part from the numerator function we defined in the last step. We get
\begin{equation*}
\bar{N} (\bar{q}) = \left( -e^2 \right) \bar{\gamma}^{\alpha} \left( \bar{\fsl{p}}_1 + \bar{\fsl{q}} + m \right) \bar{\gamma}_{\alpha} = -e^2 \lbr \gamma^{\alpha} \left( \bar{\fsl{p}}_1 + \bar{\fsl{q}} + m \right) \gamma_{\alpha} + \wtilde{\gamma}^{\alpha} \left( \bar{\fsl{p}}_1 + \bar{\fsl{q}} + m \right) \wtilde{\gamma}_{\alpha} + \gamma^{\alpha} \wtilde{\fsl{q}} \gamma_{\alpha} + \wtilde{\gamma}^{\alpha} \wtilde{\fsl{q}} \wtilde{\gamma}_{\alpha} \rbr \equiv N + \wtilde{N}
\end{equation*}
Here, the first term is again the normal 4-dimensional numerator and the rest the $\epsilon$-dimensional part we are interested in. It can be simplified further
\begin{equation*}
\wtilde{N} = -e^2 \lbr \wtilde{\gamma}^{\alpha} \left( \bar{\fsl{p}}_1 + \bar{\fsl{q}} + m \right) \wtilde{\gamma}_{\alpha} + \gamma^{\alpha} \wtilde{\fsl{q}} \gamma_{\alpha} + \wtilde{\gamma}^{\alpha} \wtilde{\fsl{q}} \wtilde{\gamma}_{\alpha} \rbr = -e^2 \lbr - \underbrace{\wtilde{\gamma}^{\alpha} \wtilde{\gamma}_{\alpha}}_{= \epsilon} \left( \bar{\fsl{p}}_1 + \bar{\fsl{q}} - m \right) - \underbrace{\gamma^{\alpha} \gamma_{\alpha}}_{=4} \wtilde{\fsl{q}} + \wtilde{\gamma}^{\alpha} \wtilde{\fsl{q}} \wtilde{\gamma}_{\alpha} \rbr
\end{equation*}
where we have used again that the $\epsilon$-dimensional gamma matrices anti-commute with the 4-dimensional gamma matrices and equation \ref{eqn:2g} to simplify the expression. Plugging $\tilde{N}$ into the definition of $R_2$ we get
\begin{align*}
R_2^{\mathrm{ee}} & = \frac{1}{\left( 2\pi \right) ^4} \int d^d\bar{q} \frac{\wtilde{N}}{\bar{D}_1\bar{D}_0} = \frac{-e^2}{\left( 2\pi \right) ^4} \int d^d\bar{q} \frac{1}{\bar{D}_1\bar{D}_0} \left( -\epsilon \left( \fsl{p}_1 + \fsl{q} - m \right) + \underbrace{\wtilde{\fsl{q}} \left( \ldots \right)}_{=0} \right) = & \\ 
& = \frac{e^2}{\left( 2\pi \right) ^4} \lbr \underbrace{\int d^d\bar{q} \frac{\epsilon \left( \fsl{p}_1 - m \right)}{\bar{D}_1\bar{D}_0}}_{=-2\epsilon \frac{i \pi^2}{\epsilon}  \left( \fsl{p}_1 -m \right)}  + \underbrace{\int d^d\bar{q} \frac{\epsilon \fsl{q}}{\bar{D}_1\bar{D}_0}}_{=\epsilon \frac{i\pi^2}{\epsilon} \fsl{p}_1} \rbr = \frac{e^2}{\left( 2\pi \right) ^4} \epsilon \frac{i\pi^2}{\epsilon} \left( \left( -2 \right) \left( \fsl{p}_1 - m \right) + \fsl{p}_1 \right) = \frac{-ie^2}{16 \pi^2}  \left( \fsl{p}_1 - 2m \right) & \numberthis \label{eqn:R2ee}
\end{align*}
Where we have used that the integral over an odd function in q integrated over the whole space vanishes at the end of the first line. We also used the 2-point integrals \ref{eqn:2ptintsca} and  \ref{eqn:2ptintq}. \\
These are all the 2-point functions that are allowed by the Feynman rules of QED. So we continue with the 3-point functions now.

\subsection{3-point functions}
There are two possible 3-point functions, the 1PI contribution to the electron-photon vertex at 1-loop in QED and the 3-photon triangle diagram. Let us start with the electron-photon vertex which is given by
\begin{align*}
\begin{gathered}
\feynmandiagram [horizontal=i1 to v2] {
	i1 [particle=\mu] -- [photon, momentum=\(p_2-p_1\)] v2,
	f1 -- [fermion, momentum=\(p_1\)] v1
	   -- [fermion, momentum=\(p_1+q\)] v2
	   -- [fermion, momentum=\(p_2+q\)] v3
	   -- [fermion, momentum=\(p_2\)] f3,
	v3 -- [boson, momentum=\(q\)] v1, 
};
\end{gathered} \qquad
& =\int\frac{d^dq}{\left( 2\pi \right)^d} ie \gamma^{\beta} \frac{i \left( \fsl{p}_1+\fsl{q}+m \right)}{\left( p_1+q \right)^2 - m^2} ie \gamma^{\mu} \frac{i \left( \fsl{p}_2 + \fsl{q}+m \right)}{\left( p_2 + q \right)^2 - m^2} ie \gamma^{\alpha} \frac{-ig_{\alpha\beta}}{q^2} & \\
& \equiv \int\frac{d^dq}{\left( 2\pi \right)^d} \frac{\bar{N}}{\bar{D}_1\bar{D}_2\bar{D}_0} &
\end{align*}
We get for $\bar{N}$
\begin{align*}
\bar{N} (\bar{q}) & = e^3 \lbr \bar{\gamma}^{\beta} \left( \bar{\fsl{p}}_1 + \bar{\fsl{q}} + m \right) \bar{\gamma}^{\mu} \left( \bar{\fsl{p}}_2 + \bar{\fsl{q}} + m \right) \bar{\gamma}_{\beta} \rbr = e^3 \lbr \gamma^{\beta} \left( \fsl{p}_1 + \fsl{q} + m \right) \gamma^{\mu} \left( \fsl{p}_2 + \fsl{q} + m \right) \gamma_{\beta} + \right. & \\
& \left. + \wtilde{\gamma}^{\beta} \left( \fsl{p}_1 + \fsl{q} + m \right) \gamma^{\mu} \left( \fsl{p}_2 + \fsl{q} + m \right) \wtilde{\gamma}_{\beta} + \underbrace{\gamma^{\beta} \wtilde{\fsl{q}} \gamma^{\mu} \wtilde{\fsl{q}} \gamma_{\beta}}_{\equiv \ \circled{1}} + \underbrace{\wtilde{\gamma}^{\beta} \wtilde{\fsl{q}} \gamma^{\mu} \wtilde{\fsl{q}} \wtilde{\gamma}_{\beta}}_{\equiv \ \circled{2}}  \rbr \equiv N + \wtilde{N} &
\end{align*}
Where the last 3 terms define $\wtilde{N}$. Let us work on $\circled{1}$ and $\circled{2}$ separately. Using the fact that 4-dimensional and $\epsilon$-dimensional gamma matrices anticommute and equations \ref{eqn:3g} and \ref{eqn:fsldot} we get
\begin{equation*}
\circled{1} = \wtilde{q}_{\rho}\wtilde{q}_{\sigma} \gamma^{\beta} \wtilde{\gamma}^{\rho} \gamma^{\mu} \wtilde{\gamma}^{\sigma} \gamma_{\beta} = \wtilde{q}_{\rho}\wtilde{q}_{\sigma} \left( -1 \right)^3 \wtilde{\gamma}^{\rho} \wtilde{\gamma}^{\sigma} \gamma^{\beta} \gamma^{\mu} \gamma_{\beta} = 2\wtilde{\fsl{q}}\wtilde{\fsl{q}}\gamma^{\mu} = 2 \wtilde{q}^2 \gamma^{\mu}
\end{equation*}
And for the other term
\begin{equation*}
\circled{2} = \wtilde{q}_{\rho}\wtilde{q}_{\sigma} \wtilde{\gamma}^{\beta} \wtilde{\gamma}^{\rho} \gamma^{\mu} \wtilde{\gamma}^{\sigma} \wtilde{\gamma}_{\beta} = \wtilde{q}_{\rho}\wtilde{q}_{\sigma} \left( -1 \right)^2 \gamma^{\mu} \wtilde{\gamma}^{\beta} \wtilde{\gamma}^{\rho} \wtilde{\gamma}^{\sigma} \wtilde{\gamma}_{\beta} = \wtilde{q}_{\rho}\wtilde{q}_{\sigma} \gamma^{\mu} \left( \epsilon \wtilde{\gamma}^{\rho}\wtilde{\gamma}^{\sigma} + 2 \left[ \wtilde{\gamma}^{\rho}, \wtilde{\gamma}^{\sigma} \right] \right) = \epsilon \wtilde{q}^2 \gamma^{\mu}
\end{equation*}
where we have used that the 4- and $\epsilon$-dimensional gamma matrices anticommute and equation \ref{eqn:4g} for $d=\epsilon$. And in the last step we used equation \ref{eqn:fsldot} which also implies $\left[ \wtilde{\fsl{q}}, \wtilde{\fsl{q}} \right] = 0$.\\
Hence, after summing all of the terms we have
\begin{equation*}
\wtilde{N} = - e^3 \epsilon \left( \fsl{p}_1 + \fsl{q} - m \right) \gamma^{\mu} \left( \fsl{p}_2 + \fsl{q} - m \right) + \left( 2 + \epsilon \right) \wtilde{q}^2 \gamma^{\mu}
\end{equation*}
We can again plug this in the definition of $R_2$ and get
\begin{align*}
R_2^{\gamma\mathrm{ee}} = & \frac{1}{\left( 2\pi \right) ^4} \int d^d\bar{q} \frac{\wtilde{N}}{\bar{D}_0 \bar{D}_1 \bar{D}_2} = \frac{1}{\left( 2\pi \right) ^4} \int d^d\bar{q} \frac{e^3}{\bar{D}_0 \bar{D}_1 \bar{D}_2} \lbr -\epsilon \left( \fsl{p}_1 + \fsl{q} - m \right) \gamma^{\mu} \left( \fsl{p}_2 + \fsl{q} - m \right) + \left( 2 + \epsilon \right) \wtilde{q}^2 \gamma^{\mu} \rbr = & \\
& = \frac{e^3}{\left( 2\pi \right) ^4} \int d^d\bar{q} \frac{1}{\bar{D}_0 \bar{D}_1 \bar{D}_2} \lbr -\epsilon \fsl{q}\gamma^{\mu}\fsl{q} + \left( 2 + \epsilon \right) \wtilde{q}^2 \gamma^{\mu} \rbr = \frac{e^3}{\left( 2\pi \right) ^4} \lbr -\epsilon \gamma^{\alpha}\gamma^{\mu}\gamma^{\beta} \left( \frac{-i\pi^2}{2\epsilon} g_{\alpha\beta} \right) + \frac{-i\pi^2}{2} \left(2 + \epsilon \right) \gamma^{\mu} \rbr = & \\
& = \frac{e^3}{\left( 2\pi \right) ^4} \frac{i \pi^2}{2} \lbr \gamma^{\alpha}\gamma^{\mu}\gamma_{\alpha} - 2\gamma^{\mu} + O(\epsilon) \rbr = \frac{-ie^3}{8\pi^2} \gamma^{\mu} & \numberthis \label{eqn:R2photonee}
\end{align*}
Where we used the 3-point integrals \ref{eqn:3ptintq2} and \ref{eqn:3ptintq2vec} and in the last step equation \ref{eqn:3g}\\
As already mentioned, there is one more 3-point function at the 1-loop level which is permitted by the Feynman rules: the 3-point function with only photons as external particles. But it does not contribute to $R_2$ which we will show now. Because of the symmetry of the 3-point function there are 2 contributing diagrams
\begin{align*}
\begin{gathered}
\feynmandiagram [small, horizontal=i1 to v] {
	i1 [particle=\alpha] -- [photon, momentum=\(-p_1-p_2\)] v [crossed dot],
	i2 [particle=\beta] -- [photon, momentum=\(p_1\)] v,
	i3 [particle=\gamma] -- [photon, momentum=\(p_2\)] v,
};
\end{gathered}
\quad = \quad  
\begin{gathered}
\feynmandiagram [horizontal=i1 to v1] {
	i1 [particle=\alpha] -- [photon, momentum=\(-p_1-p_2\)] v1,
	i3 [particle=\gamma] -- [photon, momentum=\(p_2\)] v3,
	i2 [particle=\beta] -- [photon, momentum=\(p_1\)] v2,
	v3 -- [fermion, momentum'=\(q\)] v2
	   -- [fermion, momentum'=\(p_1+q\)] v1
	   -- [fermion, momentum'=\(q-p_2\)] v3,
};
\end{gathered}
\quad + \quad
\begin{gathered}
\feynmandiagram [horizontal=i1 to v1] {
	i1 [particle=\alpha] -- [photon, momentum=\(-p_1-p_2\)] v1,
	i3 [particle=\gamma] -- [photon, momentum=\(p_2\)] v3,
	i2 [particle=\beta] -- [photon, momentum=\(p_1\)] v2,
	v2 -- [fermion, momentum=\(q\)] v3
	   -- [fermion, momentum=\(p_2+q\)] v1
	   -- [fermion, momentum=\(q-p_1\)] v2,
};
\end{gathered}
\end{align*}

We only calculate the first diagram and then symmetrize the result with $p_1 \leftrightarrow p_2$, $\beta \leftrightarrow \gamma$. Evaluating the first diagram gives
\begin{align*}
\begin{gathered}
\feynmandiagram [horizontal=i1 to v1] {
	i1 [particle=\alpha] -- [photon, momentum=\(-p_1-p_2\)] v1,
	i3 [particle=\gamma] -- [photon, momentum=\(p_2\)] v3,
	i2 [particle=\beta] -- [photon, momentum=\(p_1\)] v2,
	v3 -- [fermion, momentum'=\(q\)] v2
	   -- [fermion, momentum'=\(p_1+q\)] v1
	   -- [fermion, momentum'=\(q-p_2\)] v3,
};
\end{gathered}
\qquad
& =\int\frac{d^dq}{\left( 2\pi \right)^d} \Tr \lbr ie\gamma^{\beta} \frac{i \left( \fsl{q} + m \right)}{q^2 - m^2} ie \gamma^{\gamma} \frac{i \left( \fsl{q} - \fsl{p}_2 + m \right)}{\left( q- p_2 \right) -m^2} ie \gamma^{\alpha} \frac{i \left( \fsl{q} + \fsl{p}_1 + m \right)}{\left( q + p_1 \right) -m^2} \rbr =  & \\
& = \int\frac{d^dq}{\left( 2\pi \right)^d} e^3 \Tr \lbr \gamma^{\beta} \frac{\left( \fsl{q} + m \right)}{q^2 - m^2} \gamma^{\gamma} \frac{\left( \fsl{q} - \fsl{p}_2 + m \right)}{\left( q- p_2 \right) -m^2} \gamma^{\alpha} \frac{\left( \fsl{q} + \fsl{p}_1 + m \right)}{\left( q + p_1 \right) -m^2} \rbr = & \\
& \equiv \int\frac{d^dq}{\left( 2\pi \right)^d} \frac{\bar{N}}{\bar{D}_1\bar{D}_{-2}\bar{D}_0} &
\end{align*}
From here we can again extract the $d$-dimensional numerator
\begin{align*}
\bar{N} (\bar{q}) & = e^3 \Tr \lbr \bar{\gamma}^{\beta} \left( \bar{\fsl{q}} + m \right) \bar{\gamma}^{\gamma} \left( \bar{\fsl{q}} - \bar{\fsl{p}}_2 + m \right) \bar{\gamma}^{\alpha} \left( \bar{\fsl{q}} + \bar{\fsl{p}}_1 + m \right) \rbr = & \\
& = e^3 \Tr \lbr \gamma^{\beta} \left( \fsl{q} + m \right) \gamma^{\gamma} \left( \fsl{q} - \fsl{p}_2 + m \right) \gamma^{\alpha} \left( \fsl{q} + \fsl{p}_1 + m \right) +  \gamma^{\beta} \left( \fsl{q} + \wtilde{\fsl{q}} + m \right) \gamma^{\gamma} \left( \fsl{q} + \wtilde{\fsl{q}} - \fsl{p}_2 + m \right) \gamma^{\alpha} \left( \fsl{q} + \wtilde{\fsl{q}} + \fsl{p}_1 + m \right) \rbr = & \\
& \equiv N + \wtilde{N} &
\end{align*}
The last term is $\wtilde{N}$ which can be further simplified as follows
\begin{align*}
\wtilde{N} & = e^3 \Tr \lbr \gamma^{\beta} \left( \fsl{q} + \wtilde{\fsl{q}} + m \right) \gamma^{\gamma} \left( \fsl{q} + \wtilde{\fsl{q}} - \fsl{p}_2 + m \right) \gamma^{\alpha} \left( \fsl{q} + \wtilde{\fsl{q}} + \fsl{p}_1 + m \right) \rbr = & \\
& = e^3 \Tr \lbr \gamma^{\beta} \fsl{q} \gamma^{\gamma} \wtilde{\fsl{q}} \gamma^{\alpha} \wtilde{\fsl{q}} + \gamma^{\beta} \wtilde{\fsl{q}} \gamma^{\gamma} \fsl{q} \gamma^{\alpha} \wtilde{\fsl{q}} + \gamma^{\beta} \wtilde{\fsl{q}} \gamma^{\gamma} \wtilde{\fsl{q}} \gamma^{\alpha} \fsl{q} + \gamma^{\beta} \wtilde{\fsl{q}} \gamma^{\gamma} \left( - \fsl{p}_2 \right) \gamma^{\alpha} \wtilde{\fsl{q}} + \gamma^{\beta} \wtilde{\fsl{q}} \gamma^{\gamma} \wtilde{\fsl{q}} \gamma^{\alpha} \fsl{p}_1 \rbr = & \\
& = - 4e^3 \wtilde{q}^2 \lbr q_{\mu} \left[ \left( g^{\beta\mu}g^{\gamma\alpha} - g^{\beta\gamma}g^{\mu\alpha} + g^{\beta\alpha}g^{\mu\gamma} \right) + \left( g^{\beta\gamma}g^{\alpha\mu} - g^{\beta\mu}g^{\gamma\alpha} + g^{\beta\alpha}g^{\mu\gamma} \right) + \left( g^{\beta\gamma}g^{\mu\alpha} - g^{\beta\alpha}g^{\mu\gamma} + g^{\beta\mu}g^{\alpha\gamma} \right) \right] + \right. & \\ 
& \left. + p_{1\mu} \left( g^{\beta\gamma}g^{\alpha\mu} - g^{\beta\alpha}g^{\gamma\mu} + g^{\beta\mu}g^{\alpha\gamma} \right) - p_{2\mu} \left( g^{\beta\gamma}g^{\mu\alpha} - g^{\beta\mu}g^{\alpha\gamma} + g^{\alpha\beta}g^{\gamma\mu} \right) \rbr = & \\
& = -4e^3 \wtilde{q}^2 \lbr q^{\beta}g^{\alpha\gamma} + q^{\gamma}g^{\alpha\beta} + q^{\alpha}g^{\beta\gamma} + p_1^{\alpha}g^{\beta\gamma} - p_1^{\gamma}g^{\alpha\beta} + p_1^{\beta}g^{\alpha\gamma} - p_2^{\alpha}g^{\beta\gamma} + p_2^{\beta}g^{\alpha\gamma} - p_2^{\gamma}g^{\alpha\beta} \rbr &
\end{align*}
From the second to the third line we have used that 4- and $\epsilon$-dimensional gamma matrices commute as well as equations \ref{eqn:Tr4g} and \ref{eqn:fsldot}. \\
This gives for the $R_2$ contribution of the first diagram
\begin{align*}
R_2^1 & = \frac{1}{\left( 2\pi \right)^4} \int d^d\bar{q} \frac{\wtilde{N}}{\bar{D}_1\bar{D}_{-2}\bar{D}_0} = & \\
& = \frac{-4e^3}{\left( 2\pi \right)^4} \int d^d \bar{q} \frac{1}{\bar{D}_1\bar{D}_{-2}\bar{D}_0} \lbr \wtilde{q}^2q^{\beta}g^{\alpha\gamma} + \wtilde{q}^2q^{\gamma}g^{\alpha\beta} + \wtilde{q}^2q^{\alpha}g^{\beta\gamma} + \wtilde{q}^2 \left[ \left( p_1 - p_2 \right)^{\alpha} g^{\beta\gamma} + \left( p_1 + p_2 \right)^{\beta} g^{\alpha\gamma} - \left( p_1 + p_2 \right)^{\gamma} g^{\alpha\beta} \right] \rbr = & \\
& = \frac{-4e^3}{\left( 2\pi \right)^4} \lbr \frac{i\pi^2}{6} \left[ \left( p_1-p_2 \right)^{\beta}g^{\alpha\gamma} + \left( p_1-p_2 \right)^{\gamma}g^{\alpha\beta} + \left( p_1-p_2 \right)^{\alpha}g^{\beta\gamma} \right] - \frac{i\pi^2}{2} \left[ \left( p_1-p_2 \right)^{\alpha} g^{\beta\gamma} + \left( p_1+p_2 \right)^{\beta} g^{\alpha\gamma} + \right. \right. & \\
& \left. \left. - \left( p_1+p_2 \right)^{\gamma} g^{\alpha\beta} \right] \rbr = & \\
& = \frac{-4e^3}{\left( 2\pi \right)^4} \lbr g^{\alpha\beta} \left[ \frac{i\pi^2}{6} \left( p_1-p_2 \right)^{\gamma} + \frac{i\pi^2}{2} \left( p_1+p_2 \right)^{\gamma} \right] + g^{\beta\gamma} \left[ \frac{i\pi^2}{6} \left( p_1-p_2 \right)^{\alpha} - \frac{i\pi^2}{2} \left( p_1-p_2 \right)^{\alpha} \right] + \right. & \\
& \left. + g^{\alpha\gamma} \left[ \frac{i\pi^2}{6} \left( p_1-p_2 \right)^{\beta} - \frac{i\pi^2}{2} \left( p_1+p_2 \right)^{\beta} \right] \rbr &
\end{align*}
Here we have used the 3-point integrals \ref{eqn:3ptintq2} and \ref{eqn:3ptintq3}.\\
To obtain the contribution of the second diagram we can simply exchange $p_1 \leftrightarrow p_2$, $\beta \leftrightarrow \gamma$ in the result of the first diagram. This yields
\begin{align*}
R_2^2 & = R_2^1(p_1 \leftrightarrow p_2, \ \beta \leftrightarrow \gamma) = & \\
& = \frac{-4e^3}{\left( 2\pi \right)^4} \lbr g^{\alpha\gamma} \left[ \frac{i\pi^2}{6} \left( p_2-p_1 \right)^{\beta} + \frac{i\pi^2}{2} \left( p_2+p_1 \right)^{\beta} \right] + g^{\beta\gamma} \left[ \frac{i\pi^2}{6} \left( p_2-p_1 \right)^{\alpha} - \frac{i\pi^2}{2} \left( p_2-p_1 \right)^{\alpha} \right] + \right. & \\
& \left. + g^{\alpha\beta} \left[ \frac{i\pi^2}{6} \left( p_2-p_1 \right)^{\gamma} - \frac{i\pi^2}{2} \left( p_2+p_1 \right)^{\gamma} \right] \rbr = & \\
& = \frac{-4e^3}{\left( 2\pi \right)^4} \lbr -g^{\alpha\beta} \left[ \frac{i\pi^2}{6} \left( p_1-p_2 \right)^{\gamma} + \frac{i\pi^2}{2} \left( p_1+p_2 \right)^{\gamma} \right] - g^{\beta\gamma} \left[ \frac{i\pi^2}{6} \left( p_1-p_2 \right)^{\alpha} - \frac{i\pi^2}{2} \left( p_1-p_2 \right)^{\alpha} \right] + \right. & \\
& \left. - g^{\alpha\gamma} \left[ \frac{i\pi^2}{6} \left( p_1-p_2 \right)^{\beta} - \frac{i\pi^2}{2} \left( p_1+p_2 \right)^{\beta} \right] \rbr = -R_2^1 &
\end{align*}
Now we add up both diagrams to get the full contribution of the photon triangle diagram. We get
\begin{equation}
R_2^{3\gamma} = R_2^1 + R_2^2 = R_2^1 - R_2^1 = 0
\end{equation}

\subsection{4-point function}
For the 4-point function we have to be more careful. The 1PI contribution at the 1-loop level consists of several diagrams. They are obtained by symmetrizing the external momenta of the diagram as follows
\begin{align*}
\begin{gathered}
\feynmandiagram [small, horizontal=i1 to i2] {
	i4 [particle=\delta] -- [photon, momentum=\(p_2\)] v [crossed dot],
	i2 [particle=\beta] -- [photon, momentum=\(p_3\)] v,
	i1 [particle=\alpha] -- [photon, momentum=\(p_1\)] v,
	i3 [particle=\gamma] -- [photon, momentum=\(p_4\)] v,
};
\end{gathered}
= 2 \times \lbr
\begin{gathered}
\feynmandiagram [horizontal=i4 to i3] {
	i2 [particle=\beta] -- [photon, momentum=\(p_3\)] v2,
	i4 [particle=\delta] -- [photon, momentum=\(p_2\)] v4,
	i1 [particle=\alpha] -- [photon, momentum=\(p_1\)] v1,
	i3 [particle=\gamma] -- [photon, momentum=\(p_4\)] v3,
	v4 -- [fermion, momentum=\(q\)] v1
	   -- [fermion, momentum=\(p_1+q\)] v2
	   -- [fermion, momentum=\(q+p_1+p_3\)] v3
	   -- [fermion, momentum=\(q-p_2\)] v4,
};
\end{gathered}
\ \ + \ \ \left( \alpha \leftrightarrow \beta; \ p_1 \leftrightarrow p_3 \right) \ + \ \left( \alpha \leftrightarrow \delta; \ p_1 \leftrightarrow p_2 \right) \rbr 
\end{align*}

We only calculate one of the diagrams and do the symmetrizing with the result of our calculation, so we only have to evaluate one diagram. The first of the three diagrams gives
\begin{align*}
\begin{gathered}
\feynmandiagram [horizontal=i4 to i3] {
	i2 [particle=\beta] -- [photon, momentum=\(p_3\)] v2,
	i4 [particle=\delta] -- [photon, momentum=\(p_2\)] v4,
	i1 [particle=\alpha] -- [photon, momentum=\(p_1\)] v1,
	i3 [particle=\gamma] -- [photon, momentum=\(p_4\)] v3,
	v4 -- [fermion, momentum=\(q\)] v1
	   -- [fermion, momentum=\(p_1+q\)] v2
	   -- [fermion, momentum=\(q+p_1+p_3\)] v3
	   -- [fermion, momentum=\(q-p_2\)] v4,
};
\end{gathered}\qquad
& = \int\frac{d^dq}{\left( 2\pi \right)^d} \left( -1 \right) \mathrm{Tr} \lbr ie \gamma^{\alpha} \frac{i \left( \fsl{p}_1 + \fsl{q} + m \right)}{\left( p_1 +q \right) ^2 -m^2} ie \gamma^{\beta} \frac{i \left( \fsl{q} + \fsl{p}_3 + \fsl{p}_1 + m \right)}{\left( p_3 + p_1 +q \right) ^2 -m^2} \times \right. & \\
& \left. \times ie \gamma^{\gamma} \frac{i \left( \fsl{q} - \fsl{p}_2 + m \right)}{\left( q - p_2 \right) ^2 - m^2} ie \gamma^{\delta} \frac{i \left( \fsl{q} + m \right)}{q ^2 - m^2} \rbr \equiv \int\frac{d^dq}{\left( 2\pi \right)^d} \frac{\bar{N}}{\bar{D}_1\bar{D}_{13}\bar{D}_{-2}\bar{D}_0} &
\end{align*}
where $D_{13} = \left( p_3 + p_1 +q \right) ^2 -m^2$. From this we get for the $d$-dimensional numerator the following
\begin{align*}
\bar{N} (\bar{q}) & = -e^4 \mathrm{Tr} \lbr \bar{\gamma}^{\alpha} \left( \bar{\fsl{p}}_1 + \bar{\fsl{q}} + m \right) \bar{\gamma}^{\beta} \left( \bar{\fsl{q}} +  \bar{\fsl{p}}_1 + \bar{\fsl{p}}_3 + m \right) \bar{\gamma}^{\gamma} \left( \bar{\fsl{q}} - \bar{\fsl{p}}_2 + m \right) \bar{\gamma}^{\delta} \left( \bar{\fsl{q}} + m \right) \rbr = & \\
& = -e^4 \mathrm{Tr} \lbr \gamma^{\alpha} \left( \fsl{p}_1 + \fsl{q} + m \right) \gamma^{\beta} \left( \fsl{q} +  \fsl{p}_1 + \fsl{p}_3 + m \right) \gamma^{\gamma} \left( \fsl{q} - \fsl{p}_2 + m \right) \gamma^{\delta} \left( \fsl{q} + m \right) + \right. &  \\
&  \left. + \gamma^{\alpha} \wtilde{\fsl{q}} \gamma^{\beta} \wtilde{\fsl{q}} \gamma^{\gamma} \wtilde{\fsl{q}} \gamma^{\delta} \wtilde{\fsl{q}} + \gamma^{\alpha} \wtilde{\fsl{q}} \gamma^{\beta} \wtilde{\fsl{q}} \gamma^{\gamma} \fsl{q} \gamma^{\delta} \fsl{q}  + \gamma^{\alpha} \fsl{q} \gamma^{\beta} \wtilde{\fsl{q}} \gamma^{\gamma} \wtilde{\fsl{q}} \gamma^{\delta} \fsl{q} + \gamma^{\alpha} \fsl{q} \gamma^{\beta} \fsl{q} \gamma^{\gamma} \wtilde{\fsl{q}} \gamma^{\delta} \wtilde{\fsl{q}} + \gamma^{\alpha} \wtilde{\fsl{q}} \gamma^{\beta} \fsl{q} \gamma^{\gamma} \fsl{q} \gamma^{\delta} \wtilde{\fsl{q}} + \right. & \\
& \left. + \gamma^{\alpha} \wtilde{\fsl{q}} \gamma^{\beta} \fsl{q} \gamma^{\gamma} \wtilde{\fsl{q}} \gamma^{\delta} \fsl{q} + \gamma^{\alpha} \fsl{q} \gamma^{\beta} \wtilde{\fsl{q}} \gamma^{\gamma} \fsl{q} \gamma^{\delta} \wtilde{\fsl{q}} \rbr \equiv N + \wtilde{N} &
\end{align*}
Where all of the terms besides the first one define $\wtilde{N}$. Furthermore, we have used that the trace of an odd number of Dirac matrices is zero. Using the fact that 4- and $\epsilon$-dimensional gamma matrices commute as well as equation \ref{eqn:fsldot}, $\wtilde{N}$ can be further simplified to
\begin{align*}
\wtilde{N} = & -e^4 \mathrm{Tr} \lbr \left( -1 \right)^{10} \wtilde{q}^4 \gamma^{\alpha} \gamma^{\beta} \gamma^{\gamma} \gamma^{\delta} + \wtilde{q}^2 \left[ \left( -1 \right) ^3  \gamma^{\alpha} \gamma^{\beta} \gamma^{\gamma} \fsl{q} \gamma^{\delta} \fsl{q}  + \left( -1 \right)^7 \gamma^{\alpha} \fsl{q} \gamma^{\beta} \gamma^{\gamma} \gamma^{\delta} \fsl{q} + \left( -1 \right)^{11} \gamma^{\alpha} \fsl{q} \gamma^{\beta} \fsl{q} \gamma^{\gamma} \gamma^{\delta} + \right. \right. & \\
& \left. \left. + \left( -1 \right)^7 \gamma^{\alpha} \gamma^{\beta} \fsl{q} \gamma^{\gamma} \fsl{q} \gamma^{\delta} + \left( -1 \right)^5 \gamma^{\alpha} \gamma^{\beta} \fsl{q} \gamma^{\gamma} \gamma^{\delta} \fsl{q} + \left( -1 \right)^9 \gamma^{\alpha} \fsl{q} \gamma^{\beta} \gamma^{\gamma} \fsl{q} \gamma^{\delta} \right] \rbr = & \\
& = -e^4 \mathrm{Tr} \lbr \wtilde{q}^4 \gamma^{\alpha} \gamma^{\beta} \gamma^{\gamma} \gamma^{\delta} - \wtilde{q}^2 \left( \gamma^{\alpha} \gamma^{\beta} \gamma^{\gamma} \fsl{q} \gamma^{\delta} \fsl{q} + \gamma^{\alpha} \fsl{q} \gamma^{\beta} \gamma^{\gamma} \gamma^{\delta} \fsl{q} + \gamma^{\alpha} \fsl{q} \gamma^{\beta} \fsl{q} \gamma^{\gamma} \gamma^{\delta} + \gamma^{\alpha} \gamma^{\beta} \fsl{q} \gamma^{\gamma} \fsl{q} \gamma^{\delta} + \right. \right. & \\
& \left. \left. + \gamma^{\alpha} \gamma^{\beta} \fsl{q} \gamma^{\gamma} \gamma^{\delta} \fsl{q} + \gamma^{\alpha} \fsl{q} \gamma^{\beta} \gamma^{\gamma} \fsl{q} \gamma^{\delta} \right) \rbr &
\end{align*}
Since this expression involves the trace over up to 6 Dirac matrices, the calculation is very cumbersome. We can evaluate this expression with the help of the Mathematica package FeynCalc \cite{FeynCalc,FeynCalc2} 
\begin{figure}[h!]
  \begin{center}
    \includegraphics[width=1.05\textwidth]{Figures/Trace_4ptfct_Mathematica}
  \end{center}
  \setlength{\belowcaptionskip}{-20pt}
  \caption*{}
\end{figure} \\
As usual we plug this in the definition of $R_2$ and evaluate the integrals to get the expression of $R_2$ for the first of the contributing diagrams.
\begin{align*}
R_2 = & \frac{1}{\left( 2\pi \right) ^4} \int d^d\bar{q} \frac{\wtilde{N}}{\bar{D}_1 \bar{D}_{13} \bar{D}_{-2} \bar{D}_0} = \frac{1}{\left( 2\pi \right) ^4} \int d^d\bar{q} \frac{4e^4}{\bar{D}_1 \bar{D}_{13} \bar{D}_2 \bar{0}} \wtilde{q}^2 \lbr \left( 2q^2 + \wtilde{q}^2 \right) \left( g^{\alpha\delta}g^{\beta\gamma} - g^{\alpha\gamma}g^{\beta\delta} + g^{\alpha\beta}g^{\gamma\delta} \right) + \right. & \\
& \left. - 2 \left( g^{\alpha\beta}q^{\gamma}q^{\delta} + g^{\gamma\delta}q^{\alpha}q^{\beta} + g^{\alpha\delta}q^{\beta}q^{\gamma} + g^{\beta\gamma}q^{\alpha}q^{\delta} \right) \rbr = & \\
& = \frac{-4e^4}{\left( 2\pi \right) ^4} \lbr \left( 2 \left( \frac{-i\pi^2}{3} \right) + \left( \frac{-i\pi^2}{6} \right) \right) \left( g^{\alpha\delta}g^{\beta\gamma} - g^{\alpha\gamma}g^{\beta\delta} + g^{\alpha\beta}g^{\gamma\delta} \right) - 2 \left( \frac{-i\pi^2}{12} \right) \left( g^{\alpha\beta}g^{\gamma\delta} + g^{\gamma\delta}g^{\alpha\beta} + \right. \right. & \\
& \left. \left. + g^{\alpha\delta}g^{\beta\gamma} + g^{\beta\gamma}g^{\alpha\delta} \right) \rbr = \frac{ie^4}{4\pi^2} \lbr \frac{5}{6} \left( g^{\alpha\delta}g^{\beta\gamma} - g^{\alpha\gamma}g^{\beta\delta} + g^{\alpha\beta}g^{\gamma\delta} \right) - \frac{1}{6} \left( 2 g^{\alpha\beta}g^{\gamma\delta} + 2g^{\alpha\delta}g^{\beta\gamma} +  \right) \rbr = & \\
& = \frac{ie^4}{24 \pi^2} \left( 3 g^{\alpha\beta}g^{\gamma\delta} - 5 g^{\alpha\gamma}g^{\beta\delta} + 3 g^{\beta\gamma}g^{\alpha\delta} \right) &
\end{align*}
Where we have used the 4-point integrals \ref{eqn:4ptintq4}, \ref{eqn:4ptintvec} and \ref{eqn:4ptintq2q2}. This is independent of momenta, so we only have to symmetrize the indices to get the full 4-photon $R_2$. 
\begin{align*}
R_2^{4\gamma} = & 2 \left[ R_2 + R_2 \left( \alpha \leftrightarrow \delta \right) + R_2 \left( \alpha \leftrightarrow \beta \right) \right] = \frac{2ie^4}{24 \pi^2} \lbr \left( 3 g^{\alpha\beta}g^{\gamma\delta} - 5 g^{\alpha\gamma}g^{\beta\delta} + 3 g^{\beta\gamma}g^{\alpha\delta} \right) + \left( 3 g^{\beta\delta}g^{\alpha\gamma} - 5 g^{\gamma\delta}g^{\alpha\beta} + 3 g^{\beta\gamma}g^{\alpha\delta} \right) + \right. & \\
& \left. + \left( 3 g^{\alpha\beta}g^{\gamma\delta} - 5 g^{\beta\gamma}g^{\alpha\delta} + 3 g^{\alpha\gamma}g^{\beta\delta} \right) \rbr = \frac{ie^4}{12 \pi^2} \left( g^{\alpha\beta}g^{\gamma\delta} + g^{\alpha\gamma}g^{\beta\delta} + g^{\beta\gamma}g^{\alpha\delta} \right) & \numberthis \label{eqn:R24photon}
\end{align*}
Like for the 3-point functions all of the other 4-point functions which are permitted by the Feynman rules vanish. We will not show this here because the calculations for the 4-point functions are quite lengthy. \\
We have derived the complete set of $R_2$ in pure QED. Now we can go to the more complex Standard Model to see how QED contributes to the rational terms in the full Standard Model.