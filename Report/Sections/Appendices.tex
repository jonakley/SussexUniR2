\appendix
{\bf\Large Appendices} \\
\section{Important Integrals}
\label{app:Integrals}
In the calculation of $R_2$ the same integrals appear over and over. The most important ones are \cite{R2QCD} \\

{\bf 2-point integrals} \\
\begin{align}
\int  d^d \bar{q} \frac{\tilde{q}^2}{\bar{D}_i\bar{D}_j} & =  -\frac{i\pi^2}{2} \left[ m_i^2 + m_j^2 - \frac{\left( p_i - p_j \right)^2}{3} \right] + O(\epsilon) & \\
\mathrm{P.P.} \left( \int  d^d \bar{q} \frac{1}{\bar{D}_i\bar{D}_j} \right) & = -2\frac{i\pi^2}{\epsilon} & \\
\mathrm{P.P.} \left( \int  d^d \bar{q} \frac{q_{\mu}}{\bar{D}_i\bar{D}_j} \right) & =  \frac{i\pi^2}{\epsilon} \left( p_i + p_j \right)_{\mu} & 
\end{align}
{\bf 3-point integrals} \\
\begin{align}
\int  d^d \bar{q} \frac{\tilde{q}^2}{\bar{D}_i\bar{D}_j\bar{D}_k} & =  -\frac{i\pi^2}{2} + O(\epsilon) & \\
\int  d^d \bar{q} \frac{1}{\bar{D}_i\bar{D}_j} & = \frac{i\pi^2}{6} \left( p_i + p_j + p_k \right)_{\mu} + O(\epsilon) & \\
\mathrm{P.P.} \left( \int  d^d \bar{q} \frac{q_{\mu}q_{\nu}}{\bar{D}_i\bar{D}_j\bar{D}_k} \right) & =  -\frac{i\pi^2}{2\epsilon} g_{\mu\nu} & 
\end{align}

{\bf 4-point integrals} \\
\begin{align}
\int  d^d \bar{q} \frac{\tilde{q}^4}{\bar{D}_i\bar{D}_j\bar{D}_k\bar{D}_l} & =  -\frac{i\pi^2}{6} + O(\epsilon) & \\
\int  d^d \bar{q} \frac{\tilde{q}^2q_{\mu}q_{\nu}}{\bar{D}_i\bar{D}_j\bar{D}_k\bar{D}_l} & =  -\frac{i\pi^2}{12} g_{\mu\nu} + O(\epsilon) & \\
\int  d^d \bar{q} \frac{\tilde{q}^2 q^2}{\bar{D}_i\bar{D}_j\bar{D}_k\bar{D}_l} & =  -\frac{i\pi^2}{3} + O(\epsilon) &  
\end{align}

\section{Traceology}
\label{app:Traceology}
The following dentities for Dirac matrices are very useful when calculating Feynman diagrams in QED
\begin{itemize}
\item[1.] $\mathrm{Tr} \left( \gamma^{\alpha} \gamma^{\beta} \right) = dg^{\alpha\beta}$
\item[2.] $\mathrm{Tr} \left( \gamma^{\alpha}\gamma^{\beta}\gamma^{\gamma}\gamma^{\delta} \right) = d \left( g^{\alpha\beta}g^{\gamma\delta} - g^{\alpha\gamma}g^{\beta\delta} + g^{\alpha\delta}g^{\beta\gamma} \right)$
\item[3.] $\gamma^{\alpha}\gamma_{\alpha} = d$
\item[4.] $\gamma^{\alpha}\gamma^{\beta}\gamma_{\alpha} = \left( 2-d \right) \gamma^{\beta}$
\item[$\ldots$] $\ldots$
\item[n.]$\fsl{a}\fsl{b} = a \cdot b$
\end{itemize}


Dirac matrices obey Clifford algebra $\lbr \gamma^{\mu}, \gamma^{\nu} \rbr = 2 g^{\mu\nu} \mathds{1}_d$ with $g^{\mu\nu}$ the Minkowski metric in $d$ dimensions
\begin{equation*}
 g^{\mu\nu} = 
\begin{cases}
1 \quad\qquad\qquad \text{for } \mu = \nu = 0 \\
-1 \ \qquad\qquad \text{for } \mu = \nu = 1,2,\ldots,d-1 \\
0 \quad\qquad\qquad \text{for } \mu \neq \nu
\end{cases}
\end{equation*}



{\bf Proofs for identities}

1. $\mathrm{Tr} \left( \gamma^{\alpha} \gamma^{\beta} \right) = dg^{\alpha\beta}$
\begin{proof}
\begin{equation*}
\mathrm{Tr} \left( \gamma^{\alpha} \gamma^{\beta} \right) = \mathrm{Tr} \left( 2 g^{\alpha\beta} - \gamma^{\beta}\gamma^{\alpha} \right) = 2g^{\alpha\beta} \mathrm{Tr} \left( \mathds{1}_d \right) - \mathrm{Tr} \left( \gamma^{\beta} \gamma^{\alpha} \right) = 2dg^{\alpha\beta} - \mathrm{Tr} \left( \gamma^{\alpha} \gamma^{\beta} \right)
\end{equation*}
\begin{equation*}
\Rightarrow \mathrm{Tr} \left( \gamma^{\alpha} \gamma^{\beta} \right) = dg^{\alpha\beta}
\end{equation*}
\end{proof}

2. $\mathrm{Tr} \left( \gamma^{\alpha}\gamma^{\beta}\gamma^{\gamma}\gamma^{\delta} \right) = d \left( g^{\alpha\beta}g^{\gamma\delta} - g^{\alpha\gamma}g^{\beta\delta} + g^{\alpha\delta}g^{\beta\gamma} \right)$
\begin{proof}
\begin{align*}
\mathrm{Tr} \left( \gamma^{\alpha}\gamma^{\beta}\gamma^{\gamma}\gamma^{\delta} \right) & = \mathrm{Tr} \left( \left( 2g^{\alpha\beta} - \gamma^{\beta}\gamma^{\alpha} \right) \gamma^{\gamma} \gamma^{\delta} \right) = 2g^{\alpha\beta} \mathrm{Tr} \left( \gamma^{\gamma}\gamma^{\delta}\right) - \mathrm{Tr} \left( \gamma^{\beta} \left( 2g^{\alpha\gamma} - \gamma^{\gamma} \gamma^{\alpha} \right) \gamma^{\delta} \right) = & \\
& = 2d g^{\alpha\beta} g^{\gamma\delta} - 2g^{\alpha\gamma} \mathrm{Tr} \left( \gamma^{\beta}\gamma^{\delta} \right) + \mathrm{Tr} \left( \gamma^{\beta} \gamma^{\gamma} \left( 2g^{\alpha\delta} - \gamma^{\delta}\gamma^{\alpha} \right) \right) = & \\
& = 2d \left( g^{\alpha\beta} g^{\gamma\delta} - g^{\alpha\gamma} g^{\beta\delta} \right) + 2g^{\alpha\delta} \mathrm{Tr} \left( \gamma^{\beta}\gamma^{\gamma} \right) - \mathrm{Tr} \left( \gamma^{\beta}\gamma^{\gamma}\gamma^{\delta}\gamma^{\alpha} \right) = & \\
& = 2d \left( g^{\alpha\beta} g^{\gamma\delta} - g^{\alpha\gamma}g^{\beta\delta} + g^{\alpha\delta} g^{\beta\gamma} \right) - \mathrm{Tr} \left( \gamma^{\alpha}\gamma^{\beta}\gamma^{\gamma}\gamma^{\delta} \right) & 
\end{align*}
\begin{equation*}
\Rightarrow \mathrm{Tr} \left( \gamma^{\alpha}\gamma^{\beta}\gamma^{\gamma}\gamma^{\delta} \right) = d \left( g^{\alpha\beta}g^{\gamma\delta} - g^{\alpha\gamma}g^{\beta\delta} + g^{\alpha\delta}g^{\beta\gamma} \right)
\end{equation*}
\end{proof}


3. $\gamma^{\alpha}\gamma_{\alpha} = d$
\begin{proof}
\begin{equation*}
\gamma^{\alpha}\gamma_{\alpha} = \frac{1}{2} \left( \gamma^{\alpha}\gamma_{\alpha} + \gamma_{\alpha}\gamma^{\alpha} \right) = \frac{1}{2} \lbr \gamma^{\alpha}, \gamma_{\alpha} \rbr = \frac{1}{2} 2 g^{\alpha}_{ \ \alpha} = d
\end{equation*}
\end{proof}

4. $\gamma^{\alpha}\gamma^{\beta}\gamma_{\alpha} = \left( 2-d \right) \gamma^{\beta}$
\begin{proof}
\begin{equation*}
\gamma^{\alpha}\gamma^{\beta}\gamma_{\alpha} = \left( 2g^{\alpha\beta} - \gamma^{\beta}\gamma^{\alpha} \right) \gamma_{\alpha} = \left( 2-d \right) \gamma^{\beta}
\end{equation*}
\end{proof}

n. $\fsl{a}\fsl{b} = a \cdot b$
\begin{proof}
\begin{equation*}
\fsl{a} \fsl{b} = a_{\alpha} b_{\beta} \gamma^{\alpha}\gamma^{\beta} = a_{\alpha} b_{\beta} \left( 2 g^{\alpha\beta} - \gamma^{\beta}\gamma^{\alpha} \right) = 2 a \cdot b - \fsl{a}\fsl{b}
\end{equation*}
\begin{equation*}
\Rightarrow \fsl{a}\fsl{b} = a \cdot b
\end{equation*}
\end{proof}


\section{Relation Between Left- \& Right-handed Currents and Axial \& Vector Currents}
\label{app:Currents}
A classical Lagrangian permits symmetries which can be implemented by Lie groups $G$. An element $g \in G$ of a Lie group can be parametrized as $g = \mathrm{exp} \left(i \alpha^a T^a \right)$ where $\alpha^a$ are real parameters and $T^a$ the generators of the Lie group. Noether's theorem predicts a clasically conserved current for each generator of a continuous symmetry. For a field $\phi$ with trafo $\delta\phi = \phi' - \phi = g \phi - \phi \approx \left( 1 + i \alpha^a T^a \right) \phi - \phi = i \alpha^a T^a \phi$ the conserved current can be shown to be
\begin{equation*}
j^{\mu a} = \frac{\partial \mathcal{L}}{\partial \left( \partial_{\mu} \phi \right)} \frac{\partial \delta\phi}{\partial \alpha_a}
\end{equation*} 
The left- and right-handed part of the currents $j^{\mu a}_{L/R}$ are
\begin{equation*}
j^{\mu a}_L = \bar{\psi}_L \gamma^{\mu} T^a \psi_L = \left( P_L \psi \right)^{\dagger} \gamma^0 \gamma^{\mu} T^a P_L \psi \overset{P_L^{\dagger}=P_L}{=} \psi^{\dagger} P_L \gamma^0 \gamma^{\mu} T^a P_L \psi = \bar{\psi} \gamma^{\mu} T^a P_L^2 \psi \overset{P_L^2 = P_L}{=} \bar{\psi} \gamma^{\mu} T^a P_L \psi
\end{equation*}
\begin{equation*}
j^{\mu a}_R = \bar{\psi}_R \gamma^{\mu} T^a \psi_R = \bar{\psi} \gamma^{\mu} T^a P_R \psi
\end{equation*}
where $P_{L/R} = \frac{1}{2} \left( 1 \mp \gamma_5 \right)$ is the left-/right-handed projector. \\
From the left- and right-handed currents we can define axial-vector and vector currents
\begin{equation*}
j^{\mu a} = j^{\mu a}_R + j^{\mu a}_L = \bar{\psi} \gamma^{\mu} T^a \left( P_R + P_L \right) = \bar{\psi} \gamma^{\mu} T^a \psi
\end{equation*}
\begin{equation*}
j^{\mu a}_5 = j^{\mu a}_R - j^{\mu a}_L = \bar{\psi} \gamma^{\mu} T^a \left( P_R - P_L \right) \psi = \bar{\psi} \gamma^{\mu} T^a \gamma_5 \psi
\end{equation*}
Now we can couple the currents to vector fields to obtain interactions. E.g., the vector coupling in QED is given by the Lagrangian
\begin{equation*}
\mathcal{L}_{coupl}^{QED} = e A_ {\mu} j^{\mu} = e A_{\mu} \bar{\psi} \gamma^{\mu} Q_e \psi = -e A_{\mu} \bar{\psi} \gamma^{\mu} \psi 
\end{equation*}
In general, we can couple any linear combination of currents to a vector field as long as the combination is Lorentz and gauge invariant. E.g., the neutral current in the electroweak theory is a superposition of a vector and an axialvector current
\begin{equation*}
\mathcal{L}_{coupl}^{NC} = g Z_{\mu} \left( g_v j^{\mu} - g_A j^{\mu}_5 \right)
\end{equation*}
We can use the above relations to express this coupling in terms of right- and left-handed currents
\begin{align*}
\mathcal{L}_{coupl}^{NC} & = g Z_{\mu} \left( g_V \bar{\psi} \gamma^{\mu} \psi - g_A \bar{\psi} \gamma^{\mu} \gamma_5 \psi \right) = & \\
& = g Z_{\mu} \left( g_V \bar{\psi} \gamma^{\mu} \psi + \frac{g_A}{2} \bar{\psi} \gamma^{\mu} \psi - \frac{g_A}{2} \bar{\psi} \gamma^{\mu} \psi - g_A \bar{\psi} \gamma^{\mu} \gamma_5 \psi + \frac{g_V}{2} \bar{\psi} \gamma^{\mu} \gamma_5 \psi - \frac{g_V}{2} \bar{\psi} \gamma^{\mu} \gamma_5 \psi \right) = & \\
& = g Z_{\mu} \left( \left( g_V + g_A \right) \bar{\psi} \gamma^{\mu} \frac{1}{2} \left( 1 - \gamma_5 \right) \psi + \left( g_V - g_A \right) \bar{\psi} \gamma^{\mu} \frac{1}{2} \left( 1 + \gamma_5 \right) \psi \right) = & \\
& = g Z_{\mu} \left( \left( g_V + g_A \right) \bar{\psi} \gamma^{\mu} P_L \psi + \left( g_V - g_A \right) \bar{\psi} \gamma^{\mu} P_R \psi \right) \equiv g Z_{\mu} \left( g_L j^{\mu}_L + g_R j^{\mu}_R \right)  &
\end{align*}
This gives the following relation between the (axial-)vector and the left-/right-handed couplings
\begin{align*}
&g_L = g_V + g_A &\\
&g_R = g_V - g_A &
\end{align*}

