\appendix
{\bf\Large Appendices} \\
\section{Feynman Rules}
\label{sec:FeynmanRules}
In this appendix all of the Feynman rules which were used for the calculations are listed. The Feynman rules are given for the whole Standard Model, but the pure QED Feynman rules can be obtained by taking $Q_f \rightarrow Q_e = -1$ and $m_f \rightarrow m_e\equiv m$. \\

{\bf Propagator}
\begin{equation*}
\begin{gathered}
\feynmandiagram [horizontal=a to b] {
	a -- [fermion,momentum=\(p\)] b ,
};
\end{gathered}
= \frac{i\left( \fsl{p} + m_f \right)}{p^2 - m_f^2+i\epsilon} \qquad\qquad\qquad
\begin{gathered}
\feynmandiagram [horizontal=a to b] {
	a [particle=\alpha] -- [photon,momentum=\(p\)] b [particle=\beta],
};
\end{gathered}
= \frac{-ig^{\alpha\beta}}{p^2+i\epsilon} \qquad\text{in 't Hooft-Feynman gauge}
\end{equation*}\\

{\bf Interactions}\\
\begin{equation*}
\begin{gathered}
\feynmandiagram [horizontal=a to b] {
	a -- [scalar] b ,
	c -- [fermion] b -- [fermion] d,
};
\end{gathered}
= \frac{-ig}{2}\frac{m_f}{m_W} \qquad\qquad\qquad\qquad\qquad
\begin{gathered}
\feynmandiagram [horizontal=a to b] {
	a [particle={\gamma,\alpha}] -- [photon] b ,
	c -- [fermion] b -- [fermion] d,
};
\end{gathered}
= -ieQ_f\gamma^{\alpha}
\end{equation*}
\begin{equation*}
\begin{gathered}
\feynmandiagram [horizontal=a to b] {
	a [particle={Z,\alpha}] -- [photon] b ,
	c -- [fermion] b -- [fermion] d,
};
\end{gathered}
= \frac{ig}{\cos\theta_W}\gamma^{\alpha} \left(g_V-g_A \gamma_5\right) \qquad\qquad\qquad
\begin{gathered}
\feynmandiagram [horizontal=a to b] {
	a [particle={a,\alpha}] -- [gluon] b ,
	c [particle=i] -- [fermion] b -- [fermion] d [particle=j],
};
\end{gathered}
= ig_S T^a_{ij}\gamma^{\alpha}
\end{equation*}


\section{Important Integrals}
\label{app:Integrals}
In the calculation of $R_2$ we have to evaluate 2-,3- and 4-point functions. They can be reduced to a set of integrals which are knwon in a general form. The integrals we need are \cite{R2QCD} \\

{\bf 2-point integrals}
\begin{equation}
\label{eqn:2ptintq2} 
\int  d^d \bar{q} \frac{\tilde{q}^2}{\bar{D}_i\bar{D}_j} =  -\frac{i\pi^2}{2} \left[ m_i^2 + m_j^2 - \frac{\left( p_i - p_j \right)^2}{3} \right] + O(\epsilon)
\end{equation}
\begin{equation}
\label{eqn:2ptintsca} 
\mathrm{P.P.} \left( \int  d^d \bar{q} \frac{1}{\bar{D}_i\bar{D}_j} \right) = -2\frac{i\pi^2}{\epsilon}
\end{equation}
\begin{equation}
\label{eqn:2ptintq} 
\mathrm{P.P.} \left( \int  d^d \bar{q} \frac{q_{\mu}}{\bar{D}_i\bar{D}_j} \right) =  \frac{i\pi^2}{\epsilon} \left( p_i + p_j \right)_{\mu}
\end{equation}\\

{\bf 3-point integrals} 
\begin{equation}
\label{eqn:3ptintq2}
\int  d^d \bar{q} \frac{\tilde{q}^2}{\bar{D}_i\bar{D}_j\bar{D}_k} = -\frac{i\pi^2}{2} + O(\epsilon)
\end{equation}
\begin{equation}
\label{eqn:3ptintq3}
\int  d^d \bar{q} \frac{\tilde{q}^2q_{\mu}}{\bar{D}_i\bar{D}_j\bar{D}_k} = \frac{i\pi^2}{6} \left( p_i + p_j + p_k \right)_{\mu} + O(\epsilon)
\end{equation}
\begin{equation}
\label{eqn:3ptintq2vec}
\mathrm{P.P.} \left( \int  d^d \bar{q} \frac{q_{\mu}q_{\nu}}{\bar{D}_i\bar{D}_j\bar{D}_k} \right) =  -\frac{i\pi^2}{2\epsilon} g_{\mu\nu}
\end{equation}\\

{\bf 4-point integrals} 
\begin{equation}
\label{eqn:4ptintq4}
\int  d^d \bar{q} \frac{\tilde{q}^4}{\bar{D}_i\bar{D}_j\bar{D}_k\bar{D}_l} =  -\frac{i\pi^2}{6} + O(\epsilon)
\end{equation}
\begin{equation}
\label{eqn:4ptintvec}
\int  d^d \bar{q} \frac{\tilde{q}^2q_{\mu}q_{\nu}}{\bar{D}_i\bar{D}_j\bar{D}_k\bar{D}_l} =  -\frac{i\pi^2}{12} g_{\mu\nu} + O(\epsilon)
\end{equation}
\begin{equation}
\label{eqn:4ptintq2q2}
\int  d^d \bar{q} \frac{\tilde{q}^2 q^2}{\bar{D}_i\bar{D}_j\bar{D}_k\bar{D}_l} =  -\frac{i\pi^2}{3} + O(\epsilon) 
\end{equation}

\section{Identities for Gamma Matrices}
\label{app:Traceology}
In a theory with fermions the Dirac matrices appear as the generators of the spinor representation of the Poincaré algebra. The following identities for Dirac matrices are very useful when evaluating Feynman diagrams
\begin{itemize}
\item[1.] $\Tr \left( \gamma^{\alpha} \gamma^{\beta} \right) = dg^{\alpha\beta}$ \numberitem\label{eqn:Tr2g}

\item[2.] $\Tr \left( \gamma^{\alpha}\gamma^{\beta}\gamma^{\gamma}\gamma^{\delta} \right) = d \left( g^{\alpha\beta}g^{\gamma\delta} - g^{\alpha\gamma}g^{\beta\delta} + g^{\alpha\delta}g^{\beta\gamma} \right)$ \numberitem\label{eqn:Tr4g}

\item[3.] $\Tr \left( \gamma^{\alpha_1}\gamma^{\alpha_2} \cdots \gamma^{\alpha_n} \right) = 0$ \quad for n odd \numberitem\label{eqn:Troddg}

\item[4.] $\Tr \left( \gamma^{\alpha}\gamma^{\beta} \gamma_5 \right) = 0$ \numberitem\label{eqn:Tr2g5}

\item[5.] $\gamma^{\alpha}\gamma_{\alpha} = d$ \numberitem\label{eqn:2g}

\item[6.] $\gamma^{\alpha}\gamma^{\beta}\gamma_{\alpha} = \left( 2-d \right) \gamma^{\beta}$ \numberitem\label{eqn:3g}

\item[7.] $\gamma^{\alpha}\gamma^{\beta}\gamma^{\gamma}\gamma_{\alpha} = d\gamma^{\beta}\gamma^{\gamma} + 2 \left[ \gamma^{\gamma}, \gamma^{\beta} \right]$ \numberitem\label{eqn:4g}

\item[8.] $\fsl{a}\fsl{b} = a \cdot b$ \numberitem\label{eqn:fsldot}

\end{itemize}
The Dirac matrices obey the Clifford algebra $\lbr \gamma^{\mu}, \gamma^{\nu} \rbr = 2 g^{\mu\nu} \mathds{1}_d$ with $g^{\mu\nu}$ the Minkowski metric in $d$ dimensions
\begin{equation*}
 g^{\mu\nu} = 
\begin{cases}
1 \quad\qquad\qquad \text{for } \mu = \nu = 0 \\
-1 \ \qquad\qquad \text{for } \mu = \nu = 1,2,\ldots,d-1 \\
0 \quad\qquad\qquad \text{for } \mu \neq \nu
\end{cases}
\end{equation*}
In a few of the identities also the fifth gamma matrix appears which is defined as $\gamma_5 = i/4! \ \epsilon_{\mu_1 \dots \mu_4} \gamma^{\mu_1} \cdots \gamma^{\mu_4}$. It is hermitian, traceless and self-inverse: $\gamma_5^2 = \mathds{1}_d$.
\newpage
{\bf Proofs for identities}\\

1. $\mathrm{Tr} \left( \gamma^{\alpha} \gamma^{\beta} \right) = dg^{\alpha\beta}$
\begin{proof}
\begin{equation*}
\mathrm{Tr} \left( \gamma^{\alpha} \gamma^{\beta} \right) = \mathrm{Tr} \left( 2 g^{\alpha\beta} - \gamma^{\beta}\gamma^{\alpha} \right) = 2g^{\alpha\beta} \mathrm{Tr} \left( \mathds{1}_d \right) - \mathrm{Tr} \left( \gamma^{\beta} \gamma^{\alpha} \right) = 2dg^{\alpha\beta} - \mathrm{Tr} \left( \gamma^{\alpha} \gamma^{\beta} \right)
\end{equation*}
\begin{equation*}
\Rightarrow \mathrm{Tr} \left( \gamma^{\alpha} \gamma^{\beta} \right) = dg^{\alpha\beta}
\end{equation*}
\end{proof}

2. $\mathrm{Tr} \left( \gamma^{\alpha}\gamma^{\beta}\gamma^{\gamma}\gamma^{\delta} \right) = d \left( g^{\alpha\beta}g^{\gamma\delta} - g^{\alpha\gamma}g^{\beta\delta} + g^{\alpha\delta}g^{\beta\gamma} \right)$
\begin{proof}
\begin{align*}
\mathrm{Tr} \left( \gamma^{\alpha}\gamma^{\beta}\gamma^{\gamma}\gamma^{\delta} \right) & = \mathrm{Tr} \left( \left( 2g^{\alpha\beta} - \gamma^{\beta}\gamma^{\alpha} \right) \gamma^{\gamma} \gamma^{\delta} \right) = 2g^{\alpha\beta} \mathrm{Tr} \left( \gamma^{\gamma}\gamma^{\delta}\right) - \mathrm{Tr} \left( \gamma^{\beta} \left( 2g^{\alpha\gamma} - \gamma^{\gamma} \gamma^{\alpha} \right) \gamma^{\delta} \right) = & \\
& = 2d g^{\alpha\beta} g^{\gamma\delta} - 2g^{\alpha\gamma} \mathrm{Tr} \left( \gamma^{\beta}\gamma^{\delta} \right) + \mathrm{Tr} \left( \gamma^{\beta} \gamma^{\gamma} \left( 2g^{\alpha\delta} - \gamma^{\delta}\gamma^{\alpha} \right) \right) = & \\
& = 2d \left( g^{\alpha\beta} g^{\gamma\delta} - g^{\alpha\gamma} g^{\beta\delta} \right) + 2g^{\alpha\delta} \mathrm{Tr} \left( \gamma^{\beta}\gamma^{\gamma} \right) - \mathrm{Tr} \left( \gamma^{\beta}\gamma^{\gamma}\gamma^{\delta}\gamma^{\alpha} \right) = & \\
& = 2d \left( g^{\alpha\beta} g^{\gamma\delta} - g^{\alpha\gamma}g^{\beta\delta} + g^{\alpha\delta} g^{\beta\gamma} \right) - \mathrm{Tr} \left( \gamma^{\alpha}\gamma^{\beta}\gamma^{\gamma}\gamma^{\delta} \right) & 
\end{align*}
\begin{equation*}
\Rightarrow \mathrm{Tr} \left( \gamma^{\alpha}\gamma^{\beta}\gamma^{\gamma}\gamma^{\delta} \right) = d \left( g^{\alpha\beta}g^{\gamma\delta} - g^{\alpha\gamma}g^{\beta\delta} + g^{\alpha\delta}g^{\beta\gamma} \right)
\end{equation*}
\end{proof}

3. $\Tr \left( \gamma^{\alpha_1}\gamma^{\alpha_2} \cdots \gamma^{\alpha_n} \right) = 0$ for n odd
\begin{proof}
\begin{align*}
& \Tr \left( \gamma^{\alpha_1}\gamma^{\alpha_2} \cdots \gamma^{\alpha_n} \right) = \Tr \left( \gamma_5^2 \gamma^{\alpha_1}\gamma^{\alpha_2} \cdots \gamma^{\alpha_n} \right) = \left( -1 \right)^n \Tr \left(\gamma_5 \gamma^{\alpha_1}\gamma^{\alpha_2} \cdots \gamma^{\alpha_n} \gamma_5 \right) \overset{\text{n odd}}{=} - \Tr \left( \gamma_5^2 \gamma^{\alpha_1}\gamma^{\alpha_2} \cdots \gamma^{\alpha_n} \right) = & \\
& = - \Tr \left( \gamma^{\alpha_1}\gamma^{\alpha_2} \cdots \gamma^{\alpha_n} \right)
\end{align*}
\begin{equation*}
\Rightarrow \Tr \left( \gamma^{\alpha_1}\gamma^{\alpha_2} \cdots \gamma^{\alpha_n} \right) = 0 \quad \text{ for n odd}
\end{equation*}
\end{proof}

4. $\Tr \left( \gamma^{\alpha}\gamma^{\beta} \gamma_5 \right) = 0$
\begin{proof}
\begin{align*}
&\Tr \left( \gamma^{\alpha}\gamma^{\beta} \gamma_5 \right) \overset{5.}{=} \frac{1}{d}\Tr \left( \gamma_{\mu}\gamma^{\mu}\gamma^{\alpha}\gamma^{\beta} \gamma_5 \right) \overset{\lbr \gamma^{\mu},\gamma_5 \rbr =0}{=}  \frac{\left(-1\right)^3}{d} \Tr \left( \gamma_{\mu}\gamma_5\gamma^{\mu}\gamma^{\alpha}\gamma^{\beta} \right) = & \\
& = -\frac{1}{d} \Tr \left(\gamma^{\mu}\gamma^{\alpha}\gamma^{\beta}\gamma_{\mu}\gamma_5 \right) \overset{7.}{=} -\frac{1}{d} \Tr \left( \left( d\gamma^{\beta}\gamma^{\gamma} + 2 \left[ \gamma^{\gamma}, \gamma^{\beta} \right] \right) \gamma_5 \right) = - \left(1+\frac{2}{d}\right) \Tr \left( \gamma^{\alpha}\gamma^{\beta}\gamma_5 \right) + \frac{2}{d} \Tr \left( \gamma^{\beta}\gamma^{\alpha}\gamma_5 \right) =  & \\
& = - \left(1+\frac{2}{d}\right) \Tr \left( \gamma^{\alpha}\gamma^{\beta}\gamma_5 \right) + \frac{2}{d} \Tr \left( \left( 2g^{\beta\alpha} - \gamma^{\alpha}\gamma^{\beta} \right)\gamma_5 \right) = - \left(1+\frac{4}{d}\right) \Tr \left( \gamma^{\alpha}\gamma^{\beta}\gamma_5 \right) + \frac{4}{d} g^{\beta\alpha} \underbrace{\Tr \left(\gamma_5 \right)}_{=0} = & \\
& = - \left(1+\frac{4}{d} \right) \Tr \left( \gamma^{\alpha}\gamma^{\beta} \gamma_5 \right) &
\end{align*}
\begin{equation*}
\Rightarrow \Tr \left( \gamma^{\alpha}\gamma^{\beta} \gamma_5 \right) = 0
\end{equation*}
\end{proof}

5. $\gamma^{\alpha}\gamma_{\alpha} = d$
\begin{proof}
\begin{equation*}
\gamma^{\alpha}\gamma_{\alpha} = \frac{1}{2} \left( \gamma^{\alpha}\gamma_{\alpha} + \gamma_{\alpha}\gamma^{\alpha} \right) = \frac{1}{2} \lbr \gamma^{\alpha}, \gamma_{\alpha} \rbr = \frac{1}{2} 2 g^{\alpha}_{ \ \alpha} = d
\end{equation*}
\end{proof}

6. $\gamma^{\alpha}\gamma^{\beta}\gamma_{\alpha} = \left( 2-d \right) \gamma^{\beta}$
\begin{proof}
\begin{equation*}
\gamma^{\alpha}\gamma^{\beta}\gamma_{\alpha} = \left( 2g^{\alpha\beta} - \gamma^{\beta}\gamma^{\alpha} \right) \gamma_{\alpha} = \left( 2-d \right) \gamma^{\beta}
\end{equation*}
\end{proof}

7. $\gamma^{\alpha}\gamma^{\beta}\gamma^{\gamma}\gamma_{\alpha} = d\gamma^{\beta}\gamma^{\gamma} + 2 \left[ \gamma^{\gamma}, \gamma^{\beta} \right]$
\begin{proof}
\begin{equation*}
\gamma^{\alpha}\gamma^{\beta}\gamma^{\gamma}\gamma_{\alpha} = \left( 2g^{\alpha\beta} - \gamma^{\beta}\gamma^{\alpha} \right) \gamma^{\gamma}\gamma_{\alpha} \overset{4.}{=} 2 \left( \gamma^{\gamma}\gamma^{\beta} - \gamma^{\beta}\gamma^{\gamma} \right) + d\gamma^{\beta}\gamma^{\gamma} = d\gamma^{\beta}\gamma^{\gamma} + 2 \left[ \gamma^{\gamma}, \gamma^{\beta} \right]		
\end{equation*}
\end{proof}

8. $\fsl{a}\fsl{b} = a \cdot b$
\begin{proof}
\begin{equation*}
\fsl{a} \fsl{b} = a_{\alpha} b_{\beta} \gamma^{\alpha}\gamma^{\beta} = a_{\alpha} b_{\beta} \left( 2 g^{\alpha\beta} - \gamma^{\beta}\gamma^{\alpha} \right) = 2 a \cdot b - \fsl{a}\fsl{b}
\end{equation*}
\begin{equation*}
\Rightarrow \fsl{a}\fsl{b} = a \cdot b
\end{equation*}
\end{proof}


\section{Relation Between Left- \& Right-handed Currents and Axial \& Vector Currents}
\label{app:Currents}
A classical Lagrangian permits symmetries which can be implemented by Lie groups $G$. An element $g \in G$ of a Lie group can be parametrized as $g = \mathrm{exp} \left(i \alpha^a T^a \right)$ where $\alpha^a$ are real parameters and $T^a$ the generators of the Lie group. Noether's theorem predicts a clasically conserved current for each generator of a continuous symmetry. For a field $\phi$ with trafo $\delta\phi = \phi' - \phi = g \phi - \phi \approx \left( 1 + i \alpha^a T^a \right) \phi - \phi = i \alpha^a T^a \phi$ the conserved current can be shown to be
\begin{equation*}
j^{\mu a} = \frac{\partial \mathcal{L}}{\partial \left( \partial_{\mu} \phi \right)} \frac{\partial \delta\phi}{\partial \alpha_a}
\end{equation*} 
The left- and right-handed part $j^{\mu a}_{L/R}$ of a fermionic current are
\begin{equation*}
j^{\mu a}_L = \bar{\psi}_L \gamma^{\mu} T^a \psi_L = \left( P_L \psi \right)^{\dagger} \gamma^0 \gamma^{\mu} T^a P_L \psi \overset{P_L^{\dagger}=P_L}{=} \psi^{\dagger} P_L \gamma^0 \gamma^{\mu} T^a P_L \psi = \bar{\psi} \gamma^{\mu} T^a P_L^2 \psi \overset{P_L^2 = P_L}{=} \bar{\psi} \gamma^{\mu} T^a P_L \psi
\end{equation*}
\begin{equation*}
j^{\mu a}_R = \bar{\psi}_R \gamma^{\mu} T^a \psi_R = \bar{\psi} \gamma^{\mu} T^a P_R \psi
\end{equation*}
where $P_{L/R} = \frac{1}{2} \left( 1 \mp \gamma_5 \right)$ is the left-/right-handed projector. \\
From the left- and right-handed currents we can define axial-vector and vector currents
\begin{equation*}
j^{\mu a} = j^{\mu a}_R + j^{\mu a}_L = \bar{\psi} \gamma^{\mu} T^a \left( P_R + P_L \right) = \bar{\psi} \gamma^{\mu} T^a \psi
\end{equation*}
\begin{equation*}
j^{\mu a}_5 = j^{\mu a}_R - j^{\mu a}_L = \bar{\psi} \gamma^{\mu} T^a \left( P_R - P_L \right) \psi = \bar{\psi} \gamma^{\mu} T^a \gamma_5 \psi
\end{equation*}
Now we can couple the currents to vector fields to obtain interactions. E.g., the vector coupling in QED is given by the Lagrangian
\begin{equation*}
\mathcal{L}_{coupl}^{QED} = e A_ {\mu} j^{\mu} = e A_{\mu} \bar{\psi} \gamma^{\mu} Q_e \psi = -e A_{\mu} \bar{\psi} \gamma^{\mu} \psi 
\end{equation*}
In general, we can couple any linear combination of currents to a vector field as long as the combination is Lorentz and gauge invariant. E.g., the neutral current in the electroweak theory is a superposition of a vector and an axialvector current
\begin{equation*}
\mathcal{L}_{coupl}^{NC} = g Z_{\mu} \left( g_V j^{\mu} - g_A j^{\mu}_5 \right)
\end{equation*}
We can use the above relations to express this coupling in terms of right- and left-handed currents
\begin{align*}
\mathcal{L}_{coupl}^{NC} & = g Z_{\mu} \left( g_V \bar{\psi} \gamma^{\mu} \psi - g_A \bar{\psi} \gamma^{\mu} \gamma_5 \psi \right) = & \\
& = g Z_{\mu} \left( g_V \bar{\psi} \gamma^{\mu} \psi + \frac{g_A}{2} \bar{\psi} \gamma^{\mu} \psi - \frac{g_A}{2} \bar{\psi} \gamma^{\mu} \psi - g_A \bar{\psi} \gamma^{\mu} \gamma_5 \psi + \frac{g_V}{2} \bar{\psi} \gamma^{\mu} \gamma_5 \psi - \frac{g_V}{2} \bar{\psi} \gamma^{\mu} \gamma_5 \psi \right) = & \\
& = g Z_{\mu} \left( \left( g_V + g_A \right) \bar{\psi} \gamma^{\mu} \frac{1}{2} \left( 1 - \gamma_5 \right) \psi + \left( g_V - g_A \right) \bar{\psi} \gamma^{\mu} \frac{1}{2} \left( 1 + \gamma_5 \right) \psi \right) = & \\
& = g Z_{\mu} \left( \left( g_V + g_A \right) \bar{\psi} \gamma^{\mu} P_L \psi + \left( g_V - g_A \right) \bar{\psi} \gamma^{\mu} P_R \psi \right) \equiv g Z_{\mu} \left( g_L j^{\mu}_L + g_R j^{\mu}_R \right)  &
\end{align*}
This gives the following relation between the (axial-)vector and the left-/right-handed couplings
\begin{align*}
&g_L = g_V + g_A &\\
&g_R = g_V - g_A &
\end{align*}