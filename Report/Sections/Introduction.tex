\section{Introduction}
\label{sec:Introduction} 

The introduction goes here.

\begin{equation}
\mathcal{M} = \sum_i d_i \mathrm{Box}_i + \sum_i c_i \mathrm{Triangle}_i + \sum_i b_i \mathrm{Bubble}_i + \sum_i a_i \mathrm{Tadpole}_i + R
\end{equation}

\begin{equation}
R = R_1 + R_2
\end{equation}
where $R_2$ is the $\epsilon$-dimensional contribution of dimensional regularization to the amplitude which is just a rational combination of Lorentz tensors and parameters of the theory, i.e. the couplings or masses of the particles in the theory.


We can decompose any m-point 1-loop function $\bar{A}\bar{q})$ in a numerator $\bar{N}( \bar{q})$ and denominators $\bar{D}_i$
\begin{equation}
\label{eqn:amp}
\bar{A} (\bar{q}) = \frac{\bar{N}(\bar{q})}{\bar{D}_0\bar{D}_1\cdots\bar{D}_{m-1}}, \qquad \bar{D}_i = \left( \bar{q} + \bar{p}_i \right)^2 - m_i^2, p_0 \neq 0
\end{equation}
where $\bar{q}$ is the $d$-dimensional loop momentum and $m_i$ is the mass of the particle corresponding to the propagator with the numerator $D_i$. The $d$-dimensional numerator function $\bar{N}(\bar{q})$ can be split in a 4-dimensional and an $\epsilon$-dimensional part 
\begin{equation}
\bar{N}( \bar{q}) = N (q) + \tilde{N} (\tilde{q}^2,q,\epsilon)
\end{equation}
where only $\tilde{N} (\tilde{q}^2,q,\epsilon)$ is interesting to us because it appears in the definitions of rational terms of the form $R_2$ which are defined as
\begin{equation}
R_2 \equiv \frac{1}{\left( 2\pi \right)^4} \int d^d \bar{q} \frac{\tilde{N} ( \tilde{q}^2,q,\epsilon)}{\bar{D}_0\bar{D}_1\cdots\bar{D}_{m-1}}
\end{equation}
which is the $\epsilon$-dimensional contribution to the amplitude in eqn. \ref{eqn:amp} integrated over the $d$-dimensional loop momentum $\bar{q}$. It can be obtained by splitting the $d$-dimensional Lorentz tensors appearing in the amplitude into a 4-dimensional and an $\epsilon$-dimensional part
\begin{equation}
\bar{A}^{\mu_1\dots\mu_n} = A^{\mu_1\dots\mu_n} + \tilde{A}^{\mu_1\dots\mu_n}.
\end{equation}
To simplify our calculations later, we can establish a few identities for the manipulation of $d$-dimensional momenta. If we contract a $d$-dimensional object with an observable Lorentz tensor (like the momentum of an external particle) only the 4-dimensional part survives, e.g. for a loop momentum $\bar{q}^{\mu}$ and an external momentum $p^{\mu}$
\begin{equation}
\bar{q}\cdot p = q \cdot p.
\end{equation}
Thus, if an amplitude transforms with indices $\mu_1,\dots,\mu_n$ under a Lorentz transformation, the tensors in the amplitude bearing these indices will only appear as 4-dimensional.

e.g. in reference \cite{R2QCD} and \cite{R2QED}.

Since, we want to perform calculations in QED which contains a fermion, we have to extend the Clifford algebra $\lbr \gamma^{\mu},\gamma^{\nu} \rbr = 2g^{\mu\nu} \mathds{1}_4$ to $d$ dimensions. This is straightforward by promoting $\gamma^{\mu} \rightarrow \bar{\gamma}^{\mu}$ and extending the Minkowski metric to d dimensions by adding additional -1 on the diagonal for the extra spatial dimensions. We have
\begin{equation}
\lbr \bar{\gamma}^{\mu},\bar{\gamma}^{\nu} \rbr = 2\bar{g}^{\mu\nu} \mathds{1}_d
\end{equation}
If we want to preserve the Clifford algebra separately in 4 and $\epsilon$ dimensions this implies
\begin{equation}
\lbr \gamma^{\mu},\tilde{\gamma}^{\nu} \rbr = 0
\end{equation}
As opposed to QED the Standard Model is a chiral theory, i.e. it couples differently to left- and right-handed currents. This means that also axial-vector currents appear in the theory which are formulated with the fifth gamma matrix. The extension of $\gamma_5$ to $d$ dimensions is not as straightforward as with the four gamma matrices. This is because chirality is a property of four dimensions.\\
If we also want to impose $\lbr \gamma_5,\gamma^{\mu} \rbr = 0$ for $d \neq 4$, then $\Tr\left( \gamma_5 \gamma^{\alpha}\gamma^{\beta}\gamma^{\gamma}\gamma^{\delta} \right) = 0$  for $d \neq 0,2,4$ which clashes with $\Tr\left( \gamma_5 \gamma^{\alpha}\gamma^{\beta}\gamma^{\gamma}\gamma^{\delta} \right) = -4i \epsilon^{\alpha\beta\gamma\delta}$ \cite{Gamma5}. But the identity is essential in the evaluation of the triangle diagram for the Adler-Bell-Jackiw anomaly. The only definition of $\gamma_5$ which is consistent with the chiral anomaly is the definition of 't Hooft and Veltman \cite{HVgamma5}: $\gamma_5 = i/4! \ \epsilon_{\mu_1 \dots \mu_4} \gamma^{\mu_1} \cdots \gamma^{\mu_4}$. This definition implies
\begin{equation}
\lbr \gamma_5, \gamma^{\mu} \rbr = 0  \text{ and }  \left[ \gamma_5, \tilde{\gamma}^{\mu} \right] = 0.
\end{equation} 



