\section{Pure QED Renormalization}
\label{sec:QEDren}
Explain how to express renormalization constants in terms of scalar integrals.
\subsection{2-point functions}
{\bf Photon self-energy} \\
\begin{align*}
\begin{gathered}
\feynmandiagram [layered layout, horizontal=b to c] {
	a [particle=\(\alpha\)] -- [photon, momentum=\(p\)] b
	  -- [fermion, half left, looseness=1.5, momentum=\(p+q\)] c
	  -- [fermion, half left, looseness=1.5, momentum=\(q\)] b,
	c -- [photon, momentum=\(p\)] d [particle=\(\beta\)] ,
};
\end{gathered} \qquad
& =\int\frac{d^4q}{\left( 2\pi \right)^4} \left( -1 \right) \mathrm{Tr} \lbr ie \gamma^{\alpha} \frac{i \left( \fsl{p}+\fsl{q}+m \right)}{\left( p+q \right)^2 - m^2} ie \gamma^{\beta} \frac{i \left( \fsl{q}+m \right)}{q^2 - m^2} \rbr \equiv \mathcal{A} &
\end{align*}
Let's work on the trace so we can express the numerator of the 2-point function in terms of scalar integrals.
\begin{align*}
& \Tr \lbr \gamma^{\alpha} \left( \fsl{p} + \fsl{q} + m \right) \gamma^{\beta} \left( \fsl{q} + m \right) \rbr = \Tr \lbr m^2 \gamma^{\alpha}\gamma^{\beta} + \gamma^{\alpha} \left( \fsl{p} + \fsl{q} \right) \gamma^{\beta} \fsl{q} \rbr = & \\
& = 4 \lbr m^2 g^{\alpha\beta} + \left( p+q \right)_{\mu} q_{\nu} \left( g^{\alpha\mu}g^{\beta\nu} - g^{\alpha\beta}g^{\mu\nu} + g^{\alpha\nu}g^{\beta\mu} \right) \rbr = & \\
& = 4 \left( m^2 g^{\alpha\beta} + \left( p+q \right)^{\alpha} q^{\beta} - g^{\alpha\beta} \left( p+q \right) \cdot q + g^{\alpha} \left( p+q \right)^{\beta} \right) &
\end{align*}

\begin{align*}
\mathcal{A} & = -4e^2 \int \frac{d^4q}{\left( 2\pi \right)^4} \frac{m^2g^{\alpha\beta} + p^{\alpha}q^{\beta} + q^{\alpha}p^{\beta} + 2 q^{\alpha}q^{\beta} - g^{\alpha\beta} p \cdot q - g^{\alpha\beta} q^2 }{\left( \left( p+q \right)^2 - m^2 \right) \left( q^2 - m^2 \right)} = & \\
& = -\frac{4ie^2}{16\pi^2} \lbr m^2 B_0 g^{\alpha\beta} + 2 p^{\alpha\beta} B_1 + 2 \left( B_{11} p^{\alpha}p^{\beta} + B_{00} g^{\alpha\beta} \right) - g^{\alpha\beta} B_1 p^2 - g^{\alpha\beta} \left( 4 B_{00} + B_{11} p^2 \right) \rbr = & \\ 
& = -\frac{ie^2}{4\pi^2} \lbr g^{\alpha\beta} \left( m^2 B_0 - B_1 p^2 + B_{11} p^2 - 2 B_{00} \right) + 2 p^{\alpha} p^{\beta} \left( B_1 + B_{11} \right) \rbr &
\end{align*}
The arguments of the scalar integrals are suppressed to keep the notation compact. They are the same for all B-functions: $B_i = B_i (p^2,m^2,m^2)$. \\
The expression can be further simplified using identities between the scalar integrals.

{\bf Electron self-energy} \\
\begin{align*}
\begin{gathered}
\feynmandiagram [layered layout, horizontal=b to c] {
	a -- [fermion, momentum=\(p\)] b,
	c -- [photon, half left, looseness=1.5, momentum=\(q\)] b,
	b -- [fermion, momentum=\(p+q\)] c,
	c -- [fermion, momentum=\(p\)] d,
};
\end{gathered} \qquad
& =\int\frac{d^4q}{\left( 2\pi \right)^4} ie \gamma^{\alpha} \frac{i \left( \fsl{p} +\fsl{q}+m \right)}{\left( p+q \right)^2 - m^2} ie \gamma^{\beta} \frac{-i g_{\alpha\beta}}{q^2} =\int\frac{d^4q}{\left( 2\pi \right)^4} \left( -e^2 \right) \gamma^{\alpha} \frac{\left( \fsl{p}+\fsl{q}+m \right)}{\left( p+q \right)^2 - m^2} \gamma_{\alpha} \frac{1}{q^2} \equiv \mathcal{A} &
\end{align*}
With a bit of gamma-matrix algebra the numerator can be written as
\begin{align*}
\gamma^{\beta} \left( \fsl{p} + \fsl{q} + m \right) \gamma_{\beta} = \left( p+q \right)_{\alpha} \gamma^{\beta}\gamma^{\alpha}\gamma_{\beta} + m \gamma^{\beta}\gamma_{\beta} = 4m - 2 \left( \fsl{p} + \fsl{q} \right)
\end{align*}

\begin{align*}
\mathcal{A} & = -e^2 \int \frac{d^4q}{\left( 2\pi \right)^4} \frac{4m - 2 \left( \fsl{p} + \fsl{q} \right)}{\left( \left( p+q \right)^2 - m^2 \right)q^2} = & \\
& = - \frac{ie^2}{16\pi^2} \left[ 4m B_0 - 2\fsl{p} \left( B_0 + B_1 \right) \right] = \frac{-ie^2}{8\pi^2} \left( 2m B_0 - \fsl{p} \left( B_0+B_1 \right) \right) &
\end{align*}
Where the arguments of the B-functions are suppressed again. They are $B_i = B_i(p^2,0,m^2)$


